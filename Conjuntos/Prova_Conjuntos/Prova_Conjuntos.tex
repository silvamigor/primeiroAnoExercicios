\documentclass[a4paper,12pt]{article}

% --------------------------------------------------------
% CODIFICAÇÃO E TIPOGRAFIA
% --------------------------------------------------------
\usepackage[utf8]{inputenc}       % Permite acentuação direta no código-fonte
\usepackage[T1]{fontenc}          % Usa codificação T1 (melhor para português)
\usepackage[brazil]{babel}        % Tradução e hifenização para português
\usepackage{microtype}            % Melhora o espaçamento entre letras/palavras
\usepackage{parskip}              % Espaço entre parágrafos em vez de recuo

% --------------------------------------------------------
% PACOTES MATEMÁTICOS
% --------------------------------------------------------
\usepackage{amsmath}              % Ambientes matemáticos como align, equation
%\usepackage{amssymb}              % Símbolos matemáticos adicionais
\usepackage{amsthm}               % Ambientes de teoremas
\usepackage{mathtools}            % Extensões do amsmath (ex: \coloneqq)
\usepackage{mathrsfs}             % Fonte caligráfica com \mathscr
\usepackage{bbm}                  % Indicador \mathbbm{1}
\usepackage{dsfont}               % Alternativa para conjuntos: \mathds
\usepackage{commath}              % Notações como \abs{}, \norm{}, \eval{}
\usepackage{pgfmath}             % Permite realizar cálculos matemáticos (somas, produtos, números aleatórios, etc.)
\usepackage{siunitx}              % sistema de unidades
\sisetup{group-separator = {\,}} % espaço fino

% --------------------------------------------------------
% FONTE MODERNA (TEXTO E MATEMÁTICA)
% --------------------------------------------------------
%\usepackage{newtxtext}            % Fonte do texto (estilo Times)
%\usepackage{newtxmath}            % Fonte matemática compatível com pacotes AMS
\usepackage[bitstream-charter]{mathdesign} % Fonte elegante e matemática harmoniosa

% --------------------------------------------------------
% AMBIENTES TEÓRICOS PERSONALIZADOS
% --------------------------------------------------------
\theoremstyle{definition}
\newtheorem{definicao}{Definição}       % Definição numerada por seção
\newtheorem{exemplo}{Exemplo}           % Exemplo numerado por seção

%\theoremstyle{plain}
\newtheorem{teorema}{Teorema}           % Teorema numerado por seção
\newtheorem{lema}[teorema]{Lema}                 % Lema com numeração conjunta
\newtheorem{proposicao}[teorema]{Proposição}     % Proposição idem
\newtheorem{corolario}[teorema]{Corolário}       % Corolário idem
\newtheorem{exercicio}[teorema]{Exercício}

%\theoremstyle{remark}
\newtheorem{observacao}[teorema]{Observação}     % Observação com mesmo contador

% --------------------------------------------------------
% TABELAS E FIGURAS
% --------------------------------------------------------
\usepackage{graphicx}             % Inclusão de imagens
\usepackage{float}                % Controle de posicionamento (ex: [H])
\usepackage{caption}              % Personalização de legendas
\usepackage{subcaption}          % Subfiguras com \begin{subfigure}
\usepackage{booktabs}            % Tabelas com qualidade profissional
\usepackage{array}               % Mais opções em colunas de tabelas
\usepackage{multirow}            % Células que ocupam várias linhas
\usepackage{multicol}            % Disposição em colunas múltiplas
\usepackage{tabularx}            % Tabelas com colunas de largura ajustável
\usepackage{longtable}           % Tabelas que quebram página
\usepackage{arydshln}            % Linhas tracejadas horizontais e verticais
\usepackage{makecell}            % Células com múltiplas linhas (e quebra de linha)
\usepackage{diagbox}             % Cabeçalho diagonal em tabelas
\usepackage{pdflscape}           % Gira páginas (modo paisagem)
\newcolumntype{L}[1]{>{\raggedright\let\newline\\\arraybackslash\hspace{0pt}}m{#1}}
\newcolumntype{C}[1]{>{\centering\let\newline\\\arraybackslash\hspace{0pt}}m{#1}}
\newcolumntype{R}[1]{>{\raggedleft\let\newline\\\arraybackslash\hspace{0pt}}m{#1}}

% --------------------------------------------------------
% FORMATAÇÃO E ESTRUTURA
% --------------------------------------------------------
\usepackage{geometry}            % Configuração de margens
\geometry{a4paper, left=1cm, right=1cm, bottom=1.5cm, top=1.5cm, headsep=1cm, footskip=1cm}
\usepackage{titlesec}            % Controle sobre títulos de seções
\usepackage{fancyhdr}            % Cabeçalhos e rodapés personalizados
\pagestyle{fancy}
\fancyhf{}                % Limpa cabeçalho e rodapé
\fancyfoot[R]{\thepage}   % Numeração à direita no rodapé
\renewcommand{\headrulewidth}{0pt} % Remove linha no topo
\usepackage{enumitem}            % Listas com controle de espaçamento e símbolo
\usepackage{tikz}                % Desenho de gráficos vetoriais (diagramas, etc.)
\usetikzlibrary{matrix,arrows,patterns,snakes,decorations.pathreplacing,3d,arrows.meta,calc}
\usepackage{etoolbox}             % Permite lógica condicional e manipulação de comandos
\usepackage{background}           % Permite adicionar conteúdo fixo no fundo da página (ex: linhas, marcas d'água)
\SetBgContents{} % Remove conteúdo padrão ("DRAFT")
\definecolor{background-color}{RGB}{233 227 206}

% --------------------------------------------------------
% GRÁFICOS E PLOTAGENS
% --------------------------------------------------------
\usepackage{pgfplots}             % Criação de gráficos vetoriais em 2D e 3D
\pgfplotsset{compat=1.18}         % Compatibilidade com versão atual do pacote
\usepackage{currfile} % Permite saber qual é o caminho completo do arquivo atual,

% --------------------------------------------------------
% LINKS E CORES
% --------------------------------------------------------
\usepackage{xcolor}              % Definição de cores (texto, links, etc.)
\usepackage[hidelinks]{hyperref} % Links sem moldura colorida no PDF
\usepackage{url}                 % Quebra automática de URLs
% Dica: se quiser links coloridos, use:
% \usepackage[colorlinks=true, linkcolor=blue, citecolor=blue, urlcolor=blue]{hyperref}

% --------------------------------------------------------
% CAIXAS E DESTAQUES VISUAIS
% --------------------------------------------------------
\usepackage[most]{tcolorbox}      % Criação de caixas coloridas e personalizadas para destaques, avisos, exemplos, etc.

% --------------------------------------------------------
% CÓDIGOS E ALGORITMOS
% --------------------------------------------------------
\usepackage{listings}            % Inclusão de código-fonte com destaque
\usepackage{algorithm}           % Ambiente de algoritmo
\usepackage[noend]{algpseudocode} % Sintaxe tipo pseudocódigo


% --------------------------------------------------------
% COMANDOS
% --------------------------------------------------------

\newcommand{\bbN}{\mathbb{N}}
\newcommand{\bbZ}{\mathbb{Z}}
\newcommand{\bbQ}{\mathbb{Q}}
\newcommand{\bbR}{\mathbb{R}}

\newcommand{\assunto}{CONJUNTOS}
\newcommand{\atividade}{prova} % prova ou lista
\newcommand{\NumQObj}{5}
\newcommand{\NumQDisc}{1}
\pgfmathparse{int(\NumQObj + \NumQDisc)}
\edef\NumTotalQuestoes{\pgfmathresult}
\newcommand{\questaoDiscursiva}{Exercício \NumTotalQuestoes}
\newcommand{\ValorQObj}{1}
\newcommand{\ValorQDisc}{2}
\pgfmathparse{int(\NumQObj * \ValorQObj + \NumQDisc * \ValorQDisc)}
\edef\ValorProva{\pgfmathresult}
\newcommand{\tipo}{dupla} % dupla ou individual
\newcommand{\calculadora}{nao} % sim ou nao
\newcommand{\turma}{ELETROELETRÔNICA 1.\textordmasculine\ ANO}
\newcommand{\numProva}{B}
\newcommand{\professor}{Igor Martins Silva}
\newcommand{\bimestre}{3}
\newcommand{\ano}{2025}
\newcommand{\mesTexto}{setembro}
\newcommand{\mesNum}{09}
\newcommand{\dia}{18}
\newcommand{\dataTexto}{\dia\ de \mesTexto\ de \ano}
\newcommand{\dataNun}{\dia/\mesNum/\ano}

\ifdefstring{\atividade}{prova}{
    \newcommand{\titulo}{Avaliação de Matemática}
}{
    \newcommand{\titulo}{Lista de exercícios de Matemática}
}

\newcommand{\subtitulo}{
    %
    \begin{center}
        CEFET -- Contagem \\[3pt]
        \titulo \ -- \bimestre.\textordmasculine\ Bimestre de \ano \\[3pt]
        \professor
    \end{center}
    %
    \begin{center}
        \vspace{15pt}
        \begin{tabular}{|C{210pt}|C{70pt}|C{210pt}|}
            \hline
            \textbf{ASSUNTO} &
            \textbf{DATA} &
            \textbf{TURMA} \\
            \hline
            \assunto & \dataNun & \turma \\
            \hline
        \end{tabular}
    \end{center}
    %
    \vspace{15pt}
}

\newcommand{\valorquestao}{}

\newenvironment{exercicioBanco}[1][]{%
    \ifdefstring{\atividade}{prova}{%
        \begin{exercicio}[#1]%
    }{%
        \begin{exercicio}%
    }%
}{%
    \end{exercicio}
}

\ifdefstring{\atividade}{lista}{
    \pagecolor{background-color}
}{
    
}



\hypersetup{
    pdfauthor={\professor},
    pdftitle={\titulo -- \assunto},
}

\title{\titulo -- \assunto}
\author{\professor}
\date{\data}

%************* INÍCIO DO DOCUMENTO *************
\begin{document}
    \NoBgThispage
    \pagenumbering{alph}
    %
        
    \phantomsection
    \addcontentsline{toc}{section}{Capa}
    %%%%%%%%%%%%%%%%%%%%%%%%%%%%%%%%%
%
%       CAPA
%
%%%%%%%%%%%%%%%%%%%%%%%%%%%%%%%%%

\edef\pathCapa{\currfiledir}

\begin{titlepage}
    \begin{center}
        \includegraphics[width=3cm]{\pathCapa Logo_Cefet.png}
    \end{center}

    \begin{center}
        SERVIÇO PÚBLICO FEDERAL -- MINISTÉRIO DA EDUCAÇÃO \\
        CENTRO FEDERAL DE EDUCAÇÃO TECNOLÓGICA DE MINAS GERAIS \\
        CAMPUS CONTAGEM \\
        AVALIAÇÃO DE MATEMÁTICA -- \bimestre.\textordmasculine\ BIMESTRE DE \ano
    \end{center}
    
    \vspace{0.3cm}
    
    \noindent\textbf{ASSUNTO}: \assunto \hfill \textbf{TURMA}: \turma
    
    \begin{center}
        \ifdefstring{\calculadora}{nao}{%
            \textbf{NÃO É} PERMITIDO O USO DE CALCULADORA
        }{%
            \textbf{É} PERMITIDO O USO DE CALCULADORA
        }
    \end{center}
    
    \begin{tcolorbox}[width=\linewidth,colback=white]
        \textbf{PROFESSOR}: \professor \hfill
        \textbf{DATA}: \dataTexto \hfill
        \textbf{VALOR}: \ValorProva\ pontos
        \vspace{5pt}
        
        \ifdefstring{\tipo}{dupla}{%
            \begin{minipage}[t]{0.75\linewidth}
                \textbf{ALUNOS (AS)}: \makebox[0.85\linewidth]{\hrulefill} \\[5pt]
                \makebox[1.05\linewidth]{\hspace{78pt}\hrulefill}
            \end{minipage}%
            \hfill
            \begin{minipage}[t]{0.15\linewidth}
                \vspace*{\fill}
                \begin{tcolorbox}[width=0.8\linewidth, height=20pt, sharp corners, boxrule=0.5pt, colback=white, colframe=black]
                    \centering
                    % Espaço para a nota
                    \vspace*{\fill}
                \end{tcolorbox}
            \end{minipage}
        }{%
            \begin{minipage}[t]{0.75\linewidth}
                \textbf{ALUNO (A)}: \makebox[0.85\linewidth]{\hrulefill}
            \end{minipage}%
            \hfill
            \begin{minipage}[t]{0.15\linewidth}
                \begin{tcolorbox}[width=0.8\linewidth, height=20pt, sharp corners, boxrule=0.5pt, colback=white, colframe=black]
                    \centering
                    % Espaço para a nota
                    \vspace*{\fill}
                \end{tcolorbox}
            \end{minipage}
        }
    \end{tcolorbox}
    
    \noindent\textbf{DURAÇÃO}: máximo de 100 minutos.
    
    \begin{center}
        INSTRUÇÕES
    \end{center}
    
    \begin{itemize}[left=0pt, topsep=2pt, itemsep=2pt, parsep=0pt]
    \small
        \item Esta prova é composta por \textbf{\NumTotalQuestoes\ questões}, sendo \NumQObj\ objetivas e \NumQDisc\ discursiva.
        %
        \item Cada questão objetiva vale \ValorQObj\ ponto, e a questão discursiva vale \ValorQDisc\ pontos.
        %
        \item As respostas das questões objetivas devem ser marcadas no gabarito com caneta azul ou preta. Questões marcadas a lápis ou com rasura receberão nota zero.
        %
        \item Não é necessária justificativa nas questões objetivas; apenas a alternativa correta será considerada.
        %
        \item Na questão discursiva, é necessário explicar adequadamente seu raciocínio, pois a argumentação também será avaliada.
        %
        \item A questão discursiva deve ser respondida no verso desta folha, onde também está o gabarito das questões objetivas. Somente esta folha será recolhida para correção.
        %
        \item A folha com os enunciados e a folha de rascunho não devem ser entregues.
        %
        \item A prova é em \tipo\ e sem consulta. Alunos que copiarem respostas de colegas ou utilizarem meios indevidos para obter vantagem, como o uso de celulares, terão sua prova anulada, sem direito à segunda chamada.
        %
        \item Compreender o enunciado e os termos de cada questão faz parte da avaliação.
    \end{itemize}
    
    \noindent\rule{\linewidth}{1pt}  % linha horizontal fina
    
    \begin{center}
        \textbf{GABARITO}
    \end{center}
    
    \renewcommand{\arraystretch}{1.1} % espaçamento vertical
    \setlength{\tabcolsep}{10pt}
    
    \begin{center}
        \begin{tabular}{|*{5}{>{\centering\arraybackslash}p{0.7cm}|}}
            \hline
            \textbf{1} & \textbf{2} & \textbf{3} & \textbf{4} & \textbf{5} \\
            \hline
            A & A & A & A & A \\
            \hline
            B & B & B & B & B \\
            \hline
            C & C & C & C & C \\
            \hline
            D & D & D & D & D \\
            \hline
            E & E & E & E & E \\
            \hline
        \end{tabular}
    \end{center}
\end{titlepage}

    \newpage
        
    \phantomsection
    \addcontentsline{toc}{section}{Resposta para a questão discursiva}
    %%%%%%%%%%%%%%%%%%%%%%%%%%%%%%%%%
%
%       RESPOSTA QUESTÃO DISCURSIVA
%
%%%%%%%%%%%%%%%%%%%%%%%%%%%%%%%%%

\thispagestyle{empty}
\vspace*{1pt}
\backgroundsetup{
  position=current page.center,
  angle=0,
  scale=1,
  opacity=1,
  contents={%
    \begin{tikzpicture}[
      normal lines/.style={gray, very thin},
      every node/.append style={black, align=center, opacity=1}
    ]
    \foreach \y in {0.71,1.41,...,25.56}
      \draw[normal lines] (0,\y) -- (8.5in,\y);
    \draw[normal lines] (1.25in,0) -- (1.25in,11in);
    \node (t) [font=\large, anchor=south, yshift=10pt] at ($(0,25.56)!1/2!(8.5in,25.56)$) {Resposta da Questão Discursiva};
    \node (d) [font=\large, anchor=south west, xshift=1.3em] at (0,24.65) {\textbf{\questaoDiscursiva}};
    \end{tikzpicture}
  }
}

    \addtocounter{page}{1}
    \newpage
    %
        
    \NoBgThispage
    \pagenumbering{arabic}
    %
        
    \subtitulo
        
    %
    \begin{center}
        QUESTÕES OBJETIVAS
    \end{center}
    %
    
    \vspace{15pt}
    \newcommand{\modo}{objetivo} % objetivo ou discursivo
    
    \phantomsection
    \addcontentsline{toc}{section}{Exercício 1}
    %%%%%%%%%%%%%%%%%%%%%%%%%%%%%%%%%
%
%       EXERCÍCIO
%
%%%%%%%%%%%%%%%%%%%%%%%%%%%%%%%%%

\ifdefstring{\atividade}{prova}{%
    \ifdefstring{\modo}{objetivo}{%
        \renewcommand{\valorquestao}{\ValorQObj\ ponto}
    }{%
        \renewcommand{\valorquestao}{\ValorQDisc\ pontos}
    }%
}%

\begin{exercicioBanco}[\valorquestao]
TEXTO DA QUESTÃO.

% Define as alternativas
\newcommand{\alternativas}{%
    \begin{center}
        \begin{tabularx}{\textwidth}{XXXXX}
            (a) Alternativa A. &
            (b) Alternativa B. &
            (c) Alternativa C. &
            (d) Alternativa D. &
            (e) Alternativa E.
        \end{tabularx}
    \end{center}
}

% Define a resposta correta
\newcommand{\resposta}{X}

% Lógica condicional para exibição
\ifdefstring{\atividade}{lista}{%
    \alternativas
    \vspace{0.5em}
    
    \noindent\textbf{Resposta:} letra \textbf{\resposta}.
}{%
    \ifdefstring{\modo}{objetivo}{%
        \alternativas
    }{%
        % modo = discursiva → não mostra alternativas
    }
}
\end{exercicioBanco}


    \vspace{15pt}
    
    \phantomsection
    \addcontentsline{toc}{section}{Exercício 2}
    %%%%%%%%%%%%%%%%%%%%%%%%%%%%%%%%%
%
%       EXERCÍCIO
%
%%%%%%%%%%%%%%%%%%%%%%%%%%%%%%%%%

\ifdefstring{\atividade}{prova}{%
    \ifdefstring{\modo}{objetivo}{%
        \renewcommand{\valorquestao}{\ValorQObj\ ponto}
    }{%
        \renewcommand{\valorquestao}{\ValorQDisc\ pontos}
    }%
}%

\begin{exercicioBanco}[\valorquestao]
TEXTO DA QUESTÃO.

% Define as alternativas
\newcommand{\alternativas}{%
    \begin{center}
        \begin{tabularx}{\textwidth}{XXXXX}
            (a) Alternativa A. &
            (b) Alternativa B. &
            (c) Alternativa C. &
            (d) Alternativa D. &
            (e) Alternativa E.
        \end{tabularx}
    \end{center}
}

% Define a resposta correta
\newcommand{\resposta}{X}

% Lógica condicional para exibição
\ifdefstring{\atividade}{lista}{%
    \alternativas
    \vspace{0.5em}
    
    \noindent\textbf{Resposta:} letra \textbf{\resposta}.
}{%
    \ifdefstring{\modo}{objetivo}{%
        \alternativas
    }{%
        % modo = discursiva → não mostra alternativas
    }
}
\end{exercicioBanco}


    \vspace{15pt}
    
    \phantomsection
    \addcontentsline{toc}{section}{Exercício 3}
    %%%%%%%%%%%%%%%%%%%%%%%%%%%%%%%%%
%
%       EXERCÍCIO
%
%%%%%%%%%%%%%%%%%%%%%%%%%%%%%%%%%

\ifdefstring{\atividade}{prova}{%
    \ifdefstring{\modo}{objetivo}{%
        \renewcommand{\valorquestao}{\ValorQObj\ ponto}
    }{%
        \renewcommand{\valorquestao}{\ValorQDisc\ pontos}
    }%
}%

\begin{exercicioBanco}[\valorquestao]
TEXTO DA QUESTÃO.

% Define as alternativas
\newcommand{\alternativas}{%
    \begin{center}
        \begin{tabularx}{\textwidth}{XXXXX}
            (a) Alternativa A. &
            (b) Alternativa B. &
            (c) Alternativa C. &
            (d) Alternativa D. &
            (e) Alternativa E.
        \end{tabularx}
    \end{center}
}

% Define a resposta correta
\newcommand{\resposta}{X}

% Lógica condicional para exibição
\ifdefstring{\atividade}{lista}{%
    \alternativas
    \vspace{0.5em}
    
    \noindent\textbf{Resposta:} letra \textbf{\resposta}.
}{%
    \ifdefstring{\modo}{objetivo}{%
        \alternativas
    }{%
        % modo = discursiva → não mostra alternativas
    }
}
\end{exercicioBanco}


    \vspace{15pt}
    
    \phantomsection
    \addcontentsline{toc}{section}{Exercício 4}
    %%%%%%%%%%%%%%%%%%%%%%%%%%%%%%%%%
%
%       EXERCÍCIO
%
%%%%%%%%%%%%%%%%%%%%%%%%%%%%%%%%%

\ifdefstring{\atividade}{prova}{%
    \ifdefstring{\modo}{objetivo}{%
        \renewcommand{\valorquestao}{\ValorQObj\ ponto}
    }{%
        \renewcommand{\valorquestao}{\ValorQDisc\ pontos}
    }%
}%

\begin{exercicioBanco}[\valorquestao]
TEXTO DA QUESTÃO.

% Define as alternativas
\newcommand{\alternativas}{%
    \begin{center}
        \begin{tabularx}{\textwidth}{XXXXX}
            (a) Alternativa A. &
            (b) Alternativa B. &
            (c) Alternativa C. &
            (d) Alternativa D. &
            (e) Alternativa E.
        \end{tabularx}
    \end{center}
}

% Define a resposta correta
\newcommand{\resposta}{X}

% Lógica condicional para exibição
\ifdefstring{\atividade}{lista}{%
    \alternativas
    \vspace{0.5em}
    
    \noindent\textbf{Resposta:} letra \textbf{\resposta}.
}{%
    \ifdefstring{\modo}{objetivo}{%
        \alternativas
    }{%
        % modo = discursiva → não mostra alternativas
    }
}
\end{exercicioBanco}


    \vspace{15pt}
    
    \phantomsection
    \addcontentsline{toc}{section}{Exercício 5}
    %%%%%%%%%%%%%%%%%%%%%%%%%%%%%%%%%
%
%       EXERCÍCIO
%
%%%%%%%%%%%%%%%%%%%%%%%%%%%%%%%%%

\ifdefstring{\atividade}{prova}{%
    \ifdefstring{\modo}{objetivo}{%
        \renewcommand{\valorquestao}{\ValorQObj\ ponto}
    }{%
        \renewcommand{\valorquestao}{\ValorQDisc\ pontos}
    }%
}%

\begin{exercicioBanco}[\valorquestao]
TEXTO DA QUESTÃO.

% Define as alternativas
\newcommand{\alternativas}{%
    \begin{center}
        \begin{tabularx}{\textwidth}{XXXXX}
            (a) Alternativa A. &
            (b) Alternativa B. &
            (c) Alternativa C. &
            (d) Alternativa D. &
            (e) Alternativa E.
        \end{tabularx}
    \end{center}
}

% Define a resposta correta
\newcommand{\resposta}{X}

% Lógica condicional para exibição
\ifdefstring{\atividade}{lista}{%
    \alternativas
    \vspace{0.5em}
    
    \noindent\textbf{Resposta:} letra \textbf{\resposta}.
}{%
    \ifdefstring{\modo}{objetivo}{%
        \alternativas
    }{%
        % modo = discursiva → não mostra alternativas
    }
}
\end{exercicioBanco}


    \vspace{15pt}
    
    %
    \begin{center}
        QUESTÃO DISCURSIVA 
    \end{center}
    %
    \vspace{15pt}
    
    \renewcommand{\modo}{discursivo} % objetivo ou discursivo
    \phantomsection
    \addcontentsline{toc}{section}{Exercício 6}
    %%%%%%%%%%%%%%%%%%%%%%%%%%%%%%%%%
%
%       EXERCÍCIO
%
%%%%%%%%%%%%%%%%%%%%%%%%%%%%%%%%%

\ifdefstring{\atividade}{prova}{%
    \ifdefstring{\modo}{objetivo}{%
        \renewcommand{\valorquestao}{\ValorQObj\ ponto}
    }{%
        \renewcommand{\valorquestao}{\ValorQDisc\ pontos}
    }%
}%

\begin{exercicioBanco}[\valorquestao]
TEXTO DA QUESTÃO.

% Define as alternativas
\newcommand{\alternativas}{%
    \begin{center}
        \begin{tabularx}{\textwidth}{XXXXX}
            (a) Alternativa A. &
            (b) Alternativa B. &
            (c) Alternativa C. &
            (d) Alternativa D. &
            (e) Alternativa E.
        \end{tabularx}
    \end{center}
}

% Define a resposta correta
\newcommand{\resposta}{X}

% Lógica condicional para exibição
\ifdefstring{\atividade}{lista}{%
    \alternativas
    \vspace{0.5em}
    
    \noindent\textbf{Resposta:} letra \textbf{\resposta}.
}{%
    \ifdefstring{\modo}{objetivo}{%
        \alternativas
    }{%
        % modo = discursiva → não mostra alternativas
    }
}
\end{exercicioBanco}


    \vspace{10pt}
    \newpage
    \phantomsection
    \addcontentsline{toc}{section}{Rascunho}
    %%%%%%%%%%%%%%%%%%%%%%%%%%%%%%%%%
%
%       RASCUNHO
%
%%%%%%%%%%%%%%%%%%%%%%%%%%%%%%%%%

\thispagestyle{empty}

\begin{center}
    \textbf{RASCUNHO}
\end{center}

    
\end{document}
%*************** FINAL DO DOCUMENTO ***************
