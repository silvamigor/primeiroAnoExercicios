%%%%%%%%%%%%%%%%%%%%%%%%%%%%%%%%%
%
%       CAPA
%
%%%%%%%%%%%%%%%%%%%%%%%%%%%%%%%%%

\edef\pathCapa{\currfiledir}

\begin{titlepage}
    \begin{center}
        \includegraphics[width=3cm]{\pathCapa Logo_Cefet.png}
    \end{center}

    \begin{center}
        SERVIÇO PÚBLICO FEDERAL -- MINISTÉRIO DA EDUCAÇÃO \\
        CENTRO FEDERAL DE EDUCAÇÃO TECNOLÓGICA DE MINAS GERAIS \\
        CAMPUS CONTAGEM \\
        AVALIAÇÃO DE MATEMÁTICA -- \bimestre.\textordmasculine\ BIMESTRE DE \ano
    \end{center}
    
    \vspace{0.3cm}
    
    \noindent\textbf{ASSUNTO}: \assunto \hfill \textbf{TURMA}: \turma
    
    \begin{center}
        \ifdefstring{\calculadora}{nao}{%
            \textbf{NÃO É} PERMITIDO O USO DE CALCULADORA
        }{%
            \textbf{É} PERMITIDO O USO DE CALCULADORA
        }
    \end{center}
    
    \begin{tcolorbox}[width=\linewidth,colback=white]
        \textbf{PROFESSOR}: \professor \hfill
        \textbf{DATA}: \dataTexto \hfill
        \textbf{VALOR}: \ValorProva\ pontos
        \vspace{5pt}
        
        \ifdefstring{\tipo}{dupla}{%
            \begin{minipage}[t]{0.75\linewidth}
                \textbf{ALUNOS (AS)}: \makebox[0.85\linewidth]{\hrulefill} \\[5pt]
                \makebox[1.05\linewidth]{\hspace{78pt}\hrulefill}
            \end{minipage}%
            \hfill
            \begin{minipage}[t]{0.15\linewidth}
                \vspace*{\fill}
                \begin{tcolorbox}[width=0.8\linewidth, height=20pt, sharp corners, boxrule=0.5pt, colback=white, colframe=black]
                    \centering
                    % Espaço para a nota
                    \vspace*{\fill}
                \end{tcolorbox}
            \end{minipage}
        }{%
            \begin{minipage}[t]{0.75\linewidth}
                \textbf{ALUNO (A)}: \makebox[0.85\linewidth]{\hrulefill}
            \end{minipage}%
            \hfill
            \begin{minipage}[t]{0.15\linewidth}
                \begin{tcolorbox}[width=0.8\linewidth, height=20pt, sharp corners, boxrule=0.5pt, colback=white, colframe=black]
                    \centering
                    % Espaço para a nota
                    \vspace*{\fill}
                \end{tcolorbox}
            \end{minipage}
        }
    \end{tcolorbox}
    
    \noindent\textbf{DURAÇÃO}: máximo de 100 minutos. \hfill \boxed{\numProva}
    
    \begin{center}
        INSTRUÇÕES
    \end{center}
    
    \begin{itemize}[left=0pt, topsep=2pt, itemsep=2pt, parsep=0pt]
    \small
        \item Esta prova é composta por \textbf{\NumTotalQuestoes\ questões}, sendo \NumQObj\ objetivas e \NumQDisc\ discursiva.
        %
        \item Cada questão objetiva vale \ValorQObj\ ponto, e a questão discursiva vale \ValorQDisc\ pontos.
        %
        \item As respostas das questões objetivas devem ser marcadas no gabarito com caneta azul ou preta. Questões marcadas a lápis ou com rasura receberão nota zero.
        %
        \item Não é necessária justificativa nas questões objetivas; apenas a alternativa correta será considerada.
        %
        \item Na questão discursiva, é necessário explicar adequadamente seu raciocínio, pois a argumentação também será avaliada.
        %
        \item A questão discursiva deve ser respondida no verso desta folha, onde também está o gabarito das questões objetivas. Somente esta folha será recolhida para correção.
        %
        \item A folha com os enunciados e a folha de rascunho também devem ser entregues.
        %
        \item A prova é em \tipo\ e sem consulta. Alunos que copiarem respostas de colegas ou utilizarem meios indevidos para obter vantagem, como o uso de celulares, terão sua prova anulada, sem direito à segunda chamada.
        %
        \item Compreender o enunciado e os termos de cada questão faz parte da avaliação.
    \end{itemize}
    
    \noindent\rule{\linewidth}{1pt}  % linha horizontal fina
    
    \begin{center}
        \textbf{GABARITO}
    \end{center}
    
    \renewcommand{\arraystretch}{1.1} % espaçamento vertical
    \setlength{\tabcolsep}{10pt}
    
    \begin{center}
        \begin{tabular}{|*{5}{>{\centering\arraybackslash}p{0.7cm}|}}
            \hline
            \textbf{1} & \textbf{2} & \textbf{3} & \textbf{4} & \textbf{5} \\
            \hline
            A & A & A & A & A \\
            \hline
            B & B & B & B & B \\
            \hline
            C & C & C & C & C \\
            \hline
            D & D & D & D & D \\
            \hline
            E & E & E & E & E \\
            \hline
        \end{tabular}
    \end{center}
\end{titlepage}
