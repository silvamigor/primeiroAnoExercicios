%%%%%%%%%%%%%%%%%%%%%%%%%%%%%%%%%
%
%       EXERCÍCIO
%
%%%%%%%%%%%%%%%%%%%%%%%%%%%%%%%%%

\ifdefstring{\atividade}{prova}{%
    \ifdefstring{\modo}{objetivo}{%
        \renewcommand{\valorquestao}{\ValorQObj\ ponto}
    }{%
        \renewcommand{\valorquestao}{\ValorQDisc\ pontos}
    }%
}%

\begin{exercicioBanco}[\valorquestao]
Há uma lenda que credita a invenção do xadrez a um brâmane de uma corte indiana, que, atendendo a um pedido do rei, inventou o jogo para demonstrar o valor da inteligência. O rei, encantado com o invento, ofereceu ao brâmane a escolha de uma recompensa. De acordo com essa lenda, o inventor do jogo de xadrez pediu ao rei que a recompensa fosse paga em grãos de arroz da seguinte maneira: 1 grão para a casa 1 do tabuleiro, 2 grãos para a casa 2, 4 para a casa 3, 8 para a casa 4 e assim sucessivamente. Ou seja, a quantidade de grãos para cada casa do tabuleiro correspondia ao dobro da quantidade da casa imediatamente anterior. Escreva uma função \(f\) que expresse a quantidade de grãos de arroz em função do número \(x\) da casa do tabuleiro.

% Define as alternativas
\newcommand{\alternativas}{%
    \begin{center}
        \begin{tabularx}{\textwidth}{XXXXX}
            (a) \(f(x) = x^{64}\). &
            (b) \(f(x) = 2^{x}\). &
            (c) \(f(x) = 64x\). &
            (d) \(f(x) = 2^{x+1}\). &
            (e) \(f(x) = 2^{x-1}\).
        \end{tabularx}
    \end{center}
}

% Define a resposta correta
\newcommand{\resposta}{E}

% Lógica condicional para exibição
\ifdefstring{\atividade}{lista}{%
    \alternativas
    \vspace{0.5em}
    
    \noindent\textbf{Resposta:} letra \textbf{\resposta}.
}{%
    \ifdefstring{\modo}{objetivo}{%
        \alternativas
    }{%
        % modo = discursiva → não mostra alternativas
    }
}
\end{exercicioBanco}

\textbf{Observação}. A quantidade de \(2^{63}\) grãos de arroz equivalem a aproximadamente \(46\) trilhões de pacotes de 5 kg. Pensado esse número como distância, ele corresponde a aproximadamente \(975\) anos-luz, o que daria umas \(230\) bilhões de voltas na Terra.

