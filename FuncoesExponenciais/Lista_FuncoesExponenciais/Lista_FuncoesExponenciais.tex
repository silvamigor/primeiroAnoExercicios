\documentclass[a4paper,12pt]{article}

% --------------------------------------------------------
% CODIFICAÇÃO E TIPOGRAFIA
% --------------------------------------------------------
\usepackage[utf8]{inputenc}       % Permite acentuação direta no código-fonte
\usepackage[T1]{fontenc}          % Usa codificação T1 (melhor para português)
\usepackage[brazil]{babel}        % Tradução e hifenização para português
\usepackage{microtype}            % Melhora o espaçamento entre letras/palavras
\usepackage{parskip}              % Espaço entre parágrafos em vez de recuo

% --------------------------------------------------------
% PACOTES MATEMÁTICOS
% --------------------------------------------------------
\usepackage{amsmath}              % Ambientes matemáticos como align, equation
%\usepackage{amssymb}              % Símbolos matemáticos adicionais
\usepackage{amsthm}               % Ambientes de teoremas
\usepackage{mathtools}            % Extensões do amsmath (ex: \coloneqq)
\usepackage{mathrsfs}             % Fonte caligráfica com \mathscr
\usepackage{bbm}                  % Indicador \mathbbm{1}
\usepackage{dsfont}               % Alternativa para conjuntos: \mathds
\usepackage{commath}              % Notações como \abs{}, \norm{}, \eval{}
\usepackage{pgfmath}             % Permite realizar cálculos matemáticos (somas, produtos, números aleatórios, etc.)
\usepackage{siunitx}              % sistema de unidades
\sisetup{group-separator = {\,}} % espaço fino

% --------------------------------------------------------
% FONTE MODERNA (TEXTO E MATEMÁTICA)
% --------------------------------------------------------
%\usepackage{newtxtext}            % Fonte do texto (estilo Times)
%\usepackage{newtxmath}            % Fonte matemática compatível com pacotes AMS
\usepackage[bitstream-charter]{mathdesign} % Fonte elegante e matemática harmoniosa

% --------------------------------------------------------
% AMBIENTES TEÓRICOS PERSONALIZADOS
% --------------------------------------------------------
\theoremstyle{definition}
\newtheorem{definicao}{Definição}       % Definição numerada por seção
\newtheorem{exemplo}{Exemplo}           % Exemplo numerado por seção

%\theoremstyle{plain}
\newtheorem{teorema}{Teorema}           % Teorema numerado por seção
\newtheorem{lema}[teorema]{Lema}                 % Lema com numeração conjunta
\newtheorem{proposicao}[teorema]{Proposição}     % Proposição idem
\newtheorem{corolario}[teorema]{Corolário}       % Corolário idem
\newtheorem{exercicio}[teorema]{Exercício}

%\theoremstyle{remark}
\newtheorem{observacao}[teorema]{Observação}     % Observação com mesmo contador

% --------------------------------------------------------
% TABELAS E FIGURAS
% --------------------------------------------------------
\usepackage{graphicx}             % Inclusão de imagens
\usepackage{float}                % Controle de posicionamento (ex: [H])
\usepackage{caption}              % Personalização de legendas
\usepackage{subcaption}          % Subfiguras com \begin{subfigure}
\usepackage{booktabs}            % Tabelas com qualidade profissional
\usepackage{array}               % Mais opções em colunas de tabelas
\usepackage{multirow}            % Células que ocupam várias linhas
\usepackage{multicol}            % Disposição em colunas múltiplas
\usepackage{tabularx}            % Tabelas com colunas de largura ajustável
\usepackage{longtable}           % Tabelas que quebram página
\usepackage{arydshln}            % Linhas tracejadas horizontais e verticais
\usepackage{makecell}            % Células com múltiplas linhas (e quebra de linha)
\usepackage{diagbox}             % Cabeçalho diagonal em tabelas
\usepackage{pdflscape}           % Gira páginas (modo paisagem)
\newcolumntype{L}[1]{>{\raggedright\let\newline\\\arraybackslash\hspace{0pt}}m{#1}}
\newcolumntype{C}[1]{>{\centering\let\newline\\\arraybackslash\hspace{0pt}}m{#1}}
\newcolumntype{R}[1]{>{\raggedleft\let\newline\\\arraybackslash\hspace{0pt}}m{#1}}

% --------------------------------------------------------
% FORMATAÇÃO E ESTRUTURA
% --------------------------------------------------------
\usepackage{geometry}            % Configuração de margens
\geometry{a4paper, left=1cm, right=1cm, bottom=1.5cm, top=1.5cm, headsep=1cm, footskip=1cm}
\usepackage{titlesec}            % Controle sobre títulos de seções
\usepackage{fancyhdr}            % Cabeçalhos e rodapés personalizados
\pagestyle{fancy}
\fancyhf{}                % Limpa cabeçalho e rodapé
\fancyfoot[R]{\thepage}   % Numeração à direita no rodapé
\renewcommand{\headrulewidth}{0pt} % Remove linha no topo
\usepackage{enumitem}            % Listas com controle de espaçamento e símbolo
\usepackage{tikz}                % Desenho de gráficos vetoriais (diagramas, etc.)
\usetikzlibrary{matrix,arrows,patterns,snakes,decorations.pathreplacing,3d,arrows.meta,calc}
\usepackage{etoolbox}             % Permite lógica condicional e manipulação de comandos
\usepackage{background}           % Permite adicionar conteúdo fixo no fundo da página (ex: linhas, marcas d'água)
\SetBgContents{} % Remove conteúdo padrão ("DRAFT")
\definecolor{background-color}{RGB}{233 227 206}

% --------------------------------------------------------
% GRÁFICOS E PLOTAGENS
% --------------------------------------------------------
\usepackage{pgfplots}             % Criação de gráficos vetoriais em 2D e 3D
\pgfplotsset{compat=1.18}         % Compatibilidade com versão atual do pacote
\usepackage{currfile} % Permite saber qual é o caminho completo do arquivo atual,

% --------------------------------------------------------
% LINKS E CORES
% --------------------------------------------------------
\usepackage{xcolor}              % Definição de cores (texto, links, etc.)
\usepackage[hidelinks]{hyperref} % Links sem moldura colorida no PDF
\usepackage{url}                 % Quebra automática de URLs
% Dica: se quiser links coloridos, use:
% \usepackage[colorlinks=true, linkcolor=blue, citecolor=blue, urlcolor=blue]{hyperref}

% --------------------------------------------------------
% CAIXAS E DESTAQUES VISUAIS
% --------------------------------------------------------
\usepackage[most]{tcolorbox}      % Criação de caixas coloridas e personalizadas para destaques, avisos, exemplos, etc.

% --------------------------------------------------------
% CÓDIGOS E ALGORITMOS
% --------------------------------------------------------
\usepackage{listings}            % Inclusão de código-fonte com destaque
\usepackage{algorithm}           % Ambiente de algoritmo
\usepackage[noend]{algpseudocode} % Sintaxe tipo pseudocódigo


% --------------------------------------------------------
% COMANDOS
% --------------------------------------------------------

\newcommand{\bbN}{\mathbb{N}}
\newcommand{\bbZ}{\mathbb{Z}}
\newcommand{\bbQ}{\mathbb{Q}}
\newcommand{\bbR}{\mathbb{R}}

\newcommand{\assunto}{FUNÇÕES EXPONENCIAIS}
\newcommand{\atividade}{lista} % prova ou lista
\newcommand{\turma}{ELETROELETRÔNICA 1.\textordmasculine\ ANO}
\newcommand{\numProva}{B}
\newcommand{\professor}{Igor Martins Silva}
\newcommand{\bimestre}{3}
\newcommand{\ano}{2025}
\newcommand{\mesTexto}{setembro}
\newcommand{\mesNum}{09}
\newcommand{\dia}{18}
\newcommand{\dataTexto}{\dia\ de \mesTexto\ de \ano}
\newcommand{\dataNun}{\dia/\mesNum/\ano}

\ifdefstring{\atividade}{prova}{
    \newcommand{\titulo}{Avaliação de Matemática}
}{
    \newcommand{\titulo}{Lista de exercícios de Matemática}
}

\newcommand{\subtitulo}{
    %
    \begin{center}
        CEFET -- Contagem \\[3pt]
        \titulo \ -- \bimestre.\textordmasculine\ Bimestre de \ano \\[3pt]
        \professor
    \end{center}
    %
    \begin{center}
        \vspace{15pt}
        \begin{tabular}{|C{210pt}|C{70pt}|C{210pt}|}
            \hline
            \textbf{ASSUNTO} &
            \textbf{DATA} &
            \textbf{TURMA} \\
            \hline
            \assunto & \dataNun & \turma \\
            \hline
        \end{tabular}
    \end{center}
    %
    \vspace{15pt}
}

\newcommand{\valorquestao}{}

\newenvironment{exercicioBanco}[1][]{%
    \ifdefstring{\atividade}{prova}{%
        \begin{exercicio}[#1]%
    }{%
        \begin{exercicio}%
    }%
}{%
    \end{exercicio}
}

\ifdefstring{\atividade}{lista}{
    \pagecolor{background-color}
}{
    
}



\hypersetup{
    pdfauthor={\professor},
    pdftitle={\titulo -- \assunto},
}

\title{\titulo -- \assunto}
\author{\professor}
\date{\data}

%************* INÍCIO DO DOCUMENTO *************
\begin{document}
    \subtitulo
    
    \vspace{13pt}
    \phantomsection
    \addcontentsline{toc}{section}{Exercício 1}
    %%%%%%%%%%%%%%%%%%%%%%%%%%%%%%%%%
%
%       EXERCÍCIO
%
%%%%%%%%%%%%%%%%%%%%%%%%%%%%%%%%%

\ifdefstring{\atividade}{prova}{%
    \ifdefstring{\modo}{objetivo}{%
        \renewcommand{\valorquestao}{\ValorQObj\ ponto}
    }{%
        \renewcommand{\valorquestao}{\ValorQDisc\ pontos}
    }%
}%

\begin{exercicioBanco}[\valorquestao]
Qual é a soma dos algarismos do número que se obtém ao calcular \(2^{100}\cdot 5^{103}\)?

% Define as alternativas
\newcommand{\alternativas}{%
    \begin{center}
        \begin{tabularx}{\textwidth}{XXXXX}
            (a) \(7\). &
            (b) \(8\). &
            (c) \(9\). &
            (d) \(10\). &
            (e) \(11\).
        \end{tabularx}
    \end{center}
}

% Define a resposta correta
\newcommand{\resposta}{B}

% Lógica condicional para exibição
\ifdefstring{\atividade}{lista}{%
    \alternativas
    \vspace{0.5em}
    
    \noindent\textbf{Resposta:} letra \textbf{\resposta}.
}{%
    \ifdefstring{\modo}{objetivo}{%
        \alternativas
    }{%
        % modo = discursiva → não mostra alternativas
    }
}
\end{exercicioBanco}


    \noindent\rule{\linewidth}{1pt}  % linha horizontal fina
    
    \phantomsection
    \addcontentsline{toc}{section}{Exercício 2}
    %%%%%%%%%%%%%%%%%%%%%%%%%%%%%%%%%
%
%       EXERCÍCIO
%
%%%%%%%%%%%%%%%%%%%%%%%%%%%%%%%%%

\ifdefstring{\atividade}{prova}{%
    \ifdefstring{\modo}{objetivo}{%
        \renewcommand{\valorquestao}{\ValorQObj\ ponto}
    }{%
        \renewcommand{\valorquestao}{\ValorQDisc\ pontos}
    }%
}%

\begin{exercicioBanco}[\valorquestao]
O número de algarismos no produto \(5^{17}\cdot 4^{9}\) é igual a:

% Define as alternativas
\newcommand{\alternativas}{%
    \begin{center}
        \begin{tabularx}{\textwidth}{XXXXX}
            (a) \(17\). &
            (b) \(18\). &
            (c) \(26\). &
            (d) \(34\). &
            (e) \(35\).
        \end{tabularx}
    \end{center}
}

% Define a resposta correta
\newcommand{\resposta}{B}

% Lógica condicional para exibição
\ifdefstring{\atividade}{lista}{%
    \alternativas
    \vspace{0.5em}
    
    \noindent\textbf{Resposta:} letra \textbf{\resposta}.
}{%
    \ifdefstring{\modo}{objetivo}{%
        \alternativas
    }{%
        % modo = discursiva → não mostra alternativas
    }
}
\end{exercicioBanco}


    \noindent\rule{\linewidth}{1pt}  % linha horizontal fina
    
    \phantomsection
    \addcontentsline{toc}{section}{Exercício 3}
    %%%%%%%%%%%%%%%%%%%%%%%%%%%%%%%%%
%
%       EXERCÍCIO
%
%%%%%%%%%%%%%%%%%%%%%%%%%%%%%%%%%

\ifdefstring{\atividade}{prova}{%
    \ifdefstring{\modo}{objetivo}{%
        \renewcommand{\valorquestao}{\ValorQObj\ ponto}
    }{%
        \renewcommand{\valorquestao}{\ValorQDisc\ pontos}
    }%
}%

\begin{exercicioBanco}[\valorquestao]
O valor da expressão \(\sqrt[3]{5^{-2}}\cdot 5^{1,333\cdots}\) é:

% Define as alternativas
\newcommand{\alternativas}{%
    \begin{center}
        \begin{tabularx}{\textwidth}{XXX}
            (a) Um número primo. &
            (b) Um decimal exato. &
            (c) Uma dízima periódica. \\[5pt]
            (d) Um número irracional. &
            (e) Um número não real.
        \end{tabularx}
    \end{center}
}

% Define a resposta correta
\newcommand{\resposta}{D}

% Lógica condicional para exibição
\ifdefstring{\atividade}{lista}{%
    \alternativas
    \vspace{0.5em}
    
    \noindent\textbf{Resposta:} letra \textbf{\resposta}.
}{%
    \ifdefstring{\modo}{objetivo}{%
        \alternativas
    }{%
        % modo = discursiva → não mostra alternativas
    }
}
\end{exercicioBanco}


    \noindent\rule{\linewidth}{1pt}  % linha horizontal fina
    
    \phantomsection
    \addcontentsline{toc}{section}{Exercício 4}
    %%%%%%%%%%%%%%%%%%%%%%%%%%%%%%%%%
%
%       EXERCÍCIO
%
%%%%%%%%%%%%%%%%%%%%%%%%%%%%%%%%%

\ifdefstring{\atividade}{prova}{%
    \ifdefstring{\modo}{objetivo}{%
        \renewcommand{\valorquestao}{\ValorQObj\ ponto}
    }{%
        \renewcommand{\valorquestao}{\ValorQDisc\ pontos}
    }%
}%

\begin{exercicioBanco}[\valorquestao]
O valor da expressão \(\sqrt[3]{5^{-2}}\cdot 5^{1,333\cdots}\) é:

% Define as alternativas
\newcommand{\alternativas}{%
    \begin{center}
        \begin{tabularx}{\textwidth}{XXX}
            (a) Um número primo. &
            (b) Um decimal exato. &
            (c) Uma dízima periódica. \\[5pt]
            (d) Um número irracional. &
            (e) Um número par.
        \end{tabularx}
    \end{center}
}

% Define a resposta correta
\newcommand{\resposta}{D}

% Lógica condicional para exibição
\ifdefstring{\atividade}{lista}{%
    \alternativas
    \vspace{0.5em}
    
    \noindent\textbf{Resposta:} letra \textbf{\resposta}.
}{%
    \ifdefstring{\modo}{objetivo}{%
        \alternativas
    }{%
        % modo = discursiva → não mostra alternativas
    }
}
\end{exercicioBanco}


    \noindent\rule{\linewidth}{1pt}  % linha horizontal fina
    
    \phantomsection
    \addcontentsline{toc}{section}{Exercício 5}
    %%%%%%%%%%%%%%%%%%%%%%%%%%%%%%%%%
%
%       EXERCÍCIO
%
%%%%%%%%%%%%%%%%%%%%%%%%%%%%%%%%%

\ifdefstring{\atividade}{prova}{%
    \ifdefstring{\modo}{objetivo}{%
        \renewcommand{\valorquestao}{\ValorQObj\ ponto}
    }{%
        \renewcommand{\valorquestao}{\ValorQDisc\ pontos}
    }%
}%

\begin{exercicioBanco}[\valorquestao]
Considere \(a = 11^{50}\), \(b = 4^{100}\) e \(c = 2^{150}\). Assinale a alternativa correta. 

% Define as alternativas
\newcommand{\alternativas}{%
    \begin{center}
        \begin{tabularx}{\textwidth}{XXXXX}
            (a) \(c < a < b\). &
            (b) \(c < b < a\). &
            (c) \(a < b < c\). &
            (d) \(a < c < b\). &
            (e) \(b < a < c\).
        \end{tabularx}
    \end{center}
}

% Define a resposta correta
\newcommand{\resposta}{A}

% Lógica condicional para exibição
\ifdefstring{\atividade}{lista}{%
    \alternativas
    \vspace{0.5em}
    
    \noindent\textbf{Resposta:} letra \textbf{\resposta}.
}{%
    \ifdefstring{\modo}{objetivo}{%
        \alternativas
    }{%
        % modo = discursiva → não mostra alternativas
    }
}
\end{exercicioBanco}


    \noindent\rule{\linewidth}{1pt}  % linha horizontal fina
    
    \phantomsection
    \addcontentsline{toc}{section}{Exercício 6}
    %%%%%%%%%%%%%%%%%%%%%%%%%%%%%%%%%
%
%       EXERCÍCIO
%
%%%%%%%%%%%%%%%%%%%%%%%%%%%%%%%%%

\ifdefstring{\atividade}{prova}{%
    \ifdefstring{\modo}{objetivo}{%
        \renewcommand{\valorquestao}{\ValorQObj\ ponto}
    }{%
        \renewcommand{\valorquestao}{\ValorQDisc\ pontos}
    }%
}%

\begin{exercicioBanco}[\valorquestao]
Qual dos números a seguir é o maior?

% Define as alternativas
\newcommand{\alternativas}{%
    \begin{center}
        \begin{tabularx}{\textwidth}{XXXXX}
            (a) \(3^{45}\). &
            (b) \(9^{20}\). &
            (c) \(27^{14}\). &
            (d) \(243^{9}\). &
            (e) \(81^{12}\).
        \end{tabularx}
    \end{center}
}

% Define a resposta correta
\newcommand{\resposta}{E}

% Lógica condicional para exibição
\ifdefstring{\atividade}{lista}{%
    \alternativas
    \vspace{0.5em}
    
    \noindent\textbf{Resposta:} letra \textbf{\resposta}.
}{%
    \ifdefstring{\modo}{objetivo}{%
        \alternativas
    }{%
        % modo = discursiva → não mostra alternativas
    }
}
\end{exercicioBanco}


    \noindent\rule{\linewidth}{1pt}  % linha horizontal fina
    
    \phantomsection
    \addcontentsline{toc}{section}{Exercício 7}
    %%%%%%%%%%%%%%%%%%%%%%%%%%%%%%%%%
%
%       EXERCÍCIO
%
%%%%%%%%%%%%%%%%%%%%%%%%%%%%%%%%%

\ifdefstring{\atividade}{prova}{%
    \ifdefstring{\modo}{objetivo}{%
        \renewcommand{\valorquestao}{\ValorQObj\ ponto}
    }{%
        \renewcommand{\valorquestao}{\ValorQDisc\ pontos}
    }%
}%

\begin{exercicioBanco}[\valorquestao]
Há uma lenda que credita a invenção do xadrez a um brâmane de uma corte indiana, que, atendendo a um pedido do rei, inventou o jogo para demonstrar o valor da inteligência. O rei, encantado com o invento, ofereceu ao brâmane a escolha de uma recompensa. De acordo com essa lenda, o inventor do jogo de xadrez pediu ao rei que a recompensa fosse paga em grãos de arroz da seguinte maneira: 1 grão para a casa 1 do tabuleiro, 2 grãos para a casa 2, 4 para a casa 3, 8 para a casa 4 e assim sucessivamente. Ou seja, a quantidade de grãos para cada casa do tabuleiro correspondia ao dobro da quantidade da casa imediatamente anterior. Escreva uma função \(f\) que expresse a quantidade de grãos de arroz em função do número \(x\) da casa do tabuleiro.

% Define as alternativas
\newcommand{\alternativas}{%
    \begin{center}
        \begin{tabularx}{\textwidth}{XXXXX}
            (a) \(f(x) = x^{64}\). &
            (b) \(f(x) = 2^{x}\). &
            (c) \(f(x) = 64x\). &
            (d) \(f(x) = 2^{x+1}\). &
            (e) \(f(x) = 2^{x-1}\).
        \end{tabularx}
    \end{center}
}

% Define a resposta correta
\newcommand{\resposta}{E}

% Lógica condicional para exibição
\ifdefstring{\atividade}{lista}{%
    \alternativas
    \vspace{0.5em}
    
    \noindent\textbf{Resposta:} letra \textbf{\resposta}.
}{%
    \ifdefstring{\modo}{objetivo}{%
        \alternativas
    }{%
        % modo = discursiva → não mostra alternativas
    }
}
\end{exercicioBanco}

\textbf{Observação}. A quantidade de \(2^{63}\) grãos de arroz equivalem a aproximadamente \(46\) trilhões de pacotes de 5 kg. Pensado esse número como distância, ele corresponde a aproximadamente \(975\) anos-luz, o que daria umas \(230\) bilhões de voltas na Terra.


    \noindent\rule{\linewidth}{1pt}  % linha horizontal fina
    
    \phantomsection
    \addcontentsline{toc}{section}{Exercício 8}
    %%%%%%%%%%%%%%%%%%%%%%%%%%%%%%%%%
%
%       EXERCÍCIO
%
%%%%%%%%%%%%%%%%%%%%%%%%%%%%%%%%%

\ifdefstring{\atividade}{prova}{%
    \ifdefstring{\modo}{objetivo}{%
        \renewcommand{\valorquestao}{\ValorQObj\ ponto}
    }{%
        \renewcommand{\valorquestao}{\ValorQDisc\ pontos}
    }%
}%

\begin{exercicioBanco}[\valorquestao]
O número de bactérias numa cultura, em função do tempo \(t\) (em horas), pode ser expresso por \(N(t) = 256\cdot 2^{0,75t}\). Em quanto tempo, em horas, o número de bactérias será igual a \(2\,048\)?

% Define as alternativas
\newcommand{\alternativas}{%
    \begin{center}
        \begin{tabularx}{\textwidth}{XXXXX}
            (a) \(2\). &
            (b) \(6\). &
            (c) \(8\). &
            (d) \(3\). &
            (e) \(4\).
        \end{tabularx}
    \end{center}
}

% Define a resposta correta
\newcommand{\resposta}{E}

% Lógica condicional para exibição
\ifdefstring{\atividade}{lista}{%
    \alternativas
    \vspace{0.5em}
    
    \noindent\textbf{Resposta:} letra \textbf{\resposta}.
}{%
    \ifdefstring{\modo}{objetivo}{%
        \alternativas
    }{%
        % modo = discursiva → não mostra alternativas
    }
}
\end{exercicioBanco}


    \noindent\rule{\linewidth}{1pt}  % linha horizontal fina
    
    \phantomsection
    \addcontentsline{toc}{section}{Exercício 9}
    %%%%%%%%%%%%%%%%%%%%%%%%%%%%%%%%%
%
%       EXERCÍCIO
%
%%%%%%%%%%%%%%%%%%%%%%%%%%%%%%%%%

\ifdefstring{\atividade}{prova}{%
    \ifdefstring{\modo}{objetivo}{%
        \renewcommand{\valorquestao}{\ValorQObj\ ponto}
    }{%
        \renewcommand{\valorquestao}{\ValorQDisc\ pontos}
    }%
}%

\begin{exercicioBanco}[\valorquestao]
Em uma pesquisa, obteve‑se o gráfico abaixo, que indica o crescimento de uma cultura de bactérias no decorrer de \(6\) meses.

\begin{center}
    \begin{tikzpicture}[scale=0.5]
        % Eixos
        \draw[->] (-1,0) -- (7,0) node[right] {\small \shortstack{t (meses)}};
        \draw[->] (0,0) -- (0,4) node[right] {\small Número de bacterias};
        
        % Pontos de referência
        \draw[dashed] (6,0) -- (6,3) -- (0,3);
        
        % Pontos e marcações
        \filldraw (0,1) circle (2pt);
        \filldraw (6,3) circle (2pt);
        \node[left] at (0,1) {\small \(5\,000\)};
        \node[left] at (0,3) {\small \(15\,000\)};
        \node[below] at (0,0) {\small \(0\)};
        \node[below] at (6,0) {\small \(6\)};

        % Curva da parábola com vértice em (15,13.5)
        \draw[domain=0:6, smooth, samples=100, thick] 
            plot (\x, {(3)^(\x/6)});
    \end{tikzpicture}
\end{center}

Admitindo a lei de formação da função que representa essa situação como \(f(t) = ka^{t}\), determine os valores de \(k\) e de \(a\).

% Define as alternativas
\newcommand{\alternativas}{%
    \begin{center}
        \begin{tabularx}{\textwidth}{XXX}
            (a) \(k = 1\) e \(a = 2\). &
            (b) \(k = 5\,000\) e \(a = \sqrt[6]{3}\). &
            (c) \(k = 15\,000\) e \(a = \sqrt{3}\). \\[5pt]
            (d) \(k = \dfrac{1}{2}\) e \(a = 3\). &
            (e) \(k = \sqrt{2}\) e \(a = \dfrac{1}{2}\).
        \end{tabularx}
    \end{center}
}

% Define a resposta correta
\newcommand{\resposta}{B}

% Lógica condicional para exibição
\ifdefstring{\atividade}{lista}{%
    \alternativas
    \vspace{0.5em}
    
    \noindent\textbf{Resposta:} letra \textbf{\resposta}.
}{%
    \ifdefstring{\modo}{objetivo}{%
        \alternativas
    }{%
        % modo = discursiva → não mostra alternativas
    }
}
\end{exercicioBanco}


    \noindent\rule{\linewidth}{1pt}  % linha horizontal fina
    
    \phantomsection
    \addcontentsline{toc}{section}{Exercício 10}
    %%%%%%%%%%%%%%%%%%%%%%%%%%%%%%%%%
%
%       EXERCÍCIO
%
%%%%%%%%%%%%%%%%%%%%%%%%%%%%%%%%%

\ifdefstring{\atividade}{prova}{%
    \ifdefstring{\modo}{objetivo}{%
        \renewcommand{\valorquestao}{\ValorQObj\ ponto}
    }{%
        \renewcommand{\valorquestao}{\ValorQDisc\ pontos}
    }%
}%

\begin{exercicioBanco}[\valorquestao]
O decaimento radioativo de uma substância se dá de acordo com a fórmula \(r(t) = C\cdot 3^{-6t}\), com \(C\) sendo uma constante diferente de zero e \(r(t)\) a quantidade de radioatividade presente na substância após \(t\) segundos desde o início do decaimento. O valor de \(t\), em segundos, para que a substância fique com a terça parte da radioatividade que tinha inicialmente é igual a:

% Define as alternativas
\newcommand{\alternativas}{%
    \begin{center}
        \begin{tabularx}{\textwidth}{XXXXX}
            (a) \(\dfrac{1}{4}\). &
            (b) \(\dfrac{1}{5}\). &
            (c) \(\dfrac{1}{3}\). &
            (d) \(\dfrac{1}{6}\). &
            (e) \(\dfrac{2}{5}\).
        \end{tabularx}
    \end{center}
}

% Define a resposta correta
\newcommand{\resposta}{D}

% Lógica condicional para exibição
\ifdefstring{\atividade}{lista}{%
    \alternativas
    \vspace{0.5em}
    
    \noindent\textbf{Resposta:} letra \textbf{\resposta}.
}{%
    \ifdefstring{\modo}{objetivo}{%
        \alternativas
    }{%
        % modo = discursiva → não mostra alternativas
    }
}
\end{exercicioBanco}


    \noindent\rule{\linewidth}{1pt}  % linha horizontal fina
    
    \phantomsection
    \addcontentsline{toc}{section}{Exercício 11}
    %%%%%%%%%%%%%%%%%%%%%%%%%%%%%%%%%
%
%       EXERCÍCIO
%
%%%%%%%%%%%%%%%%%%%%%%%%%%%%%%%%%

\ifdefstring{\atividade}{prova}{%
    \ifdefstring{\modo}{objetivo}{%
        \renewcommand{\valorquestao}{\ValorQObj\ ponto}
    }{%
        \renewcommand{\valorquestao}{\ValorQDisc\ pontos}
    }%
}%

\begin{exercicioBanco}[\valorquestao]
Há uma lenda que credita a invenção do xadrez a um brâmane de uma corte indiana, que, atendendo a um pedido do rei, inventou o jogo para demonstrar o valor da inteligência. O rei, encantado com o invento, ofereceu ao brâmane a escolha de uma recompensa. De acordo com essa lenda, o inventor do jogo de xadrez pediu ao rei que a recompensa fosse paga em grãos de arroz da seguinte maneira: 1 grão para a casa 1 do tabuleiro, 2 grãos para a casa 2, 4 para a casa 3, 8 para a casa 4 e assim sucessivamente. Ou seja, a quantidade de grãos para cada casa do tabuleiro correspondia ao dobro da quantidade da casa imediatamente anterior. Escreva uma função \(f\) que expresse a quantidade de grãos de arroz em função do número \(x\) da casa do tabuleiro.

% Define as alternativas
\newcommand{\alternativas}{%
    \begin{center}
        \begin{tabularx}{\textwidth}{XXXXX}
            (a) \(f(x) = x^{64}\). &
            (b) \(f(x) = 2^{x}\). &
            (c) \(f(x) = 64x\). &
            (d) \(f(x) = 2^{x+1}\). &
            (e) \(f(x) = 2^{x-1}\).
        \end{tabularx}
    \end{center}
}

% Define a resposta correta
\newcommand{\resposta}{E}

% Lógica condicional para exibição
\ifdefstring{\atividade}{lista}{%
    \alternativas
    \vspace{0.5em}
    
    \noindent\textbf{Resposta:} letra \textbf{\resposta}.
}{%
    \ifdefstring{\modo}{objetivo}{%
        \alternativas
    }{%
        % modo = discursiva → não mostra alternativas
    }
}
\end{exercicioBanco}

\textbf{Observação}. A quantidade de \(2^{63}\) grãos de arroz equivalem a aproximadamente \(46\) trilhões de pacotes de 5 kg. Pensado esse número como distância, ele corresponde a aproximadamente \(975\) anos-luz, o que daria umas \(230\) bilhões de voltas na Terra.


    \noindent\rule{\linewidth}{1pt}  % linha horizontal fina
    
    \phantomsection
    \addcontentsline{toc}{section}{Exercício 12}
    %%%%%%%%%%%%%%%%%%%%%%%%%%%%%%%%%
%
%       EXERCÍCIO
%
%%%%%%%%%%%%%%%%%%%%%%%%%%%%%%%%%

\ifdefstring{\atividade}{prova}{%
    \ifdefstring{\modo}{objetivo}{%
        \renewcommand{\valorquestao}{\ValorQObj\ ponto}
    }{%
        \renewcommand{\valorquestao}{\ValorQDisc\ pontos}
    }%
}%

\begin{exercicioBanco}[\valorquestao]
O número de bactérias numa cultura, em função do tempo \(t\) (em horas), pode ser expresso por \(N(t) = 256\cdot 2^{0,75t}\). Em quanto tempo, em horas, o número de bactérias será igual a \(2\,048\)?

% Define as alternativas
\newcommand{\alternativas}{%
    \begin{center}
        \begin{tabularx}{\textwidth}{XXXXX}
            (a) \(2\). &
            (b) \(6\). &
            (c) \(8\). &
            (d) \(3\). &
            (e) \(4\).
        \end{tabularx}
    \end{center}
}

% Define a resposta correta
\newcommand{\resposta}{E}

% Lógica condicional para exibição
\ifdefstring{\atividade}{lista}{%
    \alternativas
    \vspace{0.5em}
    
    \noindent\textbf{Resposta:} letra \textbf{\resposta}.
}{%
    \ifdefstring{\modo}{objetivo}{%
        \alternativas
    }{%
        % modo = discursiva → não mostra alternativas
    }
}
\end{exercicioBanco}


    \noindent\rule{\linewidth}{1pt}  % linha horizontal fina
    
    \phantomsection
    \addcontentsline{toc}{section}{Exercício 13}
    %%%%%%%%%%%%%%%%%%%%%%%%%%%%%%%%%
%
%       EXERCÍCIO
%
%%%%%%%%%%%%%%%%%%%%%%%%%%%%%%%%%

\ifdefstring{\atividade}{prova}{%
    \ifdefstring{\modo}{objetivo}{%
        \renewcommand{\valorquestao}{\ValorQObj\ ponto}
    }{%
        \renewcommand{\valorquestao}{\ValorQDisc\ pontos}
    }%
}%

\begin{exercicioBanco}[\valorquestao]
Em uma pesquisa, obteve‑se o gráfico abaixo, que indica o crescimento de uma cultura de bactérias no decorrer de \(6\) meses.

\begin{center}
    \begin{tikzpicture}[scale=0.5]
        % Eixos
        \draw[->] (-1,0) -- (7,0) node[right] {\small \shortstack{t (meses)}};
        \draw[->] (0,0) -- (0,4) node[right] {\small Número de bacterias};
        
        % Pontos de referência
        \draw[dashed] (6,0) -- (6,3) -- (0,3);
        
        % Pontos e marcações
        \filldraw (0,1) circle (2pt);
        \filldraw (6,3) circle (2pt);
        \node[left] at (0,1) {\small \(5\,000\)};
        \node[left] at (0,3) {\small \(15\,000\)};
        \node[below] at (0,0) {\small \(0\)};
        \node[below] at (6,0) {\small \(6\)};

        % Curva da parábola com vértice em (15,13.5)
        \draw[domain=0:6, smooth, samples=100, thick] 
            plot (\x, {(3)^(\x/6)});
    \end{tikzpicture}
\end{center}

Admitindo a lei de formação da função que representa essa situação como \(f(t) = ka^{t}\), determine os valores de \(k\) e de \(a\).

% Define as alternativas
\newcommand{\alternativas}{%
    \begin{center}
        \begin{tabularx}{\textwidth}{XXX}
            (a) \(k = 1\) e \(a = 2\). &
            (b) \(k = 5\,000\) e \(a = \sqrt[6]{3}\). &
            (c) \(k = 15\,000\) e \(a = \sqrt{3}\). \\[5pt]
            (d) \(k = \dfrac{1}{2}\) e \(a = 3\). &
            (e) \(k = \sqrt{2}\) e \(a = \dfrac{1}{2}\).
        \end{tabularx}
    \end{center}
}

% Define a resposta correta
\newcommand{\resposta}{B}

% Lógica condicional para exibição
\ifdefstring{\atividade}{lista}{%
    \alternativas
    \vspace{0.5em}
    
    \noindent\textbf{Resposta:} letra \textbf{\resposta}.
}{%
    \ifdefstring{\modo}{objetivo}{%
        \alternativas
    }{%
        % modo = discursiva → não mostra alternativas
    }
}
\end{exercicioBanco}


    \noindent\rule{\linewidth}{1pt}  % linha horizontal fina
    
    \phantomsection
    \addcontentsline{toc}{section}{Exercício 14}
    %%%%%%%%%%%%%%%%%%%%%%%%%%%%%%%%%
%
%       EXERCÍCIO
%
%%%%%%%%%%%%%%%%%%%%%%%%%%%%%%%%%

\ifdefstring{\atividade}{prova}{%
    \ifdefstring{\modo}{objetivo}{%
        \renewcommand{\valorquestao}{\ValorQObj\ ponto}
    }{%
        \renewcommand{\valorquestao}{\ValorQDisc\ pontos}
    }%
}%

\begin{exercicioBanco}[\valorquestao]
O decaimento radioativo de uma substância se dá de acordo com a fórmula \(r(t) = C\cdot 3^{-6t}\), com \(C\) sendo uma constante diferente de zero e \(r(t)\) a quantidade de radioatividade presente na substância após \(t\) segundos desde o início do decaimento. O valor de \(t\), em segundos, para que a substância fique com a terça parte da radioatividade que tinha inicialmente é igual a:

% Define as alternativas
\newcommand{\alternativas}{%
    \begin{center}
        \begin{tabularx}{\textwidth}{XXXXX}
            (a) \(\dfrac{1}{4}\). &
            (b) \(\dfrac{1}{5}\). &
            (c) \(\dfrac{1}{3}\). &
            (d) \(\dfrac{1}{6}\). &
            (e) \(\dfrac{2}{5}\).
        \end{tabularx}
    \end{center}
}

% Define a resposta correta
\newcommand{\resposta}{D}

% Lógica condicional para exibição
\ifdefstring{\atividade}{lista}{%
    \alternativas
    \vspace{0.5em}
    
    \noindent\textbf{Resposta:} letra \textbf{\resposta}.
}{%
    \ifdefstring{\modo}{objetivo}{%
        \alternativas
    }{%
        % modo = discursiva → não mostra alternativas
    }
}
\end{exercicioBanco}


    \noindent\rule{\linewidth}{1pt}  % linha horizontal fina
    
    \phantomsection
    \addcontentsline{toc}{section}{Exercício 15}
    %%%%%%%%%%%%%%%%%%%%%%%%%%%%%%%%%
%
%       EXERCÍCIO
%
%%%%%%%%%%%%%%%%%%%%%%%%%%%%%%%%%

\ifdefstring{\atividade}{prova}{%
    \ifdefstring{\modo}{objetivo}{%
        \renewcommand{\valorquestao}{\ValorQObj\ ponto}
    }{%
        \renewcommand{\valorquestao}{\ValorQDisc\ pontos}
    }%
}%

\begin{exercicioBanco}[\valorquestao]
Considere a função \(f(x) = e^{x}\), onde \(e\) é o número de Euler. Seja \(g(x)\) a reta tangente ao gráfico de \(f\) passando pelo ponto \((0,1)\). Qual é o zero de \(g(x)\)?

% Define as alternativas
\newcommand{\alternativas}{%
    \begin{center}
        \begin{tabularx}{\textwidth}{XXXXX}
            (a) \(-2\). &
            (b) \(-1\). &
            (c) \(0\). &
            (d) \(1\). &
            (e) \(2\).
        \end{tabularx}
    \end{center}
}

% Define a resposta correta
\newcommand{\resposta}{E}

% Lógica condicional para exibição
\ifdefstring{\atividade}{lista}{%
    \alternativas
    \vspace{0.5em}
    
    \noindent\textbf{Resposta:} letra \textbf{\resposta}.
}{%
    \ifdefstring{\modo}{objetivo}{%
        \alternativas
    }{%
        % modo = discursiva → não mostra alternativas
    }
}
\end{exercicioBanco}


    \noindent\rule{\linewidth}{1pt}  % linha horizontal fina
    
    \phantomsection
    \addcontentsline{toc}{section}{Exercício 16}
    %%%%%%%%%%%%%%%%%%%%%%%%%%%%%%%%%
%
%       EXERCÍCIO
%
%%%%%%%%%%%%%%%%%%%%%%%%%%%%%%%%%

\ifdefstring{\atividade}{prova}{%
    \ifdefstring{\modo}{objetivo}{%
        \renewcommand{\valorquestao}{\ValorQObj\ ponto}
    }{%
        \renewcommand{\valorquestao}{\ValorQDisc\ pontos}
    }%
}%

\begin{exercicioBanco}[\valorquestao]
O conjunto solução da inequação 
%
\[
    \left(\dfrac{1}{7^{x}}\right)^{x^{3}-4} - 7(7^{x^{2}+1})^{2x-1} \ge 0
\]
%
é:

% Define as alternativas
\newcommand{\alternativas}{%
    \begin{center}
        \begin{tabularx}{\textwidth}{XXX}
            (a) \([-2,-1]\). &
            (b) \([0,1]\). &
            (c) \(]-\infty,-2] \cup [-1,0] \cup [1,\infty[\). \\[5pt]
            (d) \([0,\infty[\). &
            (e) \([-2,-1] \cup [0,1]\).
        \end{tabularx}
    \end{center}
}

% Define a resposta correta
\newcommand{\resposta}{E}

% Lógica condicional para exibição
\ifdefstring{\atividade}{lista}{%
    \alternativas
    \vspace{0.5em}
    
    \noindent\textbf{Resposta:} letra \textbf{\resposta}.
}{%
    \ifdefstring{\modo}{objetivo}{%
        \alternativas
    }{%
        % modo = discursiva → não mostra alternativas
    }
}
\end{exercicioBanco}


    \noindent\rule{\linewidth}{1pt}  % linha horizontal fina
    
    \phantomsection
    \addcontentsline{toc}{section}{Exercício 17}
    %%%%%%%%%%%%%%%%%%%%%%%%%%%%%%%%%
%
%       EXERCÍCIO
%
%%%%%%%%%%%%%%%%%%%%%%%%%%%%%%%%%

\ifdefstring{\atividade}{prova}{%
    \ifdefstring{\modo}{objetivo}{%
        \renewcommand{\valorquestao}{\ValorQObj\ ponto}
    }{%
        \renewcommand{\valorquestao}{\ValorQDisc\ pontos}
    }%
}%

\begin{exercicioBanco}[\valorquestao]
Considere as funções \(f(x) = 3^{x}\) e \(g(x) = x^{3}\), definidas para todo número real \(x\). O número de soluções da equação \(f\big(g(x)\big) = g\big(f(x)\big)\) é igua a:

% Define as alternativas
\newcommand{\alternativas}{%
    \begin{center}
        \begin{tabularx}{\textwidth}{XXXXX}
            (a) \(0\). &
            (b) \(1\). &
            (c) \(2\). &
            (d) \(3\). &
            (e) \(4\).
        \end{tabularx}
    \end{center}
}

% Define a resposta correta
\newcommand{\resposta}{D}

% Lógica condicional para exibição
\ifdefstring{\atividade}{lista}{%
    \alternativas
    \vspace{0.5em}
    
    \noindent\textbf{Resposta:} letra \textbf{\resposta}.
}{%
    \ifdefstring{\modo}{objetivo}{%
        \alternativas
    }{%
        % modo = discursiva → não mostra alternativas
    }
}
\end{exercicioBanco}


    \noindent\rule{\linewidth}{1pt}  % linha horizontal fina
    
\end{document}
%*************** FINAL DO DOCUMENTO ***************
