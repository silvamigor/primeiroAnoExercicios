\documentclass[a4paper,12pt]{article}

% --------------------------------------------------------
% CODIFICAÇÃO E TIPOGRAFIA
% --------------------------------------------------------
\usepackage[utf8]{inputenc}       % Permite acentuação direta no código-fonte
\usepackage[T1]{fontenc}          % Usa codificação T1 (melhor para português)
\usepackage[brazil]{babel}        % Tradução e hifenização para português
\usepackage{microtype}            % Melhora o espaçamento entre letras/palavras
\usepackage{parskip}              % Espaço entre parágrafos em vez de recuo

% --------------------------------------------------------
% PACOTES MATEMÁTICOS
% --------------------------------------------------------
\usepackage{amsmath}              % Ambientes matemáticos como align, equation
%\usepackage{amssymb}              % Símbolos matemáticos adicionais
\usepackage{amsthm}               % Ambientes de teoremas
\usepackage{mathtools}            % Extensões do amsmath (ex: \coloneqq)
\usepackage{mathrsfs}             % Fonte caligráfica com \mathscr
\usepackage{bbm}                  % Indicador \mathbbm{1}
\usepackage{dsfont}               % Alternativa para conjuntos: \mathds
\usepackage{commath}              % Notações como \abs{}, \norm{}, \eval{}
\usepackage{pgfmath}             % Permite realizar cálculos matemáticos (somas, produtos, números aleatórios, etc.)
\usepackage{siunitx}              % sistema de unidades
\sisetup{group-separator = {\,}} % espaço fino

% --------------------------------------------------------
% FONTE MODERNA (TEXTO E MATEMÁTICA)
% --------------------------------------------------------
%\usepackage{newtxtext}            % Fonte do texto (estilo Times)
%\usepackage{newtxmath}            % Fonte matemática compatível com pacotes AMS
\usepackage[bitstream-charter]{mathdesign} % Fonte elegante e matemática harmoniosa

% --------------------------------------------------------
% AMBIENTES TEÓRICOS PERSONALIZADOS
% --------------------------------------------------------
\theoremstyle{definition}
\newtheorem{definicao}{Definição}       % Definição numerada por seção
\newtheorem{exemplo}{Exemplo}           % Exemplo numerado por seção

%\theoremstyle{plain}
\newtheorem{teorema}{Teorema}           % Teorema numerado por seção
\newtheorem{lema}[teorema]{Lema}                 % Lema com numeração conjunta
\newtheorem{proposicao}[teorema]{Proposição}     % Proposição idem
\newtheorem{corolario}[teorema]{Corolário}       % Corolário idem
\newtheorem{exercicio}[teorema]{Exercício}

%\theoremstyle{remark}
\newtheorem{observacao}[teorema]{Observação}     % Observação com mesmo contador

% --------------------------------------------------------
% TABELAS E FIGURAS
% --------------------------------------------------------
\usepackage{graphicx}             % Inclusão de imagens
\usepackage{float}                % Controle de posicionamento (ex: [H])
\usepackage{caption}              % Personalização de legendas
\usepackage{subcaption}          % Subfiguras com \begin{subfigure}
\usepackage{booktabs}            % Tabelas com qualidade profissional
\usepackage{array}               % Mais opções em colunas de tabelas
\usepackage{multirow}            % Células que ocupam várias linhas
\usepackage{multicol}            % Disposição em colunas múltiplas
\usepackage{tabularx}            % Tabelas com colunas de largura ajustável
\usepackage{longtable}           % Tabelas que quebram página
\usepackage{arydshln}            % Linhas tracejadas horizontais e verticais
\usepackage{makecell}            % Células com múltiplas linhas (e quebra de linha)
\usepackage{diagbox}             % Cabeçalho diagonal em tabelas
\usepackage{pdflscape}           % Gira páginas (modo paisagem)
\newcolumntype{L}[1]{>{\raggedright\let\newline\\\arraybackslash\hspace{0pt}}m{#1}}
\newcolumntype{C}[1]{>{\centering\let\newline\\\arraybackslash\hspace{0pt}}m{#1}}
\newcolumntype{R}[1]{>{\raggedleft\let\newline\\\arraybackslash\hspace{0pt}}m{#1}}

% --------------------------------------------------------
% FORMATAÇÃO E ESTRUTURA
% --------------------------------------------------------
\usepackage{geometry}            % Configuração de margens
\geometry{a4paper, left=1cm, right=1cm, bottom=1.5cm, top=1.5cm, headsep=1cm, footskip=1cm}
\usepackage{titlesec}            % Controle sobre títulos de seções
\usepackage{fancyhdr}            % Cabeçalhos e rodapés personalizados
\pagestyle{fancy}
\fancyhf{}                % Limpa cabeçalho e rodapé
\fancyfoot[R]{\thepage}   % Numeração à direita no rodapé
\renewcommand{\headrulewidth}{0pt} % Remove linha no topo
\usepackage{enumitem}            % Listas com controle de espaçamento e símbolo
\usepackage{tikz}                % Desenho de gráficos vetoriais (diagramas, etc.)
\usetikzlibrary{matrix,arrows,patterns,snakes,decorations.pathreplacing,3d,arrows.meta,calc}
\usepackage{etoolbox}             % Permite lógica condicional e manipulação de comandos
\usepackage{background}           % Permite adicionar conteúdo fixo no fundo da página (ex: linhas, marcas d'água)
\SetBgContents{} % Remove conteúdo padrão ("DRAFT")
\definecolor{background-color}{RGB}{233 227 206}

% --------------------------------------------------------
% GRÁFICOS E PLOTAGENS
% --------------------------------------------------------
\usepackage{pgfplots}             % Criação de gráficos vetoriais em 2D e 3D
\pgfplotsset{compat=1.18}         % Compatibilidade com versão atual do pacote
\usepackage{currfile} % Permite saber qual é o caminho completo do arquivo atual,

% --------------------------------------------------------
% LINKS E CORES
% --------------------------------------------------------
\usepackage{xcolor}              % Definição de cores (texto, links, etc.)
\usepackage[hidelinks]{hyperref} % Links sem moldura colorida no PDF
\usepackage{url}                 % Quebra automática de URLs
% Dica: se quiser links coloridos, use:
% \usepackage[colorlinks=true, linkcolor=blue, citecolor=blue, urlcolor=blue]{hyperref}

% --------------------------------------------------------
% CAIXAS E DESTAQUES VISUAIS
% --------------------------------------------------------
\usepackage[most]{tcolorbox}      % Criação de caixas coloridas e personalizadas para destaques, avisos, exemplos, etc.

% --------------------------------------------------------
% CÓDIGOS E ALGORITMOS
% --------------------------------------------------------
\usepackage{listings}            % Inclusão de código-fonte com destaque
\usepackage{algorithm}           % Ambiente de algoritmo
\usepackage[noend]{algpseudocode} % Sintaxe tipo pseudocódigo


% --------------------------------------------------------
% COMANDOS
% --------------------------------------------------------

\newcommand{\bbN}{\mathbb{N}}
\newcommand{\bbZ}{\mathbb{Z}}
\newcommand{\bbQ}{\mathbb{Q}}
\newcommand{\bbR}{\mathbb{R}}

\newcommand{\assunto}{FUNÇÕES EXPONENCIAIS}
\newcommand{\atividade}{lista} % prova ou lista
\newcommand{\turma}{CONTROLE AMBIENTAL 1.\textordmasculine\ ANO}
\newcommand{\numProva}{A}
\newcommand{\professor}{Igor Martins Silva}
\newcommand{\bimestre}{1}
\newcommand{\ano}{2026}
\newcommand{\mesTexto}{fevereiro}
\newcommand{\mesNum}{02}
\newcommand{\dia}{19}
\newcommand{\dataTexto}{\dia\ de \mesTexto\ de \ano}
\newcommand{\dataNun}{\dia/\mesNum/\ano}

\newcommand{\titulo}{%
  \ifdefstring{\atividade}{prova}{Avaliação de Matemática}{%
  \ifdefstring{\atividade}{lista}{Lista de exercícios de Matemática}{%
  \ifdefstring{\atividade}{as}{Avaliação Somativa}{%
  \ifdefstring{\atividade}{ad}{Avaliação Diagnóstica}{%
  Atividade de Matemática}}}}%
}

\newcommand{\emtipo}{%
  \ifdefstring{\tipo}{individual}{\tipo}{em \tipo}%
}

\newcommand{\subtitulobase}[3]{%
    % #1 = título da 1ª coluna
    % #2 = conteúdo da 1ª coluna
    % #3 = mostra número da prova? (vazio ou algo)
    
    \begin{center}
        CEFET-MG -- Campus Contagem \\[3pt]
        \titulo\ -- \bimestre.\textordmasculine\ Bimestre de \ano \\[3pt]
        \professor
    \end{center}
    
    \begin{center}
        \vspace{15pt}
        \begin{tabular}{|C{210pt}|C{70pt}|C{210pt}|}
            \hline
            \textbf{#1} &
            \textbf{DATA} &
            \textbf{TURMA} \\
            \hline
            #2 & \dataNun & \turma \\
            \hline
        \end{tabular}
    \end{center}
    
    #3
    
    \vspace{15pt}
}

\newcommand{\subtitulolis}{%
  \subtitulobase{ASSUNTO}{\assunto}{}%
}

\newcommand{\subtitulopr}{%
  \subtitulobase{ASSUNTO}{\assunto}{\boxed{\numProva}}%
}

\newcommand{\subtituloas}{%
  \subtitulobase{DISCIPLINA}{MATEMÁTICA}{}%
}

\newcommand{\subtituload}{%
  \subtitulobase{DISCIPLINA}{MATEMÁTICA}{}%
}

\newcommand{\valorquestao}{}

\newenvironment{exercicioBanco}[1][]{%
    \ifdefstring{\atividade}{prova}{%
        \begin{exercicio}[#1]%
    }{%
        \begin{exercicio}%
    }%
}{%
    \end{exercicio}
}

\ifdefstring{\atividade}{lista}{
    \pagecolor{background-color}
}{
    
}



\hypersetup{
    pdfauthor={\professor},
    pdftitle={\titulo -- \assunto},
}

\title{\titulo -- \assunto}
\author{\professor}
\date{\data}

%************* INÍCIO DO DOCUMENTO *************
\begin{document}
    \subtitulo
    
    \vspace{13pt}
    \phantomsection
    \addcontentsline{toc}{section}{Exercício 1}
    %%%%%%%%%%%%%%%%%%%%%%%%%%%%%%%%%
%
%       EXERCÍCIO
%
%%%%%%%%%%%%%%%%%%%%%%%%%%%%%%%%%

\ifdefstring{\atividade}{prova}{%
    \ifdefstring{\modo}{objetivo}{%
        \renewcommand{\valorquestao}{\ValorQObj\ ponto}
    }{%
        \renewcommand{\valorquestao}{\ValorQDisc\ pontos}
    }%
}%

\begin{exercicioBanco}[\valorquestao]
Qual é a soma dos algarismos do número que se obtém ao calcular \(2^{100}\cdot 5^{103}\)?

% Define as alternativas
\newcommand{\alternativas}{%
    \begin{center}
        \begin{tabularx}{\textwidth}{XXXXX}
            (a) \(7\). &
            (b) \(8\). &
            (c) \(9\). &
            (d) \(10\). &
            (e) \(11\).
        \end{tabularx}
    \end{center}
}

% Define a resposta correta
\newcommand{\resposta}{B}

% Lógica condicional para exibição
\ifdefstring{\atividade}{lista}{%
    \alternativas
    \vspace{0.5em}
    
    \noindent\textbf{Resposta:} letra \textbf{\resposta}.
}{%
    \ifdefstring{\modo}{objetivo}{%
        \alternativas
    }{%
        % modo = discursiva → não mostra alternativas
    }
}
\end{exercicioBanco}


    \noindent\rule{\linewidth}{1pt}  % linha horizontal fina
    
    \phantomsection
    \addcontentsline{toc}{section}{Exercício 2}
    %%%%%%%%%%%%%%%%%%%%%%%%%%%%%%%%%
%
%       EXERCÍCIO
%
%%%%%%%%%%%%%%%%%%%%%%%%%%%%%%%%%

\ifdefstring{\atividade}{prova}{%
    \ifdefstring{\modo}{objetivo}{%
        \renewcommand{\valorquestao}{\ValorQObj\ ponto}
    }{%
        \renewcommand{\valorquestao}{\ValorQDisc\ pontos}
    }%
}%

\begin{exercicioBanco}[\valorquestao]
O número de algarismos no produto \(5^{17}\cdot 4^{9}\) é igual a:

% Define as alternativas
\newcommand{\alternativas}{%
    \begin{center}
        \begin{tabularx}{\textwidth}{XXXXX}
            (a) \(17\). &
            (b) \(18\). &
            (c) \(26\). &
            (d) \(34\). &
            (e) \(35\).
        \end{tabularx}
    \end{center}
}

% Define a resposta correta
\newcommand{\resposta}{B}

% Lógica condicional para exibição
\ifdefstring{\atividade}{lista}{%
    \alternativas
    \vspace{0.5em}
    
    \noindent\textbf{Resposta:} letra \textbf{\resposta}.
}{%
    \ifdefstring{\modo}{objetivo}{%
        \alternativas
    }{%
        % modo = discursiva → não mostra alternativas
    }
}
\end{exercicioBanco}


    \noindent\rule{\linewidth}{1pt}  % linha horizontal fina
    
    \phantomsection
    \addcontentsline{toc}{section}{Exercício 3}
    %%%%%%%%%%%%%%%%%%%%%%%%%%%%%%%%%
%
%       EXERCÍCIO
%
%%%%%%%%%%%%%%%%%%%%%%%%%%%%%%%%%

\ifdefstring{\atividade}{prova}{%
    \ifdefstring{\modo}{objetivo}{%
        \renewcommand{\valorquestao}{\ValorQObj\ ponto}
    }{%
        \renewcommand{\valorquestao}{\ValorQDisc\ pontos}
    }%
}%

\begin{exercicioBanco}[\valorquestao]
Determine o valor numérico da expressão \((\sqrt[6]{4})^{-3} - \left(\dfrac{5}{\sqrt{5}}\right)^{2}\).

% Define as alternativas
\newcommand{\alternativas}{%
    \begin{center}
        \begin{tabularx}{\textwidth}{XXXXX}
            (a) \(0\). &
            (b) \(1\). &
            (c) \(-1\). &
            (d) \(-\dfrac{9}{2}\). &
            (e) \(\dfrac{1}{2}\).
        \end{tabularx}
    \end{center}
}

% Define a resposta correta
\newcommand{\resposta}{D}

% Lógica condicional para exibição
\ifdefstring{\atividade}{lista}{%
    \alternativas
    \vspace{0.5em}
    
    \noindent\textbf{Resposta:} letra \textbf{\resposta}.
}{%
    \ifdefstring{\modo}{objetivo}{%
        \alternativas
    }{%
        % modo = discursiva → não mostra alternativas
    }
}
\end{exercicioBanco}


    \noindent\rule{\linewidth}{1pt}  % linha horizontal fina
    
    \phantomsection
    \addcontentsline{toc}{section}{Exercício 4}
    %%%%%%%%%%%%%%%%%%%%%%%%%%%%%%%%%
%
%       EXERCÍCIO
%
%%%%%%%%%%%%%%%%%%%%%%%%%%%%%%%%%

\ifdefstring{\atividade}{prova}{%
    \ifdefstring{\modo}{objetivo}{%
        \renewcommand{\valorquestao}{\ValorQObj\ ponto}
    }{%
        \renewcommand{\valorquestao}{\ValorQDisc\ pontos}
    }%
}%

\begin{exercicioBanco}[\valorquestao]
Considere \(a = 11^{50}\), \(b = 4^{100}\) e \(c = 2^{150}\). Assinale a alternativa correta. 

% Define as alternativas
\newcommand{\alternativas}{%
    \begin{center}
        \begin{tabularx}{\textwidth}{XXXXX}
            (a) \(c < a < b\). &
            (b) \(c < b < a\). &
            (c) \(a < b < c\). &
            (d) \(a < c < b\). &
            (e) \(b < a < c\).
        \end{tabularx}
    \end{center}
}

% Define a resposta correta
\newcommand{\resposta}{A}

% Lógica condicional para exibição
\ifdefstring{\atividade}{lista}{%
    \alternativas
    \vspace{0.5em}
    
    \noindent\textbf{Resposta:} letra \textbf{\resposta}.
}{%
    \ifdefstring{\modo}{objetivo}{%
        \alternativas
    }{%
        % modo = discursiva → não mostra alternativas
    }
}
\end{exercicioBanco}


    \noindent\rule{\linewidth}{1pt}  % linha horizontal fina
    
    \phantomsection
    \addcontentsline{toc}{section}{Exercício 5}
    %%%%%%%%%%%%%%%%%%%%%%%%%%%%%%%%%
%
%       EXERCÍCIO
%
%%%%%%%%%%%%%%%%%%%%%%%%%%%%%%%%%

\ifdefstring{\atividade}{prova}{%
    \ifdefstring{\modo}{objetivo}{%
        \renewcommand{\valorquestao}{\ValorQObj\ ponto}
    }{%
        \renewcommand{\valorquestao}{\ValorQDisc\ pontos}
    }%
}%

\begin{exercicioBanco}[\valorquestao]
Determine o valor da expressão \(\sqrt[\raisebox{0.7ex}{$\scriptscriptstyle 3$}]{\dfrac{14}{125}+\sqrt{\dfrac{3}{5}-\dfrac{11}{25}}}\).

% Define as alternativas
\newcommand{\alternativas}{%
    \begin{center}
        \begin{tabularx}{\textwidth}{XXXXX}
            (a) \(\dfrac{\sqrt[3]{14}+2}{5}\). &
            (b) \(\dfrac{\sqrt[3]{114}}{5}\). &
            (c) \(\dfrac{6}{5}\). &
            (d) \(\dfrac{4}{5}\). &
            (e) \(\dfrac{3}{5}\).
        \end{tabularx}
    \end{center}
}

% Define a resposta correta
\newcommand{\resposta}{D}

% Lógica condicional para exibição
\ifdefstring{\atividade}{lista}{%
    \alternativas
    \vspace{0.5em}
    
    \noindent\textbf{Resposta:} letra \textbf{\resposta}.
}{%
    \ifdefstring{\modo}{objetivo}{%
        \alternativas
    }{%
        % modo = discursiva → não mostra alternativas
    }
}
\end{exercicioBanco}


    \noindent\rule{\linewidth}{1pt}  % linha horizontal fina
    
    \phantomsection
    \addcontentsline{toc}{section}{Exercício 6}
    %%%%%%%%%%%%%%%%%%%%%%%%%%%%%%%%%
%
%       EXERCÍCIO
%
%%%%%%%%%%%%%%%%%%%%%%%%%%%%%%%%%

\ifdefstring{\atividade}{prova}{%
    \ifdefstring{\modo}{objetivo}{%
        \renewcommand{\valorquestao}{\ValorQObj\ ponto}
    }{%
        \renewcommand{\valorquestao}{\ValorQDisc\ pontos}
    }%
}%

\begin{exercicioBanco}[\valorquestao]
Qual dos números a seguir é o maior?

% Define as alternativas
\newcommand{\alternativas}{%
    \begin{center}
        \begin{tabularx}{\textwidth}{XXXXX}
            (a) \(3^{45}\). &
            (b) \(9^{20}\). &
            (c) \(27^{14}\). &
            (d) \(243^{9}\). &
            (e) \(81^{12}\).
        \end{tabularx}
    \end{center}
}

% Define a resposta correta
\newcommand{\resposta}{E}

% Lógica condicional para exibição
\ifdefstring{\atividade}{lista}{%
    \alternativas
    \vspace{0.5em}
    
    \noindent\textbf{Resposta:} letra \textbf{\resposta}.
}{%
    \ifdefstring{\modo}{objetivo}{%
        \alternativas
    }{%
        % modo = discursiva → não mostra alternativas
    }
}
\end{exercicioBanco}


    \noindent\rule{\linewidth}{1pt}  % linha horizontal fina
    
    \phantomsection
    \addcontentsline{toc}{section}{Exercício 7}
    \input{../FE_Exercicio7}
    \noindent\rule{\linewidth}{1pt}  % linha horizontal fina
    
    \phantomsection
    \addcontentsline{toc}{section}{Exercício 8}
    %%%%%%%%%%%%%%%%%%%%%%%%%%%%%%%%%
%
%       EXERCÍCIO
%
%%%%%%%%%%%%%%%%%%%%%%%%%%%%%%%%%

\ifdefstring{\atividade}{prova}{%
    \ifdefstring{\modo}{objetivo}{%
        \renewcommand{\valorquestao}{\ValorQObj\ ponto}
    }{%
        \renewcommand{\valorquestao}{\ValorQDisc\ pontos}
    }%
}%

\begin{exercicioBanco}[\valorquestao]
Determine o valor da expressão
%
\[
    \dfrac{0,5^{2}\cdot 2^{0,333\cdots} \cdot \sqrt[3]{16}}{0,125^{-3}}.
\]
%

% Define as alternativas
\newcommand{\alternativas}{%
    \begin{center}
        \begin{tabularx}{\textwidth}{XXXXX}
            (a) \(2^{-\frac{14}{3}}\). &
            (b) \(2^{-\frac{16}{3}}\). &
            (c) \(2^{-6}\). &
            (d) \(2^{-\frac{22}{3}}\). &
            (e) \(2^{-8}\).
        \end{tabularx}
    \end{center}
}

% Define a resposta correta
\newcommand{\resposta}{B}

% Lógica condicional para exibição
\ifdefstring{\atividade}{lista}{%
    \alternativas
    \vspace{0.5em}
    
    \noindent\textbf{Resposta:} letra \textbf{\resposta}.
}{%
    \ifdefstring{\modo}{objetivo}{%
        \alternativas
    }{%
        % modo = discursiva → não mostra alternativas
    }
}
\end{exercicioBanco}


    \noindent\rule{\linewidth}{1pt}  % linha horizontal fina
    
    \phantomsection
    \addcontentsline{toc}{section}{Exercício 9}
    %%%%%%%%%%%%%%%%%%%%%%%%%%%%%%%%%
%
%       EXERCÍCIO
%
%%%%%%%%%%%%%%%%%%%%%%%%%%%%%%%%%

\ifdefstring{\atividade}{prova}{%
    \ifdefstring{\modo}{objetivo}{%
        \renewcommand{\valorquestao}{\ValorQObj\ ponto}
    }{%
        \renewcommand{\valorquestao}{\ValorQDisc\ pontos}
    }%
}%

\begin{exercicioBanco}[\valorquestao]
Em uma pesquisa, obteve‑se o gráfico abaixo, que indica o crescimento de uma cultura de bactérias no decorrer de \(6\) meses.

\begin{center}
    \begin{tikzpicture}[scale=0.5]
        % Eixos
        \draw[->] (-1,0) -- (7,0) node[right] {\small \shortstack{t (meses)}};
        \draw[->] (0,0) -- (0,4) node[right] {\small Número de bacterias};
        
        % Pontos de referência
        \draw[dashed] (6,0) -- (6,3) -- (0,3);
        
        % Pontos e marcações
        \filldraw (0,1) circle (2pt);
        \filldraw (6,3) circle (2pt);
        \node[left] at (0,1) {\small \(5\,000\)};
        \node[left] at (0,3) {\small \(15\,000\)};
        \node[below] at (0,0) {\small \(0\)};
        \node[below] at (6,0) {\small \(6\)};

        % Curva da parábola com vértice em (15,13.5)
        \draw[domain=0:6, smooth, samples=100, thick] 
            plot (\x, {(3)^(\x/6)});
    \end{tikzpicture}
\end{center}

Admitindo a lei de formação da função que representa essa situação como \(f(t) = ka^{t}\), determine os valores de \(k\) e de \(a\).

% Define as alternativas
\newcommand{\alternativas}{%
    \begin{center}
        \begin{tabularx}{\textwidth}{XXX}
            (a) \(k = 1\) e \(a = 2\). &
            (b) \(k = 5\,000\) e \(a = \sqrt[6]{3}\). &
            (c) \(k = 15\,000\) e \(a = \sqrt{3}\). \\[5pt]
            (d) \(k = \dfrac{1}{2}\) e \(a = 3\). &
            (e) \(k = \sqrt{2}\) e \(a = \dfrac{1}{2}\).
        \end{tabularx}
    \end{center}
}

% Define a resposta correta
\newcommand{\resposta}{B}

% Lógica condicional para exibição
\ifdefstring{\atividade}{lista}{%
    \alternativas
    \vspace{0.5em}
    
    \noindent\textbf{Resposta:} letra \textbf{\resposta}.
}{%
    \ifdefstring{\modo}{objetivo}{%
        \alternativas
    }{%
        % modo = discursiva → não mostra alternativas
    }
}
\end{exercicioBanco}


    \noindent\rule{\linewidth}{1pt}  % linha horizontal fina
    
    \phantomsection
    \addcontentsline{toc}{section}{Exercício 10}
    %%%%%%%%%%%%%%%%%%%%%%%%%%%%%%%%%
%
%       EXERCÍCIO
%
%%%%%%%%%%%%%%%%%%%%%%%%%%%%%%%%%

\ifdefstring{\atividade}{prova}{%
    \ifdefstring{\modo}{objetivo}{%
        \renewcommand{\valorquestao}{\ValorQObj\ ponto}
    }{%
        \renewcommand{\valorquestao}{\ValorQDisc\ pontos}
    }%
}%

\begin{exercicioBanco}[\valorquestao]
Computadores utilizam, por padrão, dados em formato binário, em que cada dígito, denominado de bit, pode assumir dois valores (\(0\) ou \(1\)). Para representação de caracteres e outras informações, é necessário fazer uso de uma sequência de bits, o byte. No passado, um byte era composto de \(6\) bits em alguns computadores, mas atualmente tem-se a padronização que o byte é um octeto, ou seja, uma sequência de \(8\) bits. Esse padrão permite representar apenas \(2^{8}\) informações distintas. Se um novo padrão for proposto, de modo que um byte seja capaz de representar pelo menos \(2\,560\) informações distintas, o número de bits em um byte deve passar de 8 para:

% Define as alternativas
\newcommand{\alternativas}{%
    \begin{center}
        \begin{tabularx}{\textwidth}{XXXXX}
            (a) \(10\). &
            (b) \(12\). &
            (c) \(13\). &
            (d) \(18\). &
            (e) \(20\).
        \end{tabularx}
    \end{center}
}

% Define a resposta correta
\newcommand{\resposta}{B}

% Lógica condicional para exibição
\ifdefstring{\atividade}{lista}{%
    \alternativas
    \vspace{0.5em}
    
    \noindent\textbf{Resposta:} letra \textbf{\resposta}.
}{%
    \ifdefstring{\modo}{objetivo}{%
        \alternativas
    }{%
        % modo = discursiva → não mostra alternativas
    }
}
\end{exercicioBanco}


    \noindent\rule{\linewidth}{1pt}  % linha horizontal fina
    
    \phantomsection
    \addcontentsline{toc}{section}{Exercício 11}
    %%%%%%%%%%%%%%%%%%%%%%%%%%%%%%%%%
%
%       EXERCÍCIO
%
%%%%%%%%%%%%%%%%%%%%%%%%%%%%%%%%%

\ifdefstring{\atividade}{prova}{%
    \ifdefstring{\modo}{objetivo}{%
        \renewcommand{\valorquestao}{\ValorQObj\ ponto}
    }{%
        \renewcommand{\valorquestao}{\ValorQDisc\ pontos}
    }%
}%

\begin{exercicioBanco}[\valorquestao]
Os biólogos observaram que, em condições ideais, o número de bactérias \(Q(t)\) em uma cultura cresce exponencialmente com o tempo \(t\), de acordo com a lei \(Q(t) = Q_{0}\cdot e^{kt}\), sendo \(k > 0\) uma constante que depende da natureza das bactérias, \(e\) o número de Euler, aproximadamente \(2,718\), e \(Q_{0}\) é a quantidade inicial de bactérias. Se uma cultura tem inicialmente \(6\,000\) bactérias e, \(20\) minutos depois, aumentou para \(12\,000\), quantas bactérias estarão presentes depois de \(1\) hora?

% Define as alternativas
\newcommand{\alternativas}{%
    \begin{center}
        \begin{tabularx}{\textwidth}{XXXXX}
            (a) \(1,8 \cdot 10^{4}\). &
            (b) \(2,4 \cdot 10^{4}\). &
            (c) \(3,0 \cdot 10^{4}\). &
            (d) \(3,6 \cdot 10^{4}\). &
            (e) \(4,8 \cdot 10^{4}\).
        \end{tabularx}
    \end{center}
}

% Define a resposta correta
\newcommand{\resposta}{E}

% Lógica condicional para exibição
\ifdefstring{\atividade}{lista}{%
    \alternativas
    \vspace{0.5em}
    
    \noindent\textbf{Resposta:} letra \textbf{\resposta}.
}{%
    \ifdefstring{\modo}{objetivo}{%
        \alternativas
    }{%
        % modo = discursiva → não mostra alternativas
    }
}
\end{exercicioBanco}


    \noindent\rule{\linewidth}{1pt}  % linha horizontal fina
    
    \phantomsection
    \addcontentsline{toc}{section}{Exercício 12}
    %%%%%%%%%%%%%%%%%%%%%%%%%%%%%%%%%
%
%       EXERCÍCIO
%
%%%%%%%%%%%%%%%%%%%%%%%%%%%%%%%%%

\ifdefstring{\atividade}{prova}{%
    \ifdefstring{\modo}{objetivo}{%
        \renewcommand{\valorquestao}{\ValorQObj\ ponto}
    }{%
        \renewcommand{\valorquestao}{\ValorQDisc\ pontos}
    }%
}%

\begin{exercicioBanco}[\valorquestao]
O número de bactérias numa cultura, em função do tempo \(t\) (em horas), pode ser expresso por \(N(t) = 256\cdot 2^{0,75t}\). Em quanto tempo, em horas, o número de bactérias será igual a \(2\,048\)?

% Define as alternativas
\newcommand{\alternativas}{%
    \begin{center}
        \begin{tabularx}{\textwidth}{XXXXX}
            (a) \(2\). &
            (b) \(6\). &
            (c) \(8\). &
            (d) \(3\). &
            (e) \(4\).
        \end{tabularx}
    \end{center}
}

% Define a resposta correta
\newcommand{\resposta}{E}

% Lógica condicional para exibição
\ifdefstring{\atividade}{lista}{%
    \alternativas
    \vspace{0.5em}
    
    \noindent\textbf{Resposta:} letra \textbf{\resposta}.
}{%
    \ifdefstring{\modo}{objetivo}{%
        \alternativas
    }{%
        % modo = discursiva → não mostra alternativas
    }
}
\end{exercicioBanco}


    \noindent\rule{\linewidth}{1pt}  % linha horizontal fina
    
    \phantomsection
    \addcontentsline{toc}{section}{Exercício 13}
    \input{../FE_Exercicio13}
    \noindent\rule{\linewidth}{1pt}  % linha horizontal fina
    
    \phantomsection
    \addcontentsline{toc}{section}{Exercício 14}
    \input{../FE_Exercicio14}
    \noindent\rule{\linewidth}{1pt}  % linha horizontal fina
    
    \phantomsection
    \addcontentsline{toc}{section}{Exercício 15}
    %%%%%%%%%%%%%%%%%%%%%%%%%%%%%%%%%
%
%       EXERCÍCIO
%
%%%%%%%%%%%%%%%%%%%%%%%%%%%%%%%%%

\ifdefstring{\atividade}{prova}{%
    \ifdefstring{\modo}{objetivo}{%
        \renewcommand{\valorquestao}{\ValorQObj\ ponto}
    }{%
        \renewcommand{\valorquestao}{\ValorQDisc\ pontos}
    }%
}%

\begin{exercicioBanco}[\valorquestao]
Considere a função \(f(x) = e^{x}\), onde \(e\) é o número de Euler. Seja \(g(x)\) a reta tangente ao gráfico de \(f\) passando pelo ponto \((0,1)\). Qual é o zero de \(g(x)\)?

% Define as alternativas
\newcommand{\alternativas}{%
    \begin{center}
        \begin{tabularx}{\textwidth}{XXXXX}
            (a) \(-2\). &
            (b) \(-1\). &
            (c) \(0\). &
            (d) \(1\). &
            (e) \(2\).
        \end{tabularx}
    \end{center}
}

% Define a resposta correta
\newcommand{\resposta}{E}

% Lógica condicional para exibição
\ifdefstring{\atividade}{lista}{%
    \alternativas
    \vspace{0.5em}
    
    \noindent\textbf{Resposta:} letra \textbf{\resposta}.
}{%
    \ifdefstring{\modo}{objetivo}{%
        \alternativas
    }{%
        % modo = discursiva → não mostra alternativas
    }
}
\end{exercicioBanco}


    \noindent\rule{\linewidth}{1pt}  % linha horizontal fina
    
    \phantomsection
    \addcontentsline{toc}{section}{Exercício 16}
    %%%%%%%%%%%%%%%%%%%%%%%%%%%%%%%%%
%
%       EXERCÍCIO
%
%%%%%%%%%%%%%%%%%%%%%%%%%%%%%%%%%

\ifdefstring{\atividade}{prova}{%
    \ifdefstring{\modo}{objetivo}{%
        \renewcommand{\valorquestao}{\ValorQObj\ ponto}
    }{%
        \renewcommand{\valorquestao}{\ValorQDisc\ pontos}
    }%
}%

\begin{exercicioBanco}[\valorquestao]
O processo de resfriamento de um determinado corpo é descrito por \(T(t) = T_{A} + \alpha \cdot 3^{\beta t}\), onde \(T(t)\) é a temperatura do corpo, em graus Celsius, no instante \(t\) (em minutos), \(T_{A}\) é a temperatura ambiente e \(\alpha\) e \(\beta\) são constantes. O referido corpo foi colocado em um congelador com temperatura de \(-18^\circ\)C. Um termômetro no corpo indicou que ele atingiu \(0^\circ\)C após \(90\) minutos e chegou a \(-16^\circ\)C após \(270\) minutos. Determine o valor de \(t\) para o qual a temperatura do corpo no congelador é apenas \(\tfrac{2}{3}\;{}^\circ\text{C}\) superior à temperatura ambiente. 

% Define as alternativas
\newcommand{\alternativas}{%
    \begin{center}
        \begin{tabularx}{\textwidth}{XXXXX}
            (a) \(50\) minutos. &
            (b) \(100\) minutos. &
            (c) \(360\) minutos. &
            (d) \(900\) minutos. &
            (e) \(1\,000\) minutos.
        \end{tabularx}
    \end{center}
}

% Define a resposta correta
\newcommand{\resposta}{C}

% Lógica condicional para exibição
\ifdefstring{\atividade}{lista}{%
    \alternativas
    \vspace{0.5em}
    
    \noindent\textbf{Resposta:} letra \textbf{\resposta}.
}{%
    \ifdefstring{\modo}{objetivo}{%
        \alternativas
    }{%
        % modo = discursiva → não mostra alternativas
    }
}
\end{exercicioBanco}


    \noindent\rule{\linewidth}{1pt}  % linha horizontal fina
    
    \phantomsection
    \addcontentsline{toc}{section}{Exercício 17}
    %%%%%%%%%%%%%%%%%%%%%%%%%%%%%%%%%
%
%       EXERCÍCIO
%
%%%%%%%%%%%%%%%%%%%%%%%%%%%%%%%%%

\ifdefstring{\atividade}{prova}{%
    \ifdefstring{\modo}{objetivo}{%
        \renewcommand{\valorquestao}{\ValorQObj\ ponto}
    }{%
        \renewcommand{\valorquestao}{\ValorQDisc\ pontos}
    }%
}%

\begin{exercicioBanco}[\valorquestao]
Admita que um tipo de eucalipto tenha expectativa de crescimento exponencial, nos primeiros anos após seu plantio, modelado pela função
\[
y(t)=a^{\,t-1},
\]
na qual \(y\) representa a altura da planta em metros, \(t\) é considerado em anos, e \(a\) é uma constante maior que \(1\). O gráfico representa a função \(y\).

\begin{center}
    \begin{tikzpicture}[scale=1.2]

        % Eixos
        \draw[->] (-0.1,0) -- (2,0) node[right] {\small$t$ (tempo)};
        \draw[->] (0,-0.1) -- (0,2) node[above] {\small$y$ (metro)};

        % Gráfico de y = a^x (exponencial)
        \draw[domain=0:1.8, smooth, variable=\x, blue, thick] plot ({\x}, {exp(\x/2)-0.5});
        
        \draw node at (-0.4, 0.5) {\small\(0,5\)};
        \draw node at (1.5, -0.3) {\small\(6\)};
        \draw node at (-0.3, {exp(1.5/2)-0.5}) {\small\(32\)};
        
        \draw[dashed] (1.5, -0.1) -- (1.5,{exp(1.5/2)-0.5}) -- (-0.1, {exp(1.5/2)-0.5});
    \end{tikzpicture}
\end{center}

Admita ainda que \(y(0)\) fornece a altura da muda quando plantada, e deseja-se cortar os eucaliptos quando as mudas crescerem \(7,5\) m após o plantio. O tempo entre a plantação e o corte, em anos, é igual a:


% Define as alternativas
\newcommand{\alternativas}{%
    \begin{center}
        \begin{tabularx}{\textwidth}{XXXXX}
            (a) \(3\). &
            (b) \(4\). &
            (c) \(6\). &
            (d) \(8\). &
            (e) \(1\).
        \end{tabularx}
    \end{center}
}

% Define a resposta correta
\newcommand{\resposta}{B}

% Lógica condicional para exibição
\ifdefstring{\atividade}{lista}{%
    \alternativas
    \vspace{0.5em}
    
    \noindent\textbf{Resposta:} letra \textbf{\resposta}.
}{%
    \ifdefstring{\modo}{objetivo}{%
        \alternativas
    }{%
        % modo = discursiva → não mostra alternativas
    }
}
\end{exercicioBanco}


    \noindent\rule{\linewidth}{1pt}  % linha horizontal fina
    
    \phantomsection
    \addcontentsline{toc}{section}{Exercício 18}
    %%%%%%%%%%%%%%%%%%%%%%%%%%%%%%%%%
%
%       EXERCÍCIO
%
%%%%%%%%%%%%%%%%%%%%%%%%%%%%%%%%%

\ifdefstring{\atividade}{prova}{%
    \ifdefstring{\modo}{objetivo}{%
        \renewcommand{\valorquestao}{\ValorQObj\ ponto}
    }{%
        \renewcommand{\valorquestao}{\ValorQDisc\ pontos}
    }%
}%

\begin{exercicioBanco}[\valorquestao]
Considere as funções \(f(x) = 3^{x}\) e \(g(x) = x^{3}\), definidas para todo número real \(x\). O número de soluções da equação \(f\big(g(x)\big) = g\big(f(x)\big)\) é igua a:

% Define as alternativas
\newcommand{\alternativas}{%
    \begin{center}
        \begin{tabularx}{\textwidth}{XXXXX}
            (a) \(0\). &
            (b) \(1\). &
            (c) \(2\). &
            (d) \(3\). &
            (e) \(4\).
        \end{tabularx}
    \end{center}
}

% Define a resposta correta
\newcommand{\resposta}{D}

% Lógica condicional para exibição
\ifdefstring{\atividade}{lista}{%
    \alternativas
    \vspace{0.5em}
    
    \noindent\textbf{Resposta:} letra \textbf{\resposta}.
}{%
    \ifdefstring{\modo}{objetivo}{%
        \alternativas
    }{%
        % modo = discursiva → não mostra alternativas
    }
}
\end{exercicioBanco}


    \noindent\rule{\linewidth}{1pt}  % linha horizontal fina
    
    \phantomsection
    \addcontentsline{toc}{section}{Exercício 19}
    \input{../FE_Exercicio19}
    \noindent\rule{\linewidth}{1pt}  % linha horizontal fina
    
    \phantomsection
    \addcontentsline{toc}{section}{Exercício 20}
    %%%%%%%%%%%%%%%%%%%%%%%%%%%%%%%%%
%
%       EXERCÍCIO
%
%%%%%%%%%%%%%%%%%%%%%%%%%%%%%%%%%

\ifdefstring{\atividade}{prova}{%
    \ifdefstring{\modo}{objetivo}{%
        \renewcommand{\valorquestao}{\ValorQObj\ ponto}
    }{%
        \renewcommand{\valorquestao}{\ValorQDisc\ pontos}
    }%
}%

\begin{exercicioBanco}[\valorquestao]
Determine o conjunto solução da equação exponencial \(0,125^{4-5x} = 0,25^{2x-1}\).

% Define as alternativas
\newcommand{\alternativas}{%
    \begin{center}
        \begin{tabularx}{\textwidth}{XXXXX}
            (a) \(-\dfrac{2}{15}\). &
            (b) \(4\). &
            (c) \(-11\). &
            (d) \(\dfrac{1}{9}\). &
            (e) \(\dfrac{14}{19}\).
        \end{tabularx}
    \end{center}
}

% Define a resposta correta
\newcommand{\resposta}{E}

% Lógica condicional para exibição
\ifdefstring{\atividade}{lista}{%
    \alternativas
    \vspace{0.5em}
    
    \noindent\textbf{Resposta:} letra \textbf{\resposta}.
}{%
    \ifdefstring{\modo}{objetivo}{%
        \alternativas
    }{%
        % modo = discursiva → não mostra alternativas
    }
}
\end{exercicioBanco}


    \noindent\rule{\linewidth}{1pt}  % linha horizontal fina
    
    \phantomsection
    \addcontentsline{toc}{section}{Exercício 21}
    %%%%%%%%%%%%%%%%%%%%%%%%%%%%%%%%%
%
%       EXERCÍCIO
%
%%%%%%%%%%%%%%%%%%%%%%%%%%%%%%%%%

\ifdefstring{\atividade}{prova}{%
    \ifdefstring{\modo}{objetivo}{%
        \renewcommand{\valorquestao}{\ValorQObj\ ponto}
    }{%
        \renewcommand{\valorquestao}{\ValorQDisc\ pontos}
    }%
}%

\begin{exercicioBanco}[\valorquestao]
Determine o conjunto solução da equação \(3\cdot 5^{x^{2}} + 3^{x^{2}+1} - 8\cdot 3^{x^{2}} = 0\).

% Define as alternativas
\newcommand{\alternativas}{%
    \begin{center}
        \begin{tabularx}{\textwidth}{XXXXX}
            (a) \(\{-1,1\}\). &
            (b) \(\{0\}\). &
            (c) \(\varnothing\). &
            (d) \(\{-1,0,1\}\). &
            (e) \(\bbR\).
        \end{tabularx}
    \end{center}
}

% Define a resposta correta
\newcommand{\resposta}{A}

% Lógica condicional para exibição
\ifdefstring{\atividade}{lista}{%
    \alternativas
    \vspace{0.5em}
    
    \noindent\textbf{Resposta:} letra \textbf{\resposta}.
}{%
    \ifdefstring{\modo}{objetivo}{%
        \alternativas
    }{%
        % modo = discursiva → não mostra alternativas
    }
}
\end{exercicioBanco}


    \noindent\rule{\linewidth}{1pt}  % linha horizontal fina
    
    \phantomsection
    \addcontentsline{toc}{section}{Exercício 22}
    %%%%%%%%%%%%%%%%%%%%%%%%%%%%%%%%%
%
%       EXERCÍCIO
%
%%%%%%%%%%%%%%%%%%%%%%%%%%%%%%%%%

\ifdefstring{\atividade}{prova}{%
    \ifdefstring{\modo}{objetivo}{%
        \renewcommand{\valorquestao}{\ValorQObj\ ponto}
    }{%
        \renewcommand{\valorquestao}{\ValorQDisc\ pontos}
    }%
}%

\begin{exercicioBanco}[\valorquestao]
O conjunto solução da inequação \(0,5^{1-x} > 1\) é:

% Define as alternativas
\newcommand{\alternativas}{%
    \begin{center}
        \begin{tabularx}{\textwidth}{XXX}
            (a) \(S = \{x \in \bbR \mid x > 1\}\). &
            (b) \(S = \{x \in \bbR \mid x > 1\}\). &
            (c) \(S = \{x \in \bbR \mid x > 0\}\). \\[5pt]
            (d) \(S = \{x \in \bbR \mid x < 0\}\). &
            (e) \(\bbR\).
        \end{tabularx}
    \end{center}
}

% Define a resposta correta
\newcommand{\resposta}{E}

% Lógica condicional para exibição
\ifdefstring{\atividade}{lista}{%
    \alternativas
    \vspace{0.5em}
    
    \noindent\textbf{Resposta:} letra \textbf{\resposta}.
}{%
    \ifdefstring{\modo}{objetivo}{%
        \alternativas
    }{%
        % modo = discursiva → não mostra alternativas
    }
}
\end{exercicioBanco}


    \noindent\rule{\linewidth}{1pt}  % linha horizontal fina
    
    \phantomsection
    \addcontentsline{toc}{section}{Exercício 23}
    \input{../FE_Exercicio23}
    \noindent\rule{\linewidth}{1pt}  % linha horizontal fina
    
    \phantomsection
    \addcontentsline{toc}{section}{Exercício 24}
    %%%%%%%%%%%%%%%%%%%%%%%%%%%%%%%%%
%
%       EXERCÍCIO
%
%%%%%%%%%%%%%%%%%%%%%%%%%%%%%%%%%

\ifdefstring{\atividade}{prova}{%
    \ifdefstring{\modo}{objetivo}{%
        \renewcommand{\valorquestao}{\ValorQObj\ ponto}
    }{%
        \renewcommand{\valorquestao}{\ValorQDisc\ pontos}
    }%
}%

\begin{exercicioBanco}[\valorquestao]
O conjunto solução da inequação \(\big(3^{\frac{x}{2}}\big)^{x-1} \ge \left(\dfrac{3}{9}\right)^{x-3}\) é:

% Define as alternativas
\newcommand{\alternativas}{%
    \begin{center}
        \begin{tabularx}{\textwidth}{XXX}
            (a) \(]-\infty,-3] \cup [2, \infty[\). &
            (b) \([-3,2]\). &
            (c) \(S = \{x \in \bbR \mid x \le -3\}\). \\[5pt]
            (d) \(S = \{x \in \bbR \mid x \ge 2\}\). &
            (e) \(\bbR\).
        \end{tabularx}
    \end{center}
}

% Define a resposta correta
\newcommand{\resposta}{A}

% Lógica condicional para exibição
\ifdefstring{\atividade}{lista}{%
    \alternativas
    \vspace{0.5em}
    
    \noindent\textbf{Resposta:} letra \textbf{\resposta}.
}{%
    \ifdefstring{\modo}{objetivo}{%
        \alternativas
    }{%
        % modo = discursiva → não mostra alternativas
    }
}
\end{exercicioBanco}


    \noindent\rule{\linewidth}{1pt}  % linha horizontal fina
    
    \phantomsection
    \addcontentsline{toc}{section}{Exercício 25}
    \input{../FE_Exercicio25}
    \noindent\rule{\linewidth}{1pt}  % linha horizontal fina
    
\end{document}
%*************** FINAL DO DOCUMENTO ***************
