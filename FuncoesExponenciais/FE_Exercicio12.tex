%%%%%%%%%%%%%%%%%%%%%%%%%%%%%%%%%
%
%       EXERCÍCIO
%
%%%%%%%%%%%%%%%%%%%%%%%%%%%%%%%%%

\ifdefstring{\atividade}{prova}{%
    \ifdefstring{\modo}{objetivo}{%
        \renewcommand{\valorquestao}{\ValorQObj\ ponto}
    }{%
        \renewcommand{\valorquestao}{\ValorQDisc\ pontos}
    }%
}%

\begin{exercicioBanco}[\valorquestao]
O processo de resfriamento de um determinado corpo é descrito por \(T(t) = T_{A} + \alpha \cdot 3^{\beta t}\), onde \(T(t)\) é a temperatura do corpo, em graus Celsius, no instante \(t\) (em minutos), \(T_{A}\) é a temperatura ambiente e \(\alpha\) e \(\beta\) são constantes. O referido corpo foi colocado em um congelador com temperatura de \(-18^\circ\)C. Um termômetro no corpo indicou que ele atingiu \(0^\circ\)C após \(90\) minutos e chegou a \(-16^\circ\)C após \(270\) minutos. Determine o valor de \(t\) para o qual a temperatura do corpo no congelador é apenas \(\left(\tfrac{2}{3}\right)^\circ\text{C}\) superior à temperatura ambiente. 

% Define as alternativas
\newcommand{\alternativas}{%
    \begin{center}
        \begin{tabularx}{\textwidth}{XXXXX}
            (a) \(50\) minutos. &
            (b) \(100\) minutos. &
            (c) \(360\) minutos. &
            (d) \(900\) minutos. &
            (e) \(1\,000\) minutos.
        \end{tabularx}
    \end{center}
}

% Define a resposta correta
\newcommand{\resposta}{C}

% Lógica condicional para exibição
\ifdefstring{\atividade}{lista}{%
    \alternativas
    \vspace{0.5em}
    
    \noindent\textbf{Resposta:} letra \textbf{\resposta}.
}{%
    \ifdefstring{\modo}{objetivo}{%
        \alternativas
    }{%
        % modo = discursiva → não mostra alternativas
    }
}
\end{exercicioBanco}

