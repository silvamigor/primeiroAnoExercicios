%%%%%%%%%%%%%%%%%%%%%%%%%%%%%%%%%
%
%       EXERCÍCIO
%
%%%%%%%%%%%%%%%%%%%%%%%%%%%%%%%%%

\ifdefstring{\atividade}{prova}{%
    \ifdefstring{\modo}{objetivo}{%
        \renewcommand{\valorquestao}{\ValorQObj\ ponto}
    }{%
        \renewcommand{\valorquestao}{\ValorQDisc\ pontos}
    }%
}%

\begin{exercicioBanco}[\valorquestao]
O decaimento radioativo de uma substância se dá de acordo com a fórmula \(r(t) = C\cdot 3^{-6t}\), com \(C\) sendo uma constante diferente de zero e \(r(t)\) a quantidade de radioatividade presente na substância após \(t\) segundos desde o início do decaimento. O valor de \(t\), em segundos, para que a substância fique com a terça parte da radioatividade que tinha inicialmente é igual a:

% Define as alternativas
\newcommand{\alternativas}{%
    \begin{center}
        \begin{tabularx}{\textwidth}{XXXXX}
            (a) \(\dfrac{1}{4}\). &
            (b) \(\dfrac{1}{5}\). &
            (c) \(\dfrac{1}{3}\). &
            (d) \(\dfrac{1}{6}\). &
            (e) \(\dfrac{2}{5}\).
        \end{tabularx}
    \end{center}
}

% Define a resposta correta
\newcommand{\resposta}{D}

% Lógica condicional para exibição
\ifdefstring{\atividade}{lista}{%
    \alternativas
    \vspace{0.5em}
    
    \noindent\textbf{Resposta:} letra \textbf{\resposta}.
}{%
    \ifdefstring{\modo}{objetivo}{%
        \alternativas
    }{%
        % modo = discursiva → não mostra alternativas
    }
}
\end{exercicioBanco}

