%%%%%%%%%%%%%%%%%%%%%%%%%%%%%%%%%
%
%       EXERCÍCIO
%
%%%%%%%%%%%%%%%%%%%%%%%%%%%%%%%%%

\ifdefstring{\atividade}{prova}{%
    \ifdefstring{\modo}{objetivo}{%
        \renewcommand{\valorquestao}{\ValorQObj\ ponto}
    }{%
        \renewcommand{\valorquestao}{\ValorQDisc\ pontos}
    }%
}%

\begin{exercicioBanco}[\valorquestao]
Assinale a alternativa correta.

% Define as alternativas
\newcommand{\alternativas}{%
    \begin{center}
        \begin{tabularx}{\textwidth}{XX}
            (a) \(16^{\frac{3}{4}} - 27^{\frac{2}{3}} = 1\). &
            (b) Se \((\sqrt{2})^{x} = 64\), então \(x = 12\). \\[7pt]
            (c) \((-1)^{2\,025} = 1\). &
            (d) \(\dfrac{99^{15}}{33^{30}} = 3^{-15}\). \\[15pt]
            \multicolumn{2}{l}{(e) A solução da inequação \((\frac{1}{3})^{x} \le 27\) é \(S = \{x \in \bbR \mid x \le -3\}\)}.
        \end{tabularx}
    \end{center}
}

% Define a resposta correta
\newcommand{\resposta}{D}

% Lógica condicional para exibição
\ifdefstring{\atividade}{lista}{%
    \alternativas
    \vspace{0.5em}
    
    \noindent\textbf{Resposta:} letra \textbf{\resposta}.
}{%
    \ifdefstring{\modo}{objetivo}{%
        \alternativas
    }{%
        % modo = discursiva → não mostra alternativas
    }
}
\end{exercicioBanco}

