%%%%%%%%%%%%%%%%%%%%%%%%%%%%%%%%%
%
%       EXERCÍCIO
%
%%%%%%%%%%%%%%%%%%%%%%%%%%%%%%%%%

\ifdefstring{\atividade}{prova}{%
    \ifdefstring{\modo}{objetivo}{%
        \renewcommand{\valorquestao}{\ValorQObj\ ponto}
    }{%
        \renewcommand{\valorquestao}{\ValorQDisc\ pontos}
    }%
}%

\begin{exercicioBanco}[\valorquestao]
O conjunto solução da inequação 
%
\[
    \left(\dfrac{1}{7^{x}}\right)^{x^{3}-4} - 7(7^{x^{2}+1})^{2x-1} \ge 0
\]
%
é:

% Define as alternativas
\newcommand{\alternativas}{%
    \begin{center}
        \begin{tabularx}{\textwidth}{XXX}
            (a) \([-2,-1]\). &
            (b) \([0,1]\). &
            (c) \(]-\infty,-2] \cup [-1,0] \cup [1,\infty[\). \\[5pt]
            (d) \([0,\infty[\). &
            (e) \([-2,-1] \cup [0,1]\).
        \end{tabularx}
    \end{center}
}

% Define a resposta correta
\newcommand{\resposta}{E}

% Lógica condicional para exibição
\ifdefstring{\atividade}{lista}{%
    \alternativas
    \vspace{0.5em}
    
    \noindent\textbf{Resposta:} letra \textbf{\resposta}.
}{%
    \ifdefstring{\modo}{objetivo}{%
        \alternativas
    }{%
        % modo = discursiva → não mostra alternativas
    }
}
\end{exercicioBanco}

