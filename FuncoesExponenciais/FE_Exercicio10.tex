%%%%%%%%%%%%%%%%%%%%%%%%%%%%%%%%%
%
%       EXERCÍCIO
%
%%%%%%%%%%%%%%%%%%%%%%%%%%%%%%%%%

\ifdefstring{\atividade}{prova}{%
    \ifdefstring{\modo}{objetivo}{%
        \renewcommand{\valorquestao}{\ValorQObj\ ponto}
    }{%
        \renewcommand{\valorquestao}{\ValorQDisc\ pontos}
    }%
}%

\begin{exercicioBanco}[\valorquestao]
Computadores utilizam, por padrão, dados em formato binário, em que cada dígito, denominado de bit, pode assumir dois valores (\(0\) ou \(1\)). Para representação de caracteres e outras informações, é necessário fazer uso de uma sequência de bits, o byte. No passado, um byte era composto de \(6\) bits em alguns computadores, mas atualmente tem-se a padronização que o byte é um octeto, ou seja, uma sequência de \(8\) bits. Esse padrão permite representar apenas \(2^{8}\) informações distintas. Se um novo padrão for proposto, de modo que um byte seja capaz de representar pelo menos \(2\,560\) informações distintas, o número de bits em um byte deve passar de 8 para:

% Define as alternativas
\newcommand{\alternativas}{%
    \begin{center}
        \begin{tabularx}{\textwidth}{XXXXX}
            (a) \(10\). &
            (b) \(12\). &
            (c) \(13\). &
            (d) \(18\). &
            (e) \(20\).
        \end{tabularx}
    \end{center}
}

% Define a resposta correta
\newcommand{\resposta}{B}

% Lógica condicional para exibição
\ifdefstring{\atividade}{lista}{%
    \alternativas
    \vspace{0.5em}
    
    \noindent\textbf{Resposta:} letra \textbf{\resposta}.
}{%
    \ifdefstring{\modo}{objetivo}{%
        \alternativas
    }{%
        % modo = discursiva → não mostra alternativas
    }
}
\end{exercicioBanco}

