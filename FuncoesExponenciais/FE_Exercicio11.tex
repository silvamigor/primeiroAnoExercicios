%%%%%%%%%%%%%%%%%%%%%%%%%%%%%%%%%
%
%       EXERCÍCIO
%
%%%%%%%%%%%%%%%%%%%%%%%%%%%%%%%%%

\ifdefstring{\atividade}{prova}{%
    \ifdefstring{\modo}{objetivo}{%
        \renewcommand{\valorquestao}{\ValorQObj\ ponto}
    }{%
        \renewcommand{\valorquestao}{\ValorQDisc\ pontos}
    }%
}%

\begin{exercicioBanco}[\valorquestao]
Os biólogos observaram que, em condições ideais, o número de bactérias \(Q(t)\) em uma cultura cresce exponencialmente com o tempo \(t\), de acordo com a lei \(Q(t) = Q_{0}\cdot e^{kt}\), sendo \(k > 0\) uma constante que depende da natureza das bactérias, \(e\) o número de Euler, aproximadamente \(2,718\), e \(Q_{0}\) é a quantidade inicial de bactérias. Se uma cultura tem inicialmente \(6\,000\) bactérias e, \(20\) minutos depois, aumentou para \(12\,000\), quantas bactérias estarão presentes depois de \(1\) hora?

% Define as alternativas
\newcommand{\alternativas}{%
    \begin{center}
        \begin{tabularx}{\textwidth}{XXXXX}
            (a) \(1,8 \cdot 10^{4}\). &
            (b) \(2,4 \cdot 10^{4}\). &
            (c) \(3,0 \cdot 10^{4}\). &
            (d) \(3,6 \cdot 10^{4}\). &
            (e) \(4,8 \cdot 10^{4}\).
        \end{tabularx}
    \end{center}
}

% Define a resposta correta
\newcommand{\resposta}{E}

% Lógica condicional para exibição
\ifdefstring{\atividade}{lista}{%
    \alternativas
    \vspace{0.5em}
    
    \noindent\textbf{Resposta:} letra \textbf{\resposta}.
}{%
    \ifdefstring{\modo}{objetivo}{%
        \alternativas
    }{%
        % modo = discursiva → não mostra alternativas
    }
}
\end{exercicioBanco}

