%%%%%%%%%%%%%%%%%%%%%%%%%%%%%%%%%
%
%       EXERCÍCIO
%
%%%%%%%%%%%%%%%%%%%%%%%%%%%%%%%%%

\ifdefstring{\atividade}{prova}{%
    \ifdefstring{\modo}{objetivo}{%
        \renewcommand{\valorquestao}{\ValorQObj\ ponto}
    }{%
        \renewcommand{\valorquestao}{\ValorQDisc\ pontos}
    }%
}%

\begin{exercicioBanco}[\valorquestao]
Em uma determinada região litorânea, a maré oscila segundo a função \(h(t) = 3 - 2\sen\left(\dfrac{\pi t}{12}\right)\), sendo h a altura em metros, que a maré atinge no tempo t em horas, medido a partir de 6 h da manhã. Uma embarcação, que se encontra encalhada às 11 h da manhã, precisa de uma profundidade mínima de 2 metros para navegar. Quantas horas os tripulantes dessa embarcação ainda terão que esperar para prosseguirem viagem?

% Define as alternativas
\newcommand{\alternativas}{%
    \begin{center}
        \begin{tabularx}{\textwidth}{XXXXX}
            (a) 4 h. &
            (b) 5 h. &
            (c) 6 h. &
            (d) 7 h. &
            (e) 8 h.
        \end{tabularx}
    \end{center}
}

% Define a resposta correta
\newcommand{\resposta}{B}

% Lógica condicional para exibição
\ifdefstring{\atividade}{lista}{%
    \alternativas
    \vspace{0.5em}
    
    \noindent\textbf{Resposta:} letra \textbf{\resposta}.
}{%
    \ifdefstring{\modo}{objetivo}{%
        \alternativas
    }{%
        % modo = discursiva → não mostra alternativas
    }
}
\end{exercicioBanco}

