%%%%%%%%%%%%%%%%%%%%%%%%%%%%%%%%%
%
%       EXERCÍCIO
%
%%%%%%%%%%%%%%%%%%%%%%%%%%%%%%%%%

\ifdefstring{\atividade}{prova}{%
    \ifdefstring{\modo}{objetivo}{%
        \renewcommand{\valorquestao}{\ValorQObj\ ponto}
    }{%
        \renewcommand{\valorquestao}{\ValorQDisc\ pontos}
    }%
}%

\begin{exercicioBanco}[\valorquestao]
Sejam as funções \(f(x) = 2\sen(x)\) e \(g(x) = \sen(2x)\). A respeito delas, assinale a alternativa correta.
%
% Define as alternativas
\newcommand{\alternativas}{%
    \begin{center}
        \begin{tabularx}{\textwidth}{X}
            (a) O período de \(f(x)\) é o dobro do período de \(g(x)\). \\[5pt]
            (b) As funções \(f(x)\) e \(g(x)\) possuem os mesmos zeros. \\[5pt]
            (c) O máximo de \(f(x)\) é igual ao máximo de \(g(x)\). \\[5pt]
            (d) O máximo de \(g(x)\) é o dobro do máximo de \(f(x)\). \\[5pt]
            (e) O período de \(g(x)\) é o dobro do período de \(f(x)\).
        \end{tabularx}
    \end{center}
}

% Define a resposta correta
\newcommand{\resposta}{A}

% Lógica condicional para exibição
\ifdefstring{\atividade}{lista}{%
    \alternativas
    \vspace{0.5em}
    
    \noindent\textbf{Resposta:} letra \textbf{\resposta}.
}{%
    \ifdefstring{\modo}{objetivo}{%
        \alternativas
    }{%
        % modo = discursiva → não mostra alternativas
    }
}
\end{exercicioBanco}

