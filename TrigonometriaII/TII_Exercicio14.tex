%%%%%%%%%%%%%%%%%%%%%%%%%%%%%%%%%
%
%       EXERCÍCIO
%
%%%%%%%%%%%%%%%%%%%%%%%%%%%%%%%%%

\ifdefstring{\atividade}{prova}{%
    \ifdefstring{\modo}{objetivo}{%
        \renewcommand{\valorquestao}{\ValorQObj\ ponto}
    }{%
        \renewcommand{\valorquestao}{\ValorQDisc\ pontos}
    }%
}%

\begin{exercicioBanco}[\valorquestao]
Se \(f(x) = a + b\cdot\sen(x)\) tem como gráfico

\medskip

\begin{figure}[H]
    \centering
    \includegraphics[width=0.3\textwidth]{TII_Exercicio14_Fig.png}
\end{figure}

\smallskip

Determine o valor de \(a\) e o valor de \(b\).

% Define as alternativas
\newcommand{\alternativas}{%
    \begin{center}
        \begin{tabularx}{\textwidth}{XXX}
            (a) \(a = -2\) e \(b = 1\). &
            (b) \(a = -1\) e \(b = 2\). &
            (c) \(a = 1\) e \(b = -1\). \\[5pt]
            (d) \(a = 1\) e \(b = -2\). &
            (e) \(a = 1\) e \(b = 1\).
        \end{tabularx}
    \end{center}
}

% Define a resposta correta
\newcommand{\resposta}{D}

% Lógica condicional para exibição
\ifdefstring{\atividade}{lista}{%
    \alternativas
    \vspace{0.5em}
    
    \noindent\textbf{Resposta:} letra \textbf{\resposta}.
}{%
    \ifdefstring{\modo}{objetivo}{%
        \alternativas
    }{%
        % modo = discursiva → não mostra alternativas
    }
}
\end{exercicioBanco}

