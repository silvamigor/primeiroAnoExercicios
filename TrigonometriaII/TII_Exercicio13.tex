%%%%%%%%%%%%%%%%%%%%%%%%%%%%%%%%%
%
%       EXERCÍCIO
%
%%%%%%%%%%%%%%%%%%%%%%%%%%%%%%%%%

\ifdefstring{\atividade}{prova}{%
    \ifdefstring{\modo}{objetivo}{%
        \renewcommand{\valorquestao}{\ValorQObj\ ponto}
    }{%
        \renewcommand{\valorquestao}{\ValorQDisc\ pontos}
    }%
}%

\begin{exercicioBanco}[\valorquestao]
Considere que o volume de ar nos pulmões de um ser humano adulto, em litro, é descritos pela função
%
\[
    V(t) = a + b\sen(ct),
\]
%
onde \(t\) é o tempo em segundos. Sabendo-se que o volume máximo de ar é 4 litros, o mínimo é 2 litros e que o período de \(V(t)\) é igual a 6. Admitindo que \(b, c > 0\), determine \(a\), \(b\) e \(c\).

% Define as alternativas
\newcommand{\alternativas}{%
    \begin{center}
        \begin{tabularx}{\textwidth}{XXX}
            (a) \(a = 2\), \(b = 2\) e \(c = 6\). &
            (b) \(a = 1\), \(b = 3\) e \(c = \frac{\pi}{6}\). &
            (c) \(a = 3\), \(b = 1\) e \(c = \frac{\pi}{3}\). \\[5pt]
            (d) \(a = 3\), \(b = 1\) e \(c = 6\). &
            (e) \(a = 2\), \(b = 3\) e \(c = \pi\).
        \end{tabularx}
    \end{center}
}

% Define a resposta correta
\newcommand{\resposta}{C}

% Lógica condicional para exibição
\ifdefstring{\atividade}{lista}{%
    \alternativas
    \vspace{0.5em}
    
    \noindent\textbf{Resposta:} letra \textbf{\resposta}.
}{%
    \ifdefstring{\modo}{objetivo}{%
        \alternativas
    }{%
        % modo = discursiva → não mostra alternativas
    }
}
\end{exercicioBanco}

