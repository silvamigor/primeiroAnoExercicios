%%%%%%%%%%%%%%%%%%%%%%%%%%%%%%%%%
%
%       EXERCÍCIO
%
%%%%%%%%%%%%%%%%%%%%%%%%%%%%%%%%%

\ifdefstring{\atividade}{prova}{%
    \ifdefstring{\modo}{objetivo}{%
        \renewcommand{\valorquestao}{\ValorQObj\ ponto}
    }{%
        \renewcommand{\valorquestao}{\ValorQDisc\ pontos}
    }%
}%

\begin{exercicioBanco}[\valorquestao]
Um cientista, em seus estudos para modelar a pressão arterial de uma pessoa, utiliza uma função do tipo
\[
P(t) = A + B\cos(kt),
\]
em que $A$, $B$ e $k$ são constantes reais positivas e $t$ representa a variável tempo, medida em segundos.
Considere que um batimento cardíaco representa o intervalo de tempo entre duas sucessivas pressões máximas.

Ao analisar um caso específico, o cientista obteve os seguintes dados:
\begin{itemize}
    \item pressão mínima: $78$;
    \item pressão máxima: $120$;
    \item número de batimentos cardíacos por minuto: $90$.
\end{itemize}

Determine a função $P(t)$ obtida por este cientista, ao analisar o caso específico.

% Define as alternativas
\newcommand{\alternativas}{%
    \begin{center}
        \begin{tabularx}{\textwidth}{XXX}
            (a) \(P(t) = 99 + 21\cdot\cos(3\pi t)\). &
            (b) \(P(t) = 78 + 42\cdot\cos(3\pi t)\). &
            (c) \(P(t) = 99 + 21\cdot\cos(2\pi t)\). \\[5pt]
            (d) \(P(t) = 99 + 21\cdot\cos(t)\). &
            (e) \(P(t) = 78 + 42\cdot\cos(t)\).
        \end{tabularx}
    \end{center}
}

% Define a resposta correta
\newcommand{\resposta}{A}

% Lógica condicional para exibição
\ifdefstring{\atividade}{lista}{%
    \alternativas
    \vspace{0.5em}
    
    \noindent\textbf{Resposta:} letra \textbf{\resposta}.
}{%
    \ifdefstring{\modo}{objetivo}{%
        \alternativas
    }{%
        % modo = discursiva → não mostra alternativas
    }
}
\end{exercicioBanco}

