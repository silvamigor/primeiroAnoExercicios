%%%%%%%%%%%%%%%%%%%%%%%%%%%%%%%%%
%
%       EXERCÍCIO
%
%%%%%%%%%%%%%%%%%%%%%%%%%%%%%%%%%

\ifdefstring{\atividade}{prova}{%
    \ifdefstring{\modo}{objetivo}{%
        \renewcommand{\valorquestao}{\ValorQObj\ ponto}
    }{%
        \renewcommand{\valorquestao}{\ValorQDisc\ pontos}
    }%
}%

\begin{exercicioBanco}[\valorquestao]
Em 2014 foi inaugurada a maior roda-gigante do mundo, a High Roller, situada em Las Vegas. A figura representa um esboço dessa roda-gigante, no qual o ponto $A$ representa uma de suas cadeiras:

\medskip

\begin{figure}[H]
    \centering
    \includegraphics[width=0.2\textwidth]{TII_Exercicio17A_Fig.png}
\end{figure}

\smallskip

A partir da posição indicada, em que o segmento $OA$ se encontra paralelo ao plano do solo, rotaciona-se a High Roller no sentido anti-horário, em torno do ponto $O$. Sejam $t$ o ângulo determinado pelo segmento $OA$ em relação à sua posição inicial, e $f$ a função que descreve a altura do ponto $A$, em relação ao solo, em função de $t$. Após duas voltas completas, $f$ tem o seguinte gráfico:

\medskip

\begin{figure}[H]
    \centering
    \includegraphics[width=0.4\textwidth]{TII_Exercicio17B_Fig.png}
\end{figure}

\smallskip

Determine a expressão da função \(f\).

% Define as alternativas
\newcommand{\alternativas}{%
    \begin{center}
        \begin{tabularx}{\textwidth}{XXX}
            (a) \(f(t) = 80\sen(t) + 88\). &
            (b) \(f(t) = 80\cos(t) + 88\). &
            (c) \(f(t) = 88\cos(t) + 168\). \\[5pt]
            (d) \(f(t) = 168\sen(t) + 88\cos(t)\). &
            (e) \(f(t) = 88\sen(t) + 168\cos(t)\).
        \end{tabularx}
    \end{center}
}

% Define a resposta correta
\newcommand{\resposta}{A}

% Lógica condicional para exibição
\ifdefstring{\atividade}{lista}{%
    \alternativas
    \vspace{0.5em}
    
    \noindent\textbf{Resposta:} letra \textbf{\resposta}.
}{%
    \ifdefstring{\modo}{objetivo}{%
        \alternativas
    }{%
        % modo = discursiva → não mostra alternativas
    }
}
\end{exercicioBanco}

