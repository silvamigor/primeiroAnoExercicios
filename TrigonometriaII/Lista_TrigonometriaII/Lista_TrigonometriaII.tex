\documentclass[a4paper,12pt]{article}

% --------------------------------------------------------
% CODIFICAÇÃO E TIPOGRAFIA
% --------------------------------------------------------
\usepackage[utf8]{inputenc}       % Permite acentuação direta no código-fonte
\usepackage[T1]{fontenc}          % Usa codificação T1 (melhor para português)
\usepackage[brazil]{babel}        % Tradução e hifenização para português
\usepackage{microtype}            % Melhora o espaçamento entre letras/palavras
\usepackage{parskip}              % Espaço entre parágrafos em vez de recuo

% --------------------------------------------------------
% PACOTES MATEMÁTICOS
% --------------------------------------------------------
\usepackage{amsmath}              % Ambientes matemáticos como align, equation
%\usepackage{amssymb}              % Símbolos matemáticos adicionais
\usepackage{amsthm}               % Ambientes de teoremas
\usepackage{mathtools}            % Extensões do amsmath (ex: \coloneqq)
\usepackage{mathrsfs}             % Fonte caligráfica com \mathscr
\usepackage{bbm}                  % Indicador \mathbbm{1}
\usepackage{dsfont}               % Alternativa para conjuntos: \mathds
\usepackage{commath}              % Notações como \abs{}, \norm{}, \eval{}
\usepackage{pgfmath}             % Permite realizar cálculos matemáticos (somas, produtos, números aleatórios, etc.)
\usepackage{siunitx}              % sistema de unidades
\sisetup{group-separator = {\,}} % espaço fino

% --------------------------------------------------------
% FONTE MODERNA (TEXTO E MATEMÁTICA)
% --------------------------------------------------------
%\usepackage{newtxtext}            % Fonte do texto (estilo Times)
%\usepackage{newtxmath}            % Fonte matemática compatível com pacotes AMS
\usepackage[bitstream-charter]{mathdesign} % Fonte elegante e matemática harmoniosa

% --------------------------------------------------------
% AMBIENTES TEÓRICOS PERSONALIZADOS
% --------------------------------------------------------
\theoremstyle{definition}
\newtheorem{definicao}{Definição}       % Definição numerada por seção
\newtheorem{exemplo}{Exemplo}           % Exemplo numerado por seção

%\theoremstyle{plain}
\newtheorem{teorema}{Teorema}           % Teorema numerado por seção
\newtheorem{lema}[teorema]{Lema}                 % Lema com numeração conjunta
\newtheorem{proposicao}[teorema]{Proposição}     % Proposição idem
\newtheorem{corolario}[teorema]{Corolário}       % Corolário idem
\newtheorem{exercicio}[teorema]{Exercício}

%\theoremstyle{remark}
\newtheorem{observacao}[teorema]{Observação}     % Observação com mesmo contador

% --------------------------------------------------------
% TABELAS E FIGURAS
% --------------------------------------------------------
\usepackage{graphicx}             % Inclusão de imagens
\usepackage{float}                % Controle de posicionamento (ex: [H])
\usepackage{caption}              % Personalização de legendas
\usepackage{subcaption}          % Subfiguras com \begin{subfigure}
\usepackage{booktabs}            % Tabelas com qualidade profissional
\usepackage{array}               % Mais opções em colunas de tabelas
\usepackage{multirow}            % Células que ocupam várias linhas
\usepackage{multicol}            % Disposição em colunas múltiplas
\usepackage{tabularx}            % Tabelas com colunas de largura ajustável
\usepackage{longtable}           % Tabelas que quebram página
\usepackage{arydshln}            % Linhas tracejadas horizontais e verticais
\usepackage{makecell}            % Células com múltiplas linhas (e quebra de linha)
\usepackage{diagbox}             % Cabeçalho diagonal em tabelas
\usepackage{pdflscape}           % Gira páginas (modo paisagem)
\newcolumntype{L}[1]{>{\raggedright\let\newline\\\arraybackslash\hspace{0pt}}m{#1}}
\newcolumntype{C}[1]{>{\centering\let\newline\\\arraybackslash\hspace{0pt}}m{#1}}
\newcolumntype{R}[1]{>{\raggedleft\let\newline\\\arraybackslash\hspace{0pt}}m{#1}}

% --------------------------------------------------------
% FORMATAÇÃO E ESTRUTURA
% --------------------------------------------------------
\usepackage{geometry}            % Configuração de margens
\geometry{a4paper, left=1cm, right=1cm, bottom=1.5cm, top=1.5cm, headsep=1cm, footskip=1cm}
\usepackage{titlesec}            % Controle sobre títulos de seções
\usepackage{fancyhdr}            % Cabeçalhos e rodapés personalizados
\pagestyle{fancy}
\fancyhf{}                % Limpa cabeçalho e rodapé
\fancyfoot[R]{\thepage}   % Numeração à direita no rodapé
\renewcommand{\headrulewidth}{0pt} % Remove linha no topo
\usepackage{enumitem}            % Listas com controle de espaçamento e símbolo
\usepackage{tikz}                % Desenho de gráficos vetoriais (diagramas, etc.)
\usetikzlibrary{matrix,arrows,patterns,snakes,decorations.pathreplacing,3d,arrows.meta,calc}
\usepackage{etoolbox}             % Permite lógica condicional e manipulação de comandos
\usepackage{background}           % Permite adicionar conteúdo fixo no fundo da página (ex: linhas, marcas d'água)
\SetBgContents{} % Remove conteúdo padrão ("DRAFT")
\definecolor{background-color}{RGB}{233 227 206}

% --------------------------------------------------------
% GRÁFICOS E PLOTAGENS
% --------------------------------------------------------
\usepackage{pgfplots}             % Criação de gráficos vetoriais em 2D e 3D
\pgfplotsset{compat=1.18}         % Compatibilidade com versão atual do pacote
\usepackage{currfile} % Permite saber qual é o caminho completo do arquivo atual,

% --------------------------------------------------------
% LINKS E CORES
% --------------------------------------------------------
\usepackage{xcolor}              % Definição de cores (texto, links, etc.)
\usepackage[hidelinks]{hyperref} % Links sem moldura colorida no PDF
\usepackage{url}                 % Quebra automática de URLs
% Dica: se quiser links coloridos, use:
% \usepackage[colorlinks=true, linkcolor=blue, citecolor=blue, urlcolor=blue]{hyperref}

% --------------------------------------------------------
% CAIXAS E DESTAQUES VISUAIS
% --------------------------------------------------------
\usepackage[most]{tcolorbox}      % Criação de caixas coloridas e personalizadas para destaques, avisos, exemplos, etc.

% --------------------------------------------------------
% CÓDIGOS E ALGORITMOS
% --------------------------------------------------------
\usepackage{listings}            % Inclusão de código-fonte com destaque
\usepackage{algorithm}           % Ambiente de algoritmo
\usepackage[noend]{algpseudocode} % Sintaxe tipo pseudocódigo


% --------------------------------------------------------
% COMANDOS
% --------------------------------------------------------

\newcommand{\bbN}{\mathbb{N}}
\newcommand{\bbZ}{\mathbb{Z}}
\newcommand{\bbQ}{\mathbb{Q}}
\newcommand{\bbR}{\mathbb{R}}

\newcommand{\assunto}{TRIGONOMETRIA - PARTE II}
\newcommand{\atividade}{lista} % prova ou lista
\newcommand{\turma}{CONTROLE AMBIENTAL 1.\textordmasculine\ ANO}
\newcommand{\numProva}{A}
\newcommand{\professor}{Igor Martins Silva}
\newcommand{\bimestre}{1}
\newcommand{\ano}{2026}
\newcommand{\mesTexto}{fevereiro}
\newcommand{\mesNum}{02}
\newcommand{\dia}{19}
\newcommand{\dataTexto}{\dia\ de \mesTexto\ de \ano}
\newcommand{\dataNun}{\dia/\mesNum/\ano}

\newcommand{\titulo}{%
  \ifdefstring{\atividade}{prova}{Avaliação de Matemática}{%
  \ifdefstring{\atividade}{lista}{Lista de exercícios de Matemática}{%
  \ifdefstring{\atividade}{as}{Avaliação Somativa}{%
  \ifdefstring{\atividade}{ad}{Avaliação Diagnóstica}{%
  Atividade de Matemática}}}}%
}

\newcommand{\emtipo}{%
  \ifdefstring{\tipo}{individual}{\tipo}{em \tipo}%
}

\newcommand{\subtitulobase}[3]{%
    % #1 = título da 1ª coluna
    % #2 = conteúdo da 1ª coluna
    % #3 = mostra número da prova? (vazio ou algo)
    
    \begin{center}
        CEFET-MG -- Campus Contagem \\[3pt]
        \titulo\ -- \bimestre.\textordmasculine\ Bimestre de \ano \\[3pt]
        \professor
    \end{center}
    
    \begin{center}
        \vspace{15pt}
        \begin{tabular}{|C{210pt}|C{70pt}|C{210pt}|}
            \hline
            \textbf{#1} &
            \textbf{DATA} &
            \textbf{TURMA} \\
            \hline
            #2 & \dataNun & \turma \\
            \hline
        \end{tabular}
    \end{center}
    
    #3
    
    \vspace{15pt}
}

\newcommand{\subtitulolis}{%
  \subtitulobase{ASSUNTO}{\assunto}{}%
}

\newcommand{\subtitulopr}{%
  \subtitulobase{ASSUNTO}{\assunto}{\boxed{\numProva}}%
}

\newcommand{\subtituloas}{%
  \subtitulobase{DISCIPLINA}{MATEMÁTICA}{}%
}

\newcommand{\subtituload}{%
  \subtitulobase{DISCIPLINA}{MATEMÁTICA}{}%
}

\newcommand{\valorquestao}{}

\newenvironment{exercicioBanco}[1][]{%
    \ifdefstring{\atividade}{prova}{%
        \begin{exercicio}[#1]%
    }{%
        \begin{exercicio}%
    }%
}{%
    \end{exercicio}
}

\ifdefstring{\atividade}{lista}{
    \pagecolor{background-color}
}{
    
}



\hypersetup{
    pdfauthor={\professor},
    pdftitle={\titulo -- \assunto},
}

\title{\titulo -- \assunto}
\author{\professor}
\date{\data}

\graphicspath{
    {../../TrigonometriaII/}
}

%************* INÍCIO DO DOCUMENTO *************
\begin{document}
    \subtitulolis
        
    \phantomsection
    \addcontentsline{toc}{section}{Exercício 1}
    \input{../TII_Exercicio1}
    \noindent\rule{\linewidth}{1pt}  % linha horizontal fina
    
    \phantomsection
    \addcontentsline{toc}{section}{Exercício 2}
    %%%%%%%%%%%%%%%%%%%%%%%%%%%%%%%%%
%
%       EXERCÍCIO
%
%%%%%%%%%%%%%%%%%%%%%%%%%%%%%%%%%

\ifdefstring{\atividade}{prova}{%
    \ifdefstring{\modo}{objetivo}{%
        \renewcommand{\valorquestao}{\ValorQObj\ ponto}
    }{%
        \renewcommand{\valorquestao}{\ValorQDisc\ pontos}
    }%
}%

\begin{exercicioBanco}[\valorquestao]
Seja \(k \in \bbN\) tal que \(0 \le k < 4\). Calcule a soma dos números da forma \(\cos\left(k\cdot\dfrac{\pi}{2}\right)\).

% Define as alternativas
\newcommand{\alternativas}{%
    \begin{center}
        \begin{tabularx}{\textwidth}{XXXXX}
            (a) \(-2\). &
            (b) \(2\). &
            (c) \(0\). &
            (d) \(-1\). &
            (e) \(1\).
        \end{tabularx}
    \end{center}
}

% Define a resposta correta
\newcommand{\resposta}{C}

% Lógica condicional para exibição
\ifdefstring{\atividade}{lista}{%
    \alternativas
    \vspace{0.5em}
    
    \noindent\textbf{Resposta:} letra \textbf{\resposta}.
}{%
    \ifdefstring{\modo}{objetivo}{%
        \alternativas
    }{%
        % modo = discursiva → não mostra alternativas
    }
}
\end{exercicioBanco}


    \noindent\rule{\linewidth}{1pt}  % linha horizontal fina
    
    \phantomsection
    \addcontentsline{toc}{section}{Exercício 3}
    %%%%%%%%%%%%%%%%%%%%%%%%%%%%%%%%%
%
%       EXERCÍCIO
%
%%%%%%%%%%%%%%%%%%%%%%%%%%%%%%%%%

\ifdefstring{\atividade}{prova}{%
    \ifdefstring{\modo}{objetivo}{%
        \renewcommand{\valorquestao}{\ValorQObj\ ponto}
    }{%
        \renewcommand{\valorquestao}{\ValorQDisc\ pontos}
    }%
}%

\begin{exercicioBanco}[\valorquestao]
Calcule
%
\[
    \cos\left(\dfrac{\pi}{3}\right) + \cos\left(\dfrac{\pi}{3} + \pi\right) + \cos\left(\dfrac{\pi}{3} + 2\pi\right) + \cdots + \cos\left(\dfrac{\pi}{3} + 100\pi\right)
\]
%

% Define as alternativas
\newcommand{\alternativas}{%
    \begin{center}
        \begin{tabularx}{\textwidth}{XXXXX}
            (a) \(\dfrac{1}{2}\). &
            (b) \(\dfrac{\sqrt{2}}{2}\). &
            (c) \(\dfrac{\sqrt{3}}{2}\). &
            (d) \(\sqrt{2}\). &
            (e) \(\sqrt{3}\).
        \end{tabularx}
    \end{center}
}

% Define a resposta correta
\newcommand{\resposta}{A}

% Lógica condicional para exibição
\ifdefstring{\atividade}{lista}{%
    \alternativas
    \vspace{0.5em}
    
    \noindent\textbf{Resposta:} letra \textbf{\resposta}.
}{%
    \ifdefstring{\modo}{objetivo}{%
        \alternativas
    }{%
        % modo = discursiva → não mostra alternativas
    }
}
\end{exercicioBanco}


    \noindent\rule{\linewidth}{1pt}  % linha horizontal fina
    
    \phantomsection
    \addcontentsline{toc}{section}{Exercício 4}
    %%%%%%%%%%%%%%%%%%%%%%%%%%%%%%%%%
%
%       EXERCÍCIO
%
%%%%%%%%%%%%%%%%%%%%%%%%%%%%%%%%%

\ifdefstring{\atividade}{prova}{%
    \ifdefstring{\modo}{objetivo}{%
        \renewcommand{\valorquestao}{\ValorQObj\ ponto}
    }{%
        \renewcommand{\valorquestao}{\ValorQDisc\ pontos}
    }%
}%

\begin{exercicioBanco}[\valorquestao]
Considerando cada afirmação a seguir, responda se ela é verdadeira (V) ou falsa (F).
%
\begin{enumerate}[(a)]
    \item O produto \(\tg(28^{\circ})\cdot\tg(230^{\circ})\cdot\tg(307^{\circ})\) é negativo.
    %
    \item Vale que \(\sen(135^{\circ}) = \dfrac{\sqrt{2}}{2}\).
    \item Para todo \(\alpha \in \bbR\), \(\cos(\alpha) = \cos(-\alpha)\).
\end{enumerate}
%

% Define as alternativas
\newcommand{\alternativas}{%
    \begin{center}
        \begin{tabularx}{\textwidth}{XXXXX}
            (a) \((V,F,F)\). &
            (b) \((F,V,V)\). &
            (c) \((V,F,V)\). &
            (d) \((F,V,F)\). &
            (e) \((V,V,V)\).
        \end{tabularx}
    \end{center}
}

% Define a resposta correta
\newcommand{\resposta}{E}

% Lógica condicional para exibição
\ifdefstring{\atividade}{lista}{%
    \alternativas
    \vspace{0.5em}
    
    \noindent\textbf{Resposta:} letra \textbf{\resposta}.
}{%
    \ifdefstring{\modo}{objetivo}{%
        \alternativas
    }{%
        % modo = discursiva → não mostra alternativas
    }
}
\end{exercicioBanco}


    \noindent\rule{\linewidth}{1pt}  % linha horizontal fina
    
    \phantomsection
    \addcontentsline{toc}{section}{Exercício 5}
    %%%%%%%%%%%%%%%%%%%%%%%%%%%%%%%%%
%
%       EXERCÍCIO
%
%%%%%%%%%%%%%%%%%%%%%%%%%%%%%%%%%

\ifdefstring{\atividade}{prova}{%
    \ifdefstring{\modo}{objetivo}{%
        \renewcommand{\valorquestao}{\ValorQObj\ ponto}
    }{%
        \renewcommand{\valorquestao}{\ValorQDisc\ pontos}
    }%
}%

\begin{exercicioBanco}[\valorquestao]
Dada a expressão \(\cos(\theta) = \frac{2p - 1}{5}\), assinale a alternativa que contém o conjunto de valores que \(p\) pode assumir.

% Define as alternativas
\newcommand{\alternativas}{%
    \begin{center}
        \begin{tabularx}{\textwidth}{XXXXX}
            (a) \(-1 \le p \le 1\). &
            (b) \(-1 \le p \le 2\). &
            (c) \(-2 \le p \le 3\). &
            (d) \(-2 \le p \le 1\). &
            (e) \(-3 \le p \le 2\).
        \end{tabularx}
    \end{center}
}

% Define a resposta correta
\newcommand{\resposta}{C}

% Lógica condicional para exibição
\ifdefstring{\atividade}{lista}{%
    \alternativas
    \vspace{0.5em}
    
    \noindent\textbf{Resposta:} letra \textbf{\resposta}.
}{%
    \ifdefstring{\modo}{objetivo}{%
        \alternativas
    }{%
        % modo = discursiva → não mostra alternativas
    }
}
\end{exercicioBanco}


    \noindent\rule{\linewidth}{1pt}  % linha horizontal fina
    
    \phantomsection
    \addcontentsline{toc}{section}{Exercício 6}
    %%%%%%%%%%%%%%%%%%%%%%%%%%%%%%%%%
%
%       EXERCÍCIO
%
%%%%%%%%%%%%%%%%%%%%%%%%%%%%%%%%%

\ifdefstring{\atividade}{prova}{%
    \ifdefstring{\modo}{objetivo}{%
        \renewcommand{\valorquestao}{\ValorQObj\ ponto}
    }{%
        \renewcommand{\valorquestao}{\ValorQDisc\ pontos}
    }%
}%

\begin{exercicioBanco}[\valorquestao]
Seja \(\sen(\alpha) = \frac{3}{5}\) e \(\alpha\) um arco no segundo quadrante. Encontre o valor de \(\tg(\alpha)\).

% Define as alternativas
\newcommand{\alternativas}{%
    \begin{center}
        \begin{tabularx}{\textwidth}{XXXXX}
            (a) \(\dfrac{4}{3}\). &
            (b) \(\dfrac{3}{4}\). &
            (c) \(-\dfrac{3}{4}\). &
            (d) \(-1\). &
            (e) \(-\dfrac{4}{3}\).
        \end{tabularx}
    \end{center}
}

% Define a resposta correta
\newcommand{\resposta}{C}

% Lógica condicional para exibição
\ifdefstring{\atividade}{lista}{%
    \alternativas
    \vspace{0.5em}
    
    \noindent\textbf{Resposta:} letra \textbf{\resposta}.
}{%
    \ifdefstring{\modo}{objetivo}{%
        \alternativas
    }{%
        % modo = discursiva → não mostra alternativas
    }
}
\end{exercicioBanco}


    \noindent\rule{\linewidth}{1pt}  % linha horizontal fina
    
    \phantomsection
    \addcontentsline{toc}{section}{Exercício 7}
    %%%%%%%%%%%%%%%%%%%%%%%%%%%%%%%%%
%
%       EXERCÍCIO
%
%%%%%%%%%%%%%%%%%%%%%%%%%%%%%%%%%

\ifdefstring{\atividade}{prova}{%
    \ifdefstring{\modo}{objetivo}{%
        \renewcommand{\valorquestao}{\ValorQObj\ ponto}
    }{%
        \renewcommand{\valorquestao}{\ValorQDisc\ pontos}
    }%
}%

\begin{exercicioBanco}[\valorquestao]
Considere as afirmações a seguir:
%
\begin{enumerate}[I.]
    \item \(\sen^{2}(144^{\circ}) + \cos^{2}(144^{\circ}) = 1\).
    %
    \item Para todo \(x \in \bbR\), \(\tg(x) > \sen(x)\).
    %
    \item Para todo \(x \in \bbR\), \(0 \le \cos(x) \le 1\).
\end{enumerate}
%
Qual(quais) está(estão) correta(s)?

% Define as alternativas
\newcommand{\alternativas}{%
    \begin{center}
        \begin{tabularx}{\textwidth}{XXXXX}
            (a) Apenas I. &
            (b) Apenas II. &
            (c) Apenas III. &
            (d) Apenas I e III. &
            (e) I, II e III.
        \end{tabularx}
    \end{center}
}

% Define a resposta correta
\newcommand{\resposta}{A}

% Lógica condicional para exibição
\ifdefstring{\atividade}{lista}{%
    \alternativas
    \vspace{0.5em}
    
    \noindent\textbf{Resposta:} letra \textbf{\resposta}.
}{%
    \ifdefstring{\modo}{objetivo}{%
        \alternativas
    }{%
        % modo = discursiva → não mostra alternativas
    }
}
\end{exercicioBanco}


    \noindent\rule{\linewidth}{1pt}  % linha horizontal fina
    
    \phantomsection
    \addcontentsline{toc}{section}{Exercício 8}
    %%%%%%%%%%%%%%%%%%%%%%%%%%%%%%%%%
%
%       EXERCÍCIO
%
%%%%%%%%%%%%%%%%%%%%%%%%%%%%%%%%%

\ifdefstring{\atividade}{prova}{%
    \ifdefstring{\modo}{objetivo}{%
        \renewcommand{\valorquestao}{\ValorQObj\ ponto}
    }{%
        \renewcommand{\valorquestao}{\ValorQDisc\ pontos}
    }%
}%

\begin{exercicioBanco}[\valorquestao]
Considere o triângulo retângulo \(ABD\) exibido na figura abaixo, em que \(AB = 2\) cm, \(BC = 1\) cm e \(CD = 5 cm\). Então o ângulo \(\theta\) é igual a quanto?

\medskip

\begin{figure}[H]
    \centering
    \includegraphics[width=0.3\textwidth]{../TII_Exercicio7_Fig.png}
\end{figure}

\smallskip

% Define as alternativas
\newcommand{\alternativas}{%
    \begin{center}
        \begin{tabularx}{\textwidth}{XXXXX}
            (a) \(15^{\circ}\). &
            (b) \(30^{\circ}\). &
            (c) \(45^{\circ}\). &
            (d) \(60^{\circ}\). &
            (e) \(75^{\circ}\).
        \end{tabularx}
    \end{center}
}

% Define a resposta correta
\newcommand{\resposta}{C}

% Lógica condicional para exibição
\ifdefstring{\atividade}{lista}{%
    \alternativas
    \vspace{0.5em}
    
    \noindent\textbf{Resposta:} letra \textbf{\resposta}.
}{%
    \ifdefstring{\modo}{objetivo}{%
        \alternativas
    }{%
        % modo = discursiva → não mostra alternativas
    }
}
\end{exercicioBanco}


    \noindent\rule{\linewidth}{1pt}  % linha horizontal fina
    
    \phantomsection
    \addcontentsline{toc}{section}{Exercício 9}
    %%%%%%%%%%%%%%%%%%%%%%%%%%%%%%%%%
%
%       EXERCÍCIO
%
%%%%%%%%%%%%%%%%%%%%%%%%%%%%%%%%%

\ifdefstring{\atividade}{prova}{%
    \ifdefstring{\modo}{objetivo}{%
        \renewcommand{\valorquestao}{\ValorQObj\ ponto}
    }{%
        \renewcommand{\valorquestao}{\ValorQDisc\ pontos}
    }%
}%

\begin{exercicioBanco}[\valorquestao]
A figura a seguir mostra o corte lateral de um terreno onde será construída uma rampa reta \(\overline{AC}\), que servirá para o acesso de veículos à casa, que se encontra na parte mais alta do terreno. A distância de \(A\) a \(B\) é de 6 m, de \(B\) a \(C\) é de 10 m e o menor ângulo formado entre \(\overline{AB}\) e \(\overline{BC}\) é de \(120^{\circ}\).

\medskip

\begin{figure}[H]
    \centering
    \includegraphics[width=0.3\textwidth]{TII_Exercicio8_Fig.png}
\end{figure}

\smallskip

Determine o valor do comprimento da rampa.

% Define as alternativas
\newcommand{\alternativas}{%
    \begin{center}
        \begin{tabularx}{\textwidth}{XXXXX}
            (a) \(12\) m. &
            (b) \(12,5\) m. &
            (c) \(13\) m. &
            (d) \(13,5\) m. &
            (e) \(14\) m.
        \end{tabularx}
    \end{center}
}

% Define a resposta correta
\newcommand{\resposta}{E}

% Lógica condicional para exibição
\ifdefstring{\atividade}{lista}{%
    \alternativas
    \vspace{0.5em}
    
    \noindent\textbf{Resposta:} letra \textbf{\resposta}.
}{%
    \ifdefstring{\modo}{objetivo}{%
        \alternativas
    }{%
        % modo = discursiva → não mostra alternativas
    }
}
\end{exercicioBanco}


    \noindent\rule{\linewidth}{1pt}  % linha horizontal fina
    
    \phantomsection
    \addcontentsline{toc}{section}{Exercício 10}
    %%%%%%%%%%%%%%%%%%%%%%%%%%%%%%%%%
%
%       EXERCÍCIO
%
%%%%%%%%%%%%%%%%%%%%%%%%%%%%%%%%%

\ifdefstring{\atividade}{prova}{%
    \ifdefstring{\modo}{objetivo}{%
        \renewcommand{\valorquestao}{\ValorQObj\ ponto}
    }{%
        \renewcommand{\valorquestao}{\ValorQDisc\ pontos}
    }%
}%

\begin{exercicioBanco}[\valorquestao]
Observe o visor de um relógio de ponteiros que marca 2 horas. Sabendo que os ponteiros menor (das horas) e maior (dos minutos) medem, respectivamente, 50 cm e 80 cm, calcule a distância entre suas extremidades nesse horário.

\medskip

\begin{figure}[H]
    \centering
    \includegraphics[width=0.15\textwidth]{../TII_Exercicio9_Fig.png}
\end{figure}

\smallskip

% Define as alternativas
\newcommand{\alternativas}{%
    \begin{center}
        \begin{tabularx}{\textwidth}{XXXXX}
            (a) \(70\) cm. &
            (b) \(76,5\) cm. &
            (c) \(81,3\) cm. &
            (d) \(99,9\) cm. &
            (e) \(100\) cm.
        \end{tabularx}
    \end{center}
}

% Define a resposta correta
\newcommand{\resposta}{A}

% Lógica condicional para exibição
\ifdefstring{\atividade}{lista}{%
    \alternativas
    \vspace{0.5em}
    
    \noindent\textbf{Resposta:} letra \textbf{\resposta}.
}{%
    \ifdefstring{\modo}{objetivo}{%
        \alternativas
    }{%
        % modo = discursiva → não mostra alternativas
    }
}
\end{exercicioBanco}


    \noindent\rule{\linewidth}{1pt}  % linha horizontal fina
    
    \phantomsection
    \addcontentsline{toc}{section}{Exercício 11}
    %%%%%%%%%%%%%%%%%%%%%%%%%%%%%%%%%
%
%       EXERCÍCIO
%
%%%%%%%%%%%%%%%%%%%%%%%%%%%%%%%%%

\ifdefstring{\atividade}{prova}{%
    \ifdefstring{\modo}{objetivo}{%
        \renewcommand{\valorquestao}{\ValorQObj\ ponto}
    }{%
        \renewcommand{\valorquestao}{\ValorQDisc\ pontos}
    }%
}%

\begin{exercicioBanco}[\valorquestao]
Na figura abaixo, além das medidas dos ângulos indicados, sabe-se que \(B\) é ponto médio de \(\overline{AC}\) e \(AC = 2\) cm.

\medskip

\begin{figure}[H]
    \centering
    \includegraphics[width=0.2\textwidth]{TII_Exercicio10_Fig.png}
\end{figure}

\smallskip

A medida de \(\overline{DE}\), em centímetros, é igual a:

% Define as alternativas
\newcommand{\alternativas}{%
    \begin{center}
        \begin{tabularx}{\textwidth}{XXXXX}
            (a) \(\dfrac{1}{2}\). &
            (b) \(1\). &
            (c) \(\sqrt{2}\). &
            (d) \(1,5\). &
            (e) \(\sqrt{3}\).
        \end{tabularx}
    \end{center}
}

% Define a resposta correta
\newcommand{\resposta}{E}

% Lógica condicional para exibição
\ifdefstring{\atividade}{lista}{%
    \alternativas
    \vspace{0.5em}
    
    \noindent\textbf{Resposta:} letra \textbf{\resposta}.
}{%
    \ifdefstring{\modo}{objetivo}{%
        \alternativas
    }{%
        % modo = discursiva → não mostra alternativas
    }
}
\end{exercicioBanco}


    \noindent\rule{\linewidth}{1pt}  % linha horizontal fina
    
    \phantomsection
    \addcontentsline{toc}{section}{Exercício 12}
    \input{../TII_Exercicio11_1}
    \noindent\rule{\linewidth}{1pt}  % linha horizontal fina
    
    \phantomsection
    \addcontentsline{toc}{section}{Exercício 13}
    %%%%%%%%%%%%%%%%%%%%%%%%%%%%%%%%%
%
%       EXERCÍCIO
%
%%%%%%%%%%%%%%%%%%%%%%%%%%%%%%%%%

\ifdefstring{\atividade}{prova}{%
    \ifdefstring{\modo}{objetivo}{%
        \renewcommand{\valorquestao}{\ValorQObj\ ponto}
    }{%
        \renewcommand{\valorquestao}{\ValorQDisc\ pontos}
    }%
}%

\begin{exercicioBanco}[\valorquestao]
Considere a função \(f(x) = 3 - 5\sen(2x + 4)\). Os valores de máximo, mínimo e o período de \(f(x)\) são, respectivamente,

% Define as alternativas
\newcommand{\alternativas}{%
    \begin{center}
        \begin{tabularx}{\textwidth}{XXXXX}
            (a) \(-2,-8,\pi\). &
            (b) \(8,-2,\pi\). &
            (c) \(\pi,-2,8\). &
            (d) \(\pi,-8,-2\). &
            (e) \(8,\pi,-2\).
        \end{tabularx}
    \end{center}
}

% Define a resposta correta
\newcommand{\resposta}{B}

% Lógica condicional para exibição
\ifdefstring{\atividade}{lista}{%
    \alternativas
    \vspace{0.5em}
    
    \noindent\textbf{Resposta:} letra \textbf{\resposta}.
}{%
    \ifdefstring{\modo}{objetivo}{%
        \alternativas
    }{%
        % modo = discursiva → não mostra alternativas
    }
}
\end{exercicioBanco}


    \noindent\rule{\linewidth}{1pt}  % linha horizontal fina
    
    \phantomsection
    \addcontentsline{toc}{section}{Exercício 14}
    %%%%%%%%%%%%%%%%%%%%%%%%%%%%%%%%%
%
%       EXERCÍCIO
%
%%%%%%%%%%%%%%%%%%%%%%%%%%%%%%%%%

\ifdefstring{\atividade}{prova}{%
    \ifdefstring{\modo}{objetivo}{%
        \renewcommand{\valorquestao}{\ValorQObj\ ponto}
    }{%
        \renewcommand{\valorquestao}{\ValorQDisc\ pontos}
    }%
}%

\begin{exercicioBanco}[\valorquestao]
Admitindo-se que o peso de determinada pessoa, ao longo de um ano, possa ser modelado pela função
%
\[
    P(t) = 70 - 4\sen\left(\dfrac{(t+3)\pi}{6}\right),
\]
%
em que \(t = 1, \dots, 12\) corresponde aos meses de janeiro a dezembro, determine o peso dessa pessoa em agosto.

% Define as alternativas
\newcommand{\alternativas}{%
    \begin{center}
        \begin{tabularx}{\textwidth}{XXXXX}
            (a) \(70\). &
            (b) \(74\). &
            (c) \(72\). &
            (d) \(70 + 2\sqrt{3}\). &
            (e) \(70 - 2\sqrt{3}\).
        \end{tabularx}
    \end{center}
}

% Define a resposta correta
\newcommand{\resposta}{C}

% Lógica condicional para exibição
\ifdefstring{\atividade}{lista}{%
    \alternativas
    \vspace{0.5em}
    
    \noindent\textbf{Resposta:} letra \textbf{\resposta}.
}{%
    \ifdefstring{\modo}{objetivo}{%
        \alternativas
    }{%
        % modo = discursiva → não mostra alternativas
    }
}
\end{exercicioBanco}


    \noindent\rule{\linewidth}{1pt}  % linha horizontal fina
    
    \phantomsection
    \addcontentsline{toc}{section}{Exercício 15}
    %%%%%%%%%%%%%%%%%%%%%%%%%%%%%%%%%
%
%       EXERCÍCIO
%
%%%%%%%%%%%%%%%%%%%%%%%%%%%%%%%%%

\ifdefstring{\atividade}{prova}{%
    \ifdefstring{\modo}{objetivo}{%
        \renewcommand{\valorquestao}{\ValorQObj\ ponto}
    }{%
        \renewcommand{\valorquestao}{\ValorQDisc\ pontos}
    }%
}%

\begin{exercicioBanco}[\valorquestao]
Considere que o volume de ar nos pulmões de um ser humano adulto, em litro, é descritos pela função
%
\[
    V(t) = a + b\sen(ct),
\]
%
onde \(t\) é o tempo em segundos. Sabendo-se que o volume máximo de ar é 4 litros, o mínimo é 2 litros e que o período de \(V(t)\) é igual a 6. Admitindo que \(b, c > 0\), determine \(a\), \(b\) e \(c\).

% Define as alternativas
\newcommand{\alternativas}{%
    \begin{center}
        \begin{tabularx}{\textwidth}{XXX}
            (a) \(a = 2\), \(b = 2\) e \(c = 6\). &
            (b) \(a = 1\), \(b = 3\) e \(c = \frac{\pi}{6}\). &
            (c) \(a = 3\), \(b = 1\) e \(c = \frac{\pi}{3}\). \\[5pt]
            (d) \(a = 3\), \(b = 1\) e \(c = 6\). &
            (e) \(a = 2\), \(b = 3\) e \(c = \pi\).
        \end{tabularx}
    \end{center}
}

% Define a resposta correta
\newcommand{\resposta}{C}

% Lógica condicional para exibição
\ifdefstring{\atividade}{lista}{%
    \alternativas
    \vspace{0.5em}
    
    \noindent\textbf{Resposta:} letra \textbf{\resposta}.
}{%
    \ifdefstring{\modo}{objetivo}{%
        \alternativas
    }{%
        % modo = discursiva → não mostra alternativas
    }
}
\end{exercicioBanco}


    \noindent\rule{\linewidth}{1pt}  % linha horizontal fina
    
    \phantomsection
    \addcontentsline{toc}{section}{Exercício 16}
    %%%%%%%%%%%%%%%%%%%%%%%%%%%%%%%%%
%
%       EXERCÍCIO
%
%%%%%%%%%%%%%%%%%%%%%%%%%%%%%%%%%

\ifdefstring{\atividade}{prova}{%
    \ifdefstring{\modo}{objetivo}{%
        \renewcommand{\valorquestao}{\ValorQObj\ ponto}
    }{%
        \renewcommand{\valorquestao}{\ValorQDisc\ pontos}
    }%
}%

\begin{exercicioBanco}[\valorquestao]
Sejam as funções \(f(x) = 2\sen(x)\) e \(g(x) = \sen(2x)\). A respeito delas, assinale a alternativa correta.
%
% Define as alternativas
\newcommand{\alternativas}{%
    \begin{center}
        \begin{tabularx}{\textwidth}{X}
            (a) O período de \(f(x)\) é o dobro do período de \(g(x)\). \\[5pt]
            (b) As funções \(f(x)\) e \(g(x)\) possuem os mesmos zeros. \\[5pt]
            (c) O máximo de \(f(x)\) é igual ao máximo de \(g(x)\). \\[5pt]
            (d) O máximo de \(g(x)\) é o dobro do máximo de \(f(x)\). \\[5pt]
            (e) O período de \(g(x)\) é o dobro do período de \(f(x)\).
        \end{tabularx}
    \end{center}
}

% Define a resposta correta
\newcommand{\resposta}{A}

% Lógica condicional para exibição
\ifdefstring{\atividade}{lista}{%
    \alternativas
    \vspace{0.5em}
    
    \noindent\textbf{Resposta:} letra \textbf{\resposta}.
}{%
    \ifdefstring{\modo}{objetivo}{%
        \alternativas
    }{%
        % modo = discursiva → não mostra alternativas
    }
}
\end{exercicioBanco}


    \noindent\rule{\linewidth}{1pt}  % linha horizontal fina
    
    \phantomsection
    \addcontentsline{toc}{section}{Exercício 17}
    %%%%%%%%%%%%%%%%%%%%%%%%%%%%%%%%%
%
%       EXERCÍCIO
%
%%%%%%%%%%%%%%%%%%%%%%%%%%%%%%%%%

\ifdefstring{\atividade}{prova}{%
    \ifdefstring{\modo}{objetivo}{%
        \renewcommand{\valorquestao}{\ValorQObj\ ponto}
    }{%
        \renewcommand{\valorquestao}{\ValorQDisc\ pontos}
    }%
}%

\begin{exercicioBanco}[\valorquestao]
A tensão em um circuito é dada, em volts, pela função
%
\[
    T(t) = 120\sen\left(100\pi t + \dfrac{\pi}{6}\right),
\]
%
onde \(t\) é o tempo em segundos. Determine a tensão no instante \(t = 0,01\) s.

% Define as alternativas
\newcommand{\alternativas}{%
    \begin{center}
        \begin{tabularx}{\textwidth}{XXXXX}
            (a) \(60\) V. &
            (b) \(-60\) V. &
            (c) \(60\sqrt{3}\) V. &
            (d) \(-60\sqrt{3}\) V. &
            (e) \(60\sqrt{2}\) V.
        \end{tabularx}
    \end{center}
}

% Define a resposta correta
\newcommand{\resposta}{B}

% Lógica condicional para exibição
\ifdefstring{\atividade}{lista}{%
    \alternativas
    \vspace{0.5em}
    
    \noindent\textbf{Resposta:} letra \textbf{\resposta}.
}{%
    \ifdefstring{\modo}{objetivo}{%
        \alternativas
    }{%
        % modo = discursiva → não mostra alternativas
    }
}
\end{exercicioBanco}


    \noindent\rule{\linewidth}{1pt}  % linha horizontal fina
    
    \phantomsection
    \addcontentsline{toc}{section}{Exercício 18}
    \input{../TII_Exercicio16}
    \noindent\rule{\linewidth}{1pt}  % linha horizontal fina
    
    \phantomsection
    \addcontentsline{toc}{section}{Exercício 19}
    %%%%%%%%%%%%%%%%%%%%%%%%%%%%%%%%%
%
%       EXERCÍCIO
%
%%%%%%%%%%%%%%%%%%%%%%%%%%%%%%%%%

\ifdefstring{\atividade}{prova}{%
    \ifdefstring{\modo}{objetivo}{%
        \renewcommand{\valorquestao}{\ValorQObj\ ponto}
    }{%
        \renewcommand{\valorquestao}{\ValorQDisc\ pontos}
    }%
}%

\begin{exercicioBanco}[\valorquestao]
Um cientista, em seus estudos para modelar a pressão arterial de uma pessoa, utiliza uma função do tipo
\[
P(t) = A + B\cos(kt),
\]
em que $A$, $B$ e $k$ são constantes reais positivas e $t$ representa a variável tempo, medida em segundos.
Considere que um batimento cardíaco representa o intervalo de tempo entre duas sucessivas pressões máximas.

Ao analisar um caso específico, o cientista obteve os seguintes dados:
\begin{itemize}
    \item pressão mínima: $78$;
    \item pressão máxima: $120$;
    \item número de batimentos cardíacos por minuto: $90$.
\end{itemize}

Determine a função $P(t)$ obtida por este cientista, ao analisar o caso específico.

% Define as alternativas
\newcommand{\alternativas}{%
    \begin{center}
        \begin{tabularx}{\textwidth}{XXX}
            (a) \(P(t) = 99 + 21\cdot\cos(3\pi t)\). &
            (b) \(P(t) = 78 + 42\cdot\cos(3\pi t)\). &
            (c) \(P(t) = 99 + 21\cdot\cos(2\pi t)\). \\[5pt]
            (d) \(P(t) = 99 + 21\cdot\cos(t)\). &
            (e) \(P(t) = 78 + 42\cdot\cos(t)\).
        \end{tabularx}
    \end{center}
}

% Define a resposta correta
\newcommand{\resposta}{A}

% Lógica condicional para exibição
\ifdefstring{\atividade}{lista}{%
    \alternativas
    \vspace{0.5em}
    
    \noindent\textbf{Resposta:} letra \textbf{\resposta}.
}{%
    \ifdefstring{\modo}{objetivo}{%
        \alternativas
    }{%
        % modo = discursiva → não mostra alternativas
    }
}
\end{exercicioBanco}


    \noindent\rule{\linewidth}{1pt}  % linha horizontal fina
    
    \phantomsection
    \addcontentsline{toc}{section}{Exercício 20}
    \input{../TII_Exercicio18}
    \noindent\rule{\linewidth}{1pt}  % linha horizontal fina
    
    \phantomsection
    \addcontentsline{toc}{section}{Exercício 21}
    %%%%%%%%%%%%%%%%%%%%%%%%%%%%%%%%%
%
%       EXERCÍCIO
%
%%%%%%%%%%%%%%%%%%%%%%%%%%%%%%%%%

\ifdefstring{\atividade}{prova}{%
    \ifdefstring{\modo}{objetivo}{%
        \renewcommand{\valorquestao}{\ValorQObj\ ponto}
    }{%
        \renewcommand{\valorquestao}{\ValorQDisc\ pontos}
    }%
}%

\begin{exercicioBanco}[\valorquestao]
Em uma determinada região litorânea, a maré oscila segundo a função \(h(t) = 3 - 2\sen\left(\dfrac{\pi t}{12}\right)\), sendo h a altura em metros, que a maré atinge no tempo t em horas, medido a partir de 6 h da manhã. Uma embarcação, que se encontra encalhada às 11 h da manhã, precisa de uma profundidade mínima de 2 metros para navegar. Quantas horas os tripulantes dessa embarcação ainda terão que esperar para prosseguirem viagem?

% Define as alternativas
\newcommand{\alternativas}{%
    \begin{center}
        \begin{tabularx}{\textwidth}{XXXXX}
            (a) 4 h. &
            (b) 5 h. &
            (c) 6 h. &
            (d) 7 h. &
            (e) 8 h.
        \end{tabularx}
    \end{center}
}

% Define a resposta correta
\newcommand{\resposta}{B}

% Lógica condicional para exibição
\ifdefstring{\atividade}{lista}{%
    \alternativas
    \vspace{0.5em}
    
    \noindent\textbf{Resposta:} letra \textbf{\resposta}.
}{%
    \ifdefstring{\modo}{objetivo}{%
        \alternativas
    }{%
        % modo = discursiva → não mostra alternativas
    }
}
\end{exercicioBanco}


    \noindent\rule{\linewidth}{1pt}  % linha horizontal fina
    
\end{document}
%*************** FINAL DO DOCUMENTO ***************
