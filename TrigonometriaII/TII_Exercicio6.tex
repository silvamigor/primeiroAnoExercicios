%%%%%%%%%%%%%%%%%%%%%%%%%%%%%%%%%
%
%       EXERCÍCIO
%
%%%%%%%%%%%%%%%%%%%%%%%%%%%%%%%%%

\ifdefstring{\atividade}{prova}{%
    \ifdefstring{\modo}{objetivo}{%
        \renewcommand{\valorquestao}{\ValorQObj\ ponto}
    }{%
        \renewcommand{\valorquestao}{\ValorQDisc\ pontos}
    }%
}%

\begin{exercicioBanco}[\valorquestao]
Considere as afirmações a seguir:
%
\begin{enumerate}[I.]
    \item \(\sen^{2}(144^{\circ}) + \cos^{2}(144^{\circ}) = 1\).
    %
    \item Para todo \(x \in \bbR\), \(\tg(x) > \sen(x)\).
    %
    \item Para todo \(x \in \bbR\), \(0 \le \cos(x) \le 1\).
\end{enumerate}
%
Qual(quais) está(estão) correta(s)?

% Define as alternativas
\newcommand{\alternativas}{%
    \begin{center}
        \begin{tabularx}{\textwidth}{XXXXX}
            (a) Apenas I. &
            (b) Apenas II. &
            (c) Apenas III. &
            (d) Apenas I e III. &
            (e) I, II e III.
        \end{tabularx}
    \end{center}
}

% Define a resposta correta
\newcommand{\resposta}{A}

% Lógica condicional para exibição
\ifdefstring{\atividade}{lista}{%
    \alternativas
    \vspace{0.5em}
    
    \noindent\textbf{Resposta:} letra \textbf{\resposta}.
}{%
    \ifdefstring{\modo}{objetivo}{%
        \alternativas
    }{%
        % modo = discursiva → não mostra alternativas
    }
}
\end{exercicioBanco}

