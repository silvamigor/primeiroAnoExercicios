%%%%%%%%%%%%%%%%%%%%%%%%%%%%%%%%%
%
%       EXERCÍCIO
%
%%%%%%%%%%%%%%%%%%%%%%%%%%%%%%%%%

\ifdefstring{\atividade}{prova}{%
    \ifdefstring{\modo}{objetivo}{%
        \renewcommand{\valorquestao}{\ValorQObj\ ponto}
    }{%
        \renewcommand{\valorquestao}{\ValorQDisc\ pontos}
    }%
}%

\begin{exercicioBanco}[\valorquestao]
Na figura abaixo, além das medidas dos ângulos indicados, sabe-se que \(B\) é ponto médio de \(\overline{AC}\) e \(AC = 2\) cm.

\medskip

\begin{figure}[H]
    \centering
    \includegraphics[width=0.2\textwidth]{TII_Exercicio10_Fig.png}
\end{figure}

\smallskip

A medida de \(\overline{DE}\), em centímetros, é igual a:

% Define as alternativas
\newcommand{\alternativas}{%
    \begin{center}
        \begin{tabularx}{\textwidth}{XXXXX}
            (a) \(\dfrac{1}{2}\). &
            (b) \(1\). &
            (c) \(\sqrt{2}\). &
            (d) \(1,5\). &
            (e) \(\sqrt{3}\).
        \end{tabularx}
    \end{center}
}

% Define a resposta correta
\newcommand{\resposta}{E}

% Lógica condicional para exibição
\ifdefstring{\atividade}{lista}{%
    \alternativas
    \vspace{0.5em}
    
    \noindent\textbf{Resposta:} letra \textbf{\resposta}.
}{%
    \ifdefstring{\modo}{objetivo}{%
        \alternativas
    }{%
        % modo = discursiva → não mostra alternativas
    }
}
\end{exercicioBanco}

