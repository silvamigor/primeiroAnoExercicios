%%%%%%%%%%%%%%%%%%%%%%%%%%%%%%%%%
%
%       EXERCÍCIO
%
%%%%%%%%%%%%%%%%%%%%%%%%%%%%%%%%%

\ifdefstring{\atividade}{prova}{%
    \ifdefstring{\modo}{objetivo}{%
        \renewcommand{\valorquestao}{\ValorQObj\ ponto}
    }{%
        \renewcommand{\valorquestao}{\ValorQDisc\ pontos}
    }%
}%

\begin{exercicioBanco}[\valorquestao]
A figura a seguir mostra o corte lateral de um terreno onde será construída uma rampa reta \(\overline{AC}\), que servirá para o acesso de veículos à casa, que se encontra na parte mais alta do terreno. A distância de \(A\) a \(B\) é de 6 m, de \(B\) a \(C\) é de 10 m e o menor ângulo formado entre \(\overline{AB}\) e \(\overline{BC}\) é de \(120^{\circ}\).

\medskip

\begin{figure}[H]
    \centering
    \includegraphics[width=0.3\textwidth]{TII_Exercicio8_Fig.png}
\end{figure}

\smallskip

Determine o valor do comprimento da rampa.

% Define as alternativas
\newcommand{\alternativas}{%
    \begin{center}
        \begin{tabularx}{\textwidth}{XXXXX}
            (a) \(12\) m. &
            (b) \(12,5\) m. &
            (c) \(13\) m. &
            (d) \(13,5\) m. &
            (e) \(14\) m.
        \end{tabularx}
    \end{center}
}

% Define a resposta correta
\newcommand{\resposta}{E}

% Lógica condicional para exibição
\ifdefstring{\atividade}{lista}{%
    \alternativas
    \vspace{0.5em}
    
    \noindent\textbf{Resposta:} letra \textbf{\resposta}.
}{%
    \ifdefstring{\modo}{objetivo}{%
        \alternativas
    }{%
        % modo = discursiva → não mostra alternativas
    }
}
\end{exercicioBanco}

