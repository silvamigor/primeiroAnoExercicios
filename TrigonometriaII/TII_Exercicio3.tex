%%%%%%%%%%%%%%%%%%%%%%%%%%%%%%%%%
%
%       EXERCÍCIO
%
%%%%%%%%%%%%%%%%%%%%%%%%%%%%%%%%%

\ifdefstring{\atividade}{prova}{%
    \ifdefstring{\modo}{objetivo}{%
        \renewcommand{\valorquestao}{\ValorQObj\ ponto}
    }{%
        \renewcommand{\valorquestao}{\ValorQDisc\ pontos}
    }%
}%

\begin{exercicioBanco}[\valorquestao]
Considerando cada afirmação a seguir, responda se ela é verdadeira (V) ou falsa (F).
%
\begin{enumerate}[(a)]
    \item O produto \(\tg(28^{\circ})\cdot\tg(230^{\circ})\cdot\tg(307^{\circ})\) é negativo.
    %
    \item Vale que \(\sen(135^{\circ}) = \dfrac{\sqrt{2}}{2}\).
    \item Para todo \(\alpha \in \bbR\), \(\cos(\alpha) = \cos(-\alpha)\).
\end{enumerate}
%

% Define as alternativas
\newcommand{\alternativas}{%
    \begin{center}
        \begin{tabularx}{\textwidth}{XXXXX}
            (a) \((V,F,F)\). &
            (b) \((F,V,V)\). &
            (c) \((V,F,V)\). &
            (d) \((F,V,F)\). &
            (e) \((V,V,V)\).
        \end{tabularx}
    \end{center}
}

% Define a resposta correta
\newcommand{\resposta}{E}

% Lógica condicional para exibição
\ifdefstring{\atividade}{lista}{%
    \alternativas
    \vspace{0.5em}
    
    \noindent\textbf{Resposta:} letra \textbf{\resposta}.
}{%
    \ifdefstring{\modo}{objetivo}{%
        \alternativas
    }{%
        % modo = discursiva → não mostra alternativas
    }
}
\end{exercicioBanco}

