%%%%%%%%%%%%%%%%%%%%%%%%%%%%%%%%%
%
%       EXERCÍCIO
%
%%%%%%%%%%%%%%%%%%%%%%%%%%%%%%%%%

\ifdefstring{\atividade}{prova}{%
    \ifdefstring{\modo}{objetivo}{%
        \renewcommand{\valorquestao}{\ValorQObj\ ponto}
    }{%
        \renewcommand{\valorquestao}{\ValorQDisc\ pontos}
    }%
}%

\begin{exercicioBanco}[\valorquestao]
Existem 5 bolas dispostas em uma mesa de bilhar. A reta formada entre as bolas 1 e 2 é paralela à reta formada entre as bolas 4 e 5.

\medskip

\begin{figure}[H]
    \centering
    \includegraphics[width=0.5\textwidth]{TI_Exercicio18_Fig.png}
\end{figure}

\smallskip

Conforme as medidas dispostas na imagem responda: qual a distância entre as bolas 1 e 3?

% Define as alternativas
\newcommand{\alternativas}{%
    \begin{center}
        \begin{tabularx}{\textwidth}{XXXXX}
            (a) \(20\). &
            (b) \(30\). &
            (c) \(40\). &
            (d) \(50\). &
            (e) \(60\).
        \end{tabularx}
    \end{center}
}

% Define a resposta correta
\newcommand{\resposta}{C}

% Lógica condicional para exibição
\ifdefstring{\atividade}{lista}{%
    \alternativas
    \vspace{0.5em}
    
    \noindent\textbf{Resposta:} letra \textbf{\resposta}.
}{%
    \ifdefstring{\modo}{objetivo}{%
        \alternativas
    }{%
        % modo = discursiva → não mostra alternativas
    }
}
\end{exercicioBanco}

