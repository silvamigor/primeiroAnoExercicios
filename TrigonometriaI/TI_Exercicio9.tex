%%%%%%%%%%%%%%%%%%%%%%%%%%%%%%%%%
%
%       EXERCÍCIO
%
%%%%%%%%%%%%%%%%%%%%%%%%%%%%%%%%%

\ifdefstring{\atividade}{prova}{%
    \ifdefstring{\modo}{objetivo}{%
        \renewcommand{\valorquestao}{\ValorQObj\ ponto}
    }{%
        \renewcommand{\valorquestao}{\ValorQDisc\ pontos}
    }%
}%

\begin{exercicioBanco}[\valorquestao]
A unidade de medida utilizada para anunciar o tamanho das telas de televisores no Brasil é a polegada, que corresponde a 2,54 cm. [...] dizer que a tela de uma TV tem x polegadas significa que a diagonal do retângulo que representa sua tela mede \(x\) polegadas, conforme ilustração.

\medskip

\begin{figure}[H]
    \centering
    \includegraphics[width=0.3\textwidth]{TI_Exercicio9_Fig.png}
\end{figure}

\smallskip

O administrador de um museu recebeu uma TV convencional de 20 polegadas, que tem como razão do comprimento (C) pela altura (A) a proporção \(4:3\), e precisa calcular o comprimento (C) dessa TV a fim de colocá-la em uma estante para exposição. A tela dessa TV tem medida do comprimento C, em centímetro, igual a:

% Define as alternativas
\newcommand{\alternativas}{%
    \begin{center}
        \begin{tabularx}{\textwidth}{XXXXX}
            (a) \(12,00\). &
            (b) \(16,00\). &
            (c) \(30,48\). &
            (d) \(40,64\). &
            (e) \(50,80\).
        \end{tabularx}
    \end{center}
}

% Define a resposta correta
\newcommand{\resposta}{D}

% Lógica condicional para exibição
\ifdefstring{\atividade}{lista}{%
    \alternativas
    \vspace{0.5em}
    
    \noindent\textbf{Resposta:} letra \textbf{\resposta}.
}{%
    \ifdefstring{\modo}{objetivo}{%
        \alternativas
    }{%
        % modo = discursiva → não mostra alternativas
    }
}
\end{exercicioBanco}

