%%%%%%%%%%%%%%%%%%%%%%%%%%%%%%%%%
%
%       EXERCÍCIO
%
%%%%%%%%%%%%%%%%%%%%%%%%%%%%%%%%%

\ifdefstring{\atividade}{prova}{%
    \ifdefstring{\modo}{objetivo}{%
        \renewcommand{\valorquestao}{\ValorQObj\ ponto}
    }{%
        \renewcommand{\valorquestao}{\ValorQDisc\ pontos}
    }%
}%

\begin{exercicioBanco}[\valorquestao]
Na figura abaixo, os pontos A, C e F estão alinhados, \(FC = CA = (1 + 2\sqrt{3})\), EDCF é um quadrado e ABC é um triângulo equilátero. Determine CG.

\medskip

\begin{figure}[H]
    \centering
    \includegraphics[width=0.4\textwidth]{../TI_Exercicio12_Fig.png}
\end{figure}

\smallskip

% Define as alternativas
\newcommand{\alternativas}{%
    \begin{center}
        \begin{tabularx}{\textwidth}{XXXXX}
            (a) \(\dfrac{\sqrt{3}}{2}\). &
            (b) \(2\). &
            (c) \(\dfrac{1}{2}\). &
            (d) \(\sqrt{2}\). &
            (e) \(\dfrac{1}{\sqrt{2}}\).
        \end{tabularx}
    \end{center}
}

% Define a resposta correta
\newcommand{\resposta}{B}

% Lógica condicional para exibição
\ifdefstring{\atividade}{lista}{%
    \alternativas
    \vspace{0.5em}
    
    \noindent\textbf{Resposta:} letra \textbf{\resposta}.
}{%
    \ifdefstring{\modo}{objetivo}{%
        \alternativas
    }{%
        % modo = discursiva → não mostra alternativas
    }
}
\end{exercicioBanco}

