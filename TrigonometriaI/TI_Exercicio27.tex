%%%%%%%%%%%%%%%%%%%%%%%%%%%%%%%%%
%
%       EXERCÍCIO
%
%%%%%%%%%%%%%%%%%%%%%%%%%%%%%%%%%

\ifdefstring{\atividade}{prova}{%
    \ifdefstring{\modo}{objetivo}{%
        \renewcommand{\valorquestao}{\ValorQObj\ ponto}
    }{%
        \renewcommand{\valorquestao}{\ValorQDisc\ pontos}
    }%
}%

\begin{exercicioBanco}[\valorquestao]
A famosa Torre de Pisa, localizada na Itália, assim como muitos outros prédios, por motivos adversos, sofrem inclinações durante ou após suas construções. 

Um prédio, quando construído, dispunha-se verticalmente e tinha 60 metros de altura. Ele sofreu uma inclinação de um ângulo $\alpha$, e a projeção ortogonal de sua fachada lateral sobre o solo tem largura medindo 1,80 metro, conforme mostra a figura.

O valor do ângulo de inclinação pode ser determinado fazendo-se o uso de uma tabela como a apresentada.

\begin{figure}[H]
\centering
\begin{minipage}{0.4\textwidth}
\centering
\begin{tabular}{|>{\centering\arraybackslash}m{3cm}|>{\centering\arraybackslash}m{1cm}|>{\centering\arraybackslash}m{2cm}|}
\hline
\textbf{Ângulo} $\boldsymbol{\alpha}$ \textbf{(grau)}& \textbf{Seno} \\ \hline
0,0 & 0,0 \\ \hline
1,0 & 0,017 \\ \hline
1,5 & 0,026 \\ \hline
1,8 & 0,031 \\ \hline
2,0 & 0,034 \\ \hline
3,0 & 0,052 \\ \hline
\end{tabular}
\end{minipage}%
\hfill
\begin{minipage}{0.4\textwidth}
\centering
\includegraphics[width=0.5\textwidth]{TI_Exercicio27_Fig.png}
\end{minipage}
\end{figure}

Considerando os dados apresentados e a tabela de senos fornecida, determine uma estimativa para o ângulo \(\alpha\).

% Define as alternativas
\newcommand{\alternativas}{%
    \begin{center}
        \begin{tabularx}{\textwidth}{XXXXX}
            (a) \(0 \le \alpha \le 1,0\). &
            (b) \(1,0 \le \alpha \le 1,5\). &
            (c) \(1,5 \le \alpha \le 1,8\). &
            (d) \(1,8 \le \alpha \le 2,0\). &
            (e) \(2,0 \le \alpha \le 3,0\).
        \end{tabularx}
    \end{center}
}

% Define a resposta correta
\newcommand{\resposta}{C}

% Lógica condicional para exibição
\ifdefstring{\atividade}{lista}{%
    \alternativas
    \vspace{0.5em}
    
    \noindent\textbf{Resposta:} letra \textbf{\resposta}.
}{%
    \ifdefstring{\modo}{objetivo}{%
        \alternativas
    }{%
        % modo = discursiva → não mostra alternativas
    }
}
\end{exercicioBanco}

