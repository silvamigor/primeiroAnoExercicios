%%%%%%%%%%%%%%%%%%%%%%%%%%%%%%%%%
%
%       EXERCÍCIO
%
%%%%%%%%%%%%%%%%%%%%%%%%%%%%%%%%%

\ifdefstring{\atividade}{prova}{%
    \ifdefstring{\modo}{objetivo}{%
        \renewcommand{\valorquestao}{\ValorQObj\ ponto}
    }{%
        \renewcommand{\valorquestao}{\ValorQDisc\ pontos}
    }%
}%

\begin{exercicioBanco}[\valorquestao]
Na figura a seguir, ABCD é um retângulo e e AMCN é um losango.

\medskip

\begin{figure}[H]
    \centering
    \includegraphics[width=0.3\textwidth]{../TI_Exercicio7_Fig.png}
\end{figure}

\smallskip

Determine a medida do segmento NB , sabendo que AB = 2AD = 20 cm.

% Define as alternativas
\newcommand{\alternativas}{%
    \begin{center}
        \begin{tabularx}{\textwidth}{XXXXX}
            (a) \(\dfrac{10\sqrt{41}}{4}\) cm. &
            (b) \(\dfrac{1}{2\sqrt{13}}\) cm. &
            (c) \(25\pi\) cm. &
            (d) \(16\) cm. &
            (e) \(20,5\) cm.
        \end{tabularx}
    \end{center}
}

% Define a resposta correta
\newcommand{\resposta}{A}

% Lógica condicional para exibição
\ifdefstring{\atividade}{lista}{%
    \alternativas
    \vspace{0.5em}
    
    \noindent\textbf{Resposta:} letra \textbf{\resposta}.
}{%
    \ifdefstring{\modo}{objetivo}{%
        \alternativas
    }{%
        % modo = discursiva → não mostra alternativas
    }
}
\end{exercicioBanco}

