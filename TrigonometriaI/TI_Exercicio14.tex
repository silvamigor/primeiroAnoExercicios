%%%%%%%%%%%%%%%%%%%%%%%%%%%%%%%%%
%
%       EXERCÍCIO
%
%%%%%%%%%%%%%%%%%%%%%%%%%%%%%%%%%

\ifdefstring{\atividade}{prova}{%
    \ifdefstring{\modo}{objetivo}{%
        \renewcommand{\valorquestao}{\ValorQObj\ ponto}
    }{%
        \renewcommand{\valorquestao}{\ValorQDisc\ pontos}
    }%
}%

\begin{exercicioBanco}[\valorquestao]
Para dar sustentação a um poste telefônico, utilizou-se um outro poste com 8 m de comprimento, fixado ao solo a 4 m de distância do poste telefônico, inclinado sob um ângulo de \(60^\circ\), conforme a figura abaixo.

\medskip

\begin{figure}[H]
    \centering
    \includegraphics[width=0.2\textwidth]{../TI_Exercicio14_Fig.png}
\end{figure}

\smallskip

Considerando-se que foram utilizados 10 m de cabo para ligar os dois postes e usando a aproximação \(\sqrt{3} \approx 1,73\), determine a altura do poste telefônico em relação ao solo.

% Define as alternativas
\newcommand{\alternativas}{%
    \begin{center}
        \begin{tabularx}{\textwidth}{XXXXX}
            (a) \(2,25\) m. &
            (b) \(6,75\) m. &
            (c) \(10,33\) m. &
            (d) \(12,92\) m. &
            (e) \(15,87\) m.
        \end{tabularx}
    \end{center}
}

% Define a resposta correta
\newcommand{\resposta}{D}

% Lógica condicional para exibição
\ifdefstring{\atividade}{lista}{%
    \alternativas
    \vspace{0.5em}
    
    \noindent\textbf{Resposta:} letra \textbf{\resposta}.
}{%
    \ifdefstring{\modo}{objetivo}{%
        \alternativas
    }{%
        % modo = discursiva → não mostra alternativas
    }
}
\end{exercicioBanco}

