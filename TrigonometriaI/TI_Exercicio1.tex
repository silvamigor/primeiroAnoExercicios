%%%%%%%%%%%%%%%%%%%%%%%%%%%%%%%%%
%
%       EXERCÍCIO
%
%%%%%%%%%%%%%%%%%%%%%%%%%%%%%%%%%

\ifdefstring{\atividade}{prova}{%
    \ifdefstring{\modo}{objetivo}{%
        \renewcommand{\valorquestao}{\ValorQObj\ ponto}
    }{%
        \renewcommand{\valorquestao}{\ValorQDisc\ pontos}
    }%
}%

\begin{exercicioBanco}[\valorquestao]
Duas avenidas se encontram em um ponto $A$. Essas avenidas cruzam três ruas, $r_1$, $r_2$ e $r_3$, que são paralelas entre si.  Os segmentos de reta $\overline{AD}$, $\overline{AB}$ e $\overline{BC}$ representam quarteirões da avenida \(2\). Na figura, estão indicados os comprimentos, em metros, de alguns quarteirões.

\medskip

\begin{figure}[H]
    \centering
    \includegraphics[width=0.5\textwidth]{TI_Exercicio1_Fig.png}
\end{figure}

\smallskip

Determine os comprimentos dos quarteirões representados pelos segmentos de reta $\overline{BC}$ e $\overline{AM}$.


% Define as alternativas
\newcommand{\alternativas}{%
    \begin{center}
        \begin{tabularx}{\textwidth}{XXX}
            (a) \(BC = 89,9\) m e \(AM = 25\) m. &
            (b) \(BC = 210\) m e \(AM = 50,2\) m. &
            (c) \(BC = 85,6\) m e \(AM = 20,4\) m. \\[5pt]
            (d) \(BC = 157,2\) m e \(AM = 80,3\) m. &
            (e) \(BC = 128\) m e \(AM = 37,5\) m.
        \end{tabularx}
    \end{center}
}

% Define a resposta correta
\newcommand{\resposta}{E}

% Lógica condicional para exibição
\ifdefstring{\atividade}{lista}{%
    \alternativas
    \vspace{0.5em}
    
    \noindent\textbf{Resposta:} letra \textbf{\resposta}.
}{%
    \ifdefstring{\modo}{objetivo}{%
        \alternativas
    }{%
        % modo = discursiva → não mostra alternativas
    }
}
\end{exercicioBanco}

