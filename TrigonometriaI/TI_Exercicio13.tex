%%%%%%%%%%%%%%%%%%%%%%%%%%%%%%%%%
%
%       EXERCÍCIO
%
%%%%%%%%%%%%%%%%%%%%%%%%%%%%%%%%%

\ifdefstring{\atividade}{prova}{%
    \ifdefstring{\modo}{objetivo}{%
        \renewcommand{\valorquestao}{\ValorQObj\ ponto}
    }{%
        \renewcommand{\valorquestao}{\ValorQDisc\ pontos}
    }%
}%

\begin{exercicioBanco}[\valorquestao]
Um navio, navegando em linha reta, passa sucessivamente pelos pontos \( A \), \( B \) e \( C \).
O comandante, quando o navio está em \( A \), observa um farol \( F \) e determina que o ângulo 
\(F\widehat{A}C\) mede \(30^\circ\). Após navegar \(6\,\text{km}\) até o ponto \(B\), ele verifica que o ângulo 
\(F\widehat{B}C\) mede \(90^\circ\). Calcule a distância, em quilômetros, que separa o farol \(F\) do navio 
quando este se encontra no ponto \(C\), situado a \(2\,\text{km}\) do ponto \(B\).


% Define as alternativas
\newcommand{\alternativas}{%
    \begin{center}
        \begin{tabularx}{\textwidth}{XXXXX}
            (a) \(5\). &
            (b) \(3\). &
            (c) \(7\). &
            (d) \(12\). &
            (e) \(4\).
        \end{tabularx}
    \end{center}
}

% Define a resposta correta
\newcommand{\resposta}{E}

% Lógica condicional para exibição
\ifdefstring{\atividade}{lista}{%
    \alternativas
    \vspace{0.5em}
    
    \noindent\textbf{Resposta:} letra \textbf{\resposta}.
}{%
    \ifdefstring{\modo}{objetivo}{%
        \alternativas
    }{%
        % modo = discursiva → não mostra alternativas
    }
}
\end{exercicioBanco}

