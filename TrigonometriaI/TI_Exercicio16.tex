%%%%%%%%%%%%%%%%%%%%%%%%%%%%%%%%%
%
%       EXERCÍCIO
%
%%%%%%%%%%%%%%%%%%%%%%%%%%%%%%%%%

\ifdefstring{\atividade}{prova}{%
    \ifdefstring{\modo}{objetivo}{%
        \renewcommand{\valorquestao}{\ValorQObj\ ponto}
    }{%
        \renewcommand{\valorquestao}{\ValorQDisc\ pontos}
    }%
}%

\begin{exercicioBanco}[\valorquestao]
Sabendo que \(\cos(\theta) = -\dfrac{3}{7}\) e \(\tg(\theta) < 0\), calcule o valor da expressão \(x = \dfrac{2\tg(\theta)}{1 - \tg^{2}(\theta)}\).

% Define as alternativas
\newcommand{\alternativas}{%
    \begin{center}
        \begin{tabularx}{\textwidth}{XXXXX}
            (a) \(\dfrac{12\sqrt{10}}{31}\). &
            (b) \(-\dfrac{12\sqrt{10}}{31}\). &
            (c) \(\dfrac{4\sqrt{10}}{31}\). &
            (d) \(\dfrac{12\sqrt{10}}{49}\). &
            (e) \(\dfrac{6\sqrt{10}}{31}\).
        \end{tabularx}
    \end{center}
}

% Define a resposta correta
\newcommand{\resposta}{A}

% Lógica condicional para exibição
\ifdefstring{\atividade}{lista}{%
    \alternativas
    \vspace{0.5em}
    
    \noindent\textbf{Resposta:} letra \textbf{\resposta}.
}{%
    \ifdefstring{\modo}{objetivo}{%
        \alternativas
    }{%
        % modo = discursiva → não mostra alternativas
    }
}
\end{exercicioBanco}

