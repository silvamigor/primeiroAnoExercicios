%%%%%%%%%%%%%%%%%%%%%%%%%%%%%%%%%
%
%       EXERCÍCIO
%
%%%%%%%%%%%%%%%%%%%%%%%%%%%%%%%%%

\ifdefstring{\atividade}{prova}{%
    \ifdefstring{\modo}{objetivo}{%
        \renewcommand{\valorquestao}{\ValorQObj\ ponto}
    }{%
        \renewcommand{\valorquestao}{\ValorQDisc\ pontos}
    }%
}%

\begin{exercicioBanco}[\valorquestao]
A partir de um ponto, observa-se o topo de um prédio sob um ângulo de $30^\circ$. 
Caminhando $24\,\text{m}$ em direção ao prédio, atinge-se outro ponto, $C$, 
de onde se vê o topo do prédio segundo um ângulo de $60^\circ$. 

\medskip

\begin{figure}[H]
    \centering
    \includegraphics[width=0.4\textwidth]{../TI_Exercicio10_Fig.png}
\end{figure}

\smallskip

Desprezando a altura do observador, calcule, em metros, a altura do prédio.

% Define as alternativas
\newcommand{\alternativas}{%
    \begin{center}
        \begin{tabularx}{\textwidth}{XXXXX}
            (a) \(6\sqrt{2}\). &
            (b) \(\dfrac{3\sqrt{3}}{2}\). &
            (c) \(12\sqrt{3}\). &
            (d) \(\sqrt{3} + \sqrt{2}\). &
            (e) \(\dfrac{\sqrt{6}}{2}\).
        \end{tabularx}
    \end{center}
}

% Define a resposta correta
\newcommand{\resposta}{C}

% Lógica condicional para exibição
\ifdefstring{\atividade}{lista}{%
    \alternativas
    \vspace{0.5em}
    
    \noindent\textbf{Resposta:} letra \textbf{\resposta}.
}{%
    \ifdefstring{\modo}{objetivo}{%
        \alternativas
    }{%
        % modo = discursiva → não mostra alternativas
    }
}
\end{exercicioBanco}

