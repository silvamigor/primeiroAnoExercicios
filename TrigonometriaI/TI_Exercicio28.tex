%%%%%%%%%%%%%%%%%%%%%%%%%%%%%%%%%
%
%       EXERCÍCIO
%
%%%%%%%%%%%%%%%%%%%%%%%%%%%%%%%%%

\ifdefstring{\atividade}{prova}{%
    \ifdefstring{\modo}{objetivo}{%
        \renewcommand{\valorquestao}{\ValorQObj\ ponto}
    }{%
        \renewcommand{\valorquestao}{\ValorQDisc\ pontos}
    }%
}%

\begin{exercicioBanco}[\valorquestao]
Um empreendedor está desenvolvendo um sistema para auxiliar o julgamento de lances duvidosos em partidas de futebol. Seu projeto consiste de um chip instalado na bola e um sensor posicionado em um dos cantos do campo (ponto \(P\)). O sensor detecta a distância \(r\) entre os pontos \(P\) e \(B\) (bola) e a medida \(\alpha\) do ângulo \(B\widehat{P}Q\). Em seguida, transforma essas informações nas distâncias \(x\) e \(y\) indicadas na figura. Usando relações trigonométricas, deduza expressões para \(x\) e \(y\) em função de \(r\) e \(\alpha\).

\medskip

\begin{figure}[H]
    \centering
    \includegraphics[width=0.4\textwidth]{../TI_Exercicio28_Fig.png}
\end{figure}

\smallskip

% Define as alternativas
\newcommand{\alternativas}{%
    \begin{center}
        \begin{tabularx}{\textwidth}{XXX}
            (a) \(x = \frac{1}{r}\sen(\alpha)\) e \(y = \frac{1}{r}\cos(\alpha)\). &
            (b) \(x = r^{2}\cos(\alpha)\) e \(y = r^{2}\sen(\alpha)\). &
            (c) \(x = r\sen(2\alpha)\) e \(y = r\cos(2\alpha)\). \\[5pt]
            (d) \(x = r\cos(\alpha)\) e \(y = r\sen(\alpha)\). &
            \multicolumn{2}{l}{(e) \(x = \frac{1}{r}\sen(2\alpha)\) e \(y = \frac{1}{r}\cos(2\alpha)\).}
        \end{tabularx}
    \end{center}
}

% Define a resposta correta
\newcommand{\resposta}{D}

% Lógica condicional para exibição
\ifdefstring{\atividade}{lista}{%
    \alternativas
    \vspace{0.5em}
    
    \noindent\textbf{Resposta:} letra \textbf{\resposta}.
}{%
    \ifdefstring{\modo}{objetivo}{%
        \alternativas
    }{%
        % modo = discursiva → não mostra alternativas
    }
}
\end{exercicioBanco}

