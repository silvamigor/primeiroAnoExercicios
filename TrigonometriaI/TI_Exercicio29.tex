%%%%%%%%%%%%%%%%%%%%%%%%%%%%%%%%%
%
%       EXERCÍCIO
%
%%%%%%%%%%%%%%%%%%%%%%%%%%%%%%%%%

\ifdefstring{\atividade}{prova}{%
    \ifdefstring{\modo}{objetivo}{%
        \renewcommand{\valorquestao}{\ValorQObj\ ponto}
    }{%
        \renewcommand{\valorquestao}{\ValorQDisc\ pontos}
    }%
}%

\begin{exercicioBanco}[\valorquestao]
O teodolito é um instrumento de medida de ângulos bastante útil na topografia. Com ele, é possível determinar distâncias que não poderiam ser medidas diretamente. Para calcular a altura de um morro em relação a uma região plana no seu entorno, o topógrafo pode utilizar esse instrumento adotando o seguinte procedimento: situa o teodolito no ponto \(A\) e, mirando o ponto \(T\) no topo do morro, mede o ângulo de 30\(^{\circ}\) com a horizontal; desloca o teodolito 160 metros em direção ao morro, colocando-o agora no ponto \(B\), do qual, novamente mirando o ponto T, mede o ângulo de 60\(^{\circ}\) com a horizontal. Se a altura do teodolito é de 1,5 metros, determine a altura do morro com relação à região plana à qual pertencem \(A\) e \(B\). Observação: a figura a seguir está fora de escala.


\medskip

\begin{figure}[H]
    \centering
    \includegraphics[width=0.3\textwidth]{TI_Exercicio29_Fig.png}
\end{figure}

\smallskip

% Define as alternativas
\newcommand{\alternativas}{%
    \begin{center}
        \begin{tabularx}{\textwidth}{XXX}
            (a) \(80\sqrt{3} + 1,5\) m. &
            (b) \(80\sqrt{3} - 1,5\) m. &
            (c) \(\dfrac{160\sqrt{3}}{3} + 1,5\) m. \\[5pt]
            (d) \(\dfrac{160\sqrt{3}}{3} - 1,5\) m. &
            (e) \(160\sqrt{3}\) m.
        \end{tabularx}
    \end{center}
}

% Define a resposta correta
\newcommand{\resposta}{A}

% Lógica condicional para exibição
\ifdefstring{\atividade}{lista}{%
    \alternativas
    \vspace{0.5em}
    
    \noindent\textbf{Resposta:} letra \textbf{\resposta}.
}{%
    \ifdefstring{\modo}{objetivo}{%
        \alternativas
    }{%
        % modo = discursiva → não mostra alternativas
    }
}
\end{exercicioBanco}

