%%%%%%%%%%%%%%%%%%%%%%%%%%%%%%%%%
%
%       EXERCÍCIO
%
%%%%%%%%%%%%%%%%%%%%%%%%%%%%%%%%%

\ifdefstring{\atividade}{prova}{%
    \ifdefstring{\modo}{objetivo}{%
        \renewcommand{\valorquestao}{\ValorQObj\ ponto}
    }{%
        \renewcommand{\valorquestao}{\ValorQDisc\ pontos}
    }%
}%

\begin{exercicioBanco}[\valorquestao]
A medida da altura de uma árvore é $10\,\text{m}$, 
a medida da distância entre ela e o observador é $50\,\text{m}$ 
e a medida da distância entre a árvore e uma torre é $70\,\text{m}$. 
Considerando que o olho do observador, o topo da árvore e o topo da torre estão alinhados, 
qual é a medida da altura da torre?

\medskip

\begin{figure}[H]
    \centering
    \includegraphics[width=0.4\textwidth]{../TI_Exercicio3_Fig.png}
\end{figure}

\smallskip

% Define as alternativas
\newcommand{\alternativas}{%
    \begin{center}
        \begin{tabularx}{\textwidth}{XXXXX}
            (a) \(14\) m. &
            (b) \(24\) m. &
            (c) \(20\) m. &
            (d) \(12\) m. &
            (e) \(30\) m.
        \end{tabularx}
    \end{center}
}

% Define a resposta correta
\newcommand{\resposta}{B}

% Lógica condicional para exibição
\ifdefstring{\atividade}{lista}{%
    \alternativas
    \vspace{0.5em}
    
    \noindent\textbf{Resposta:} letra \textbf{\resposta}.
}{%
    \ifdefstring{\modo}{objetivo}{%
        \alternativas
    }{%
        % modo = discursiva → não mostra alternativas
    }
}
\end{exercicioBanco}

