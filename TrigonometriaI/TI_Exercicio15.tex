%%%%%%%%%%%%%%%%%%%%%%%%%%%%%%%%%
%
%       EXERCÍCIO
%
%%%%%%%%%%%%%%%%%%%%%%%%%%%%%%%%%

\ifdefstring{\atividade}{prova}{%
    \ifdefstring{\modo}{objetivo}{%
        \renewcommand{\valorquestao}{\ValorQObj\ ponto}
    }{%
        \renewcommand{\valorquestao}{\ValorQDisc\ pontos}
    }%
}%

\begin{exercicioBanco}[\valorquestao]
Um passageiro em um avião avista duas cidades, A e B, sob ângulos de \(15^\circ\) e \(30^\circ\), respectivamente, conforme a figura a seguir

\medskip

\begin{figure}[H]
    \centering
    \includegraphics[width=0.4\textwidth]{TI_Exercicio15_Fig.png}
\end{figure}

\smallskip

Se o avião está a uma altitude de 3 km, a distância entre as cidades A e B é:

% Define as alternativas
\newcommand{\alternativas}{%
    \begin{center}
        \begin{tabularx}{\textwidth}{XXXXX}
            (a) \(7\) km. &
            (b) \(5,5\) km. &
            (c) \(5\) km. &
            (d) \(6,5\) km. &
            (e) \(6\) km.
        \end{tabularx}
    \end{center}
}

% Define a resposta correta
\newcommand{\resposta}{E}

% Lógica condicional para exibição
\ifdefstring{\atividade}{lista}{%
    \alternativas
    \vspace{0.5em}
    
    \noindent\textbf{Resposta:} letra \textbf{\resposta}.
}{%
    \ifdefstring{\modo}{objetivo}{%
        \alternativas
    }{%
        % modo = discursiva → não mostra alternativas
    }
}
\end{exercicioBanco}

