%%%%%%%%%%%%%%%%%%%%%%%%%%%%%%%%%
%
%       EXERCÍCIO
%
%%%%%%%%%%%%%%%%%%%%%%%%%%%%%%%%%

\ifdefstring{\atividade}{prova}{%
    \ifdefstring{\modo}{objetivo}{%
        \renewcommand{\valorquestao}{\ValorQObj\ ponto}
    }{%
        \renewcommand{\valorquestao}{\ValorQDisc\ pontos}
    }%
}%

\begin{exercicioBanco}[\valorquestao]
Um jardim, representado na figura pelo triângulo retângulo \(ABC\), foi dividido em dois canteiros, \(S\) e \(S_1\), por uma grade, indicada pelo segmento \(\overline{BE}\).

\medskip

\begin{figure}[H]
    \centering
    \includegraphics[width=0.3\textwidth]{../TI_Exercicio25_Fig.png}
\end{figure}

\smallskip

Sabendo que \(AB = 6\) m, determine o perímetro do triângulo \(ABE\).

% Define as alternativas
\newcommand{\alternativas}{%
    \begin{center}
        \begin{tabularx}{\textwidth}{XXXXX}
            (a) \(4 + 10\sqrt{3}\) m. &
            (b) \(12\sqrt{3}\) m. &
            (c) \(6 + 6\sqrt{3}\) m. &
            (d) \(14\sqrt{3}\) m. &
            (e) \(6 + 10\sqrt{3}\) m.
        \end{tabularx}
    \end{center}
}

% Define a resposta correta
\newcommand{\resposta}{C}

% Lógica condicional para exibição
\ifdefstring{\atividade}{lista}{%
    \alternativas
    \vspace{0.5em}
    
    \noindent\textbf{Resposta:} letra \textbf{\resposta}.
}{%
    \ifdefstring{\modo}{objetivo}{%
        \alternativas
    }{%
        % modo = discursiva → não mostra alternativas
    }
}
\end{exercicioBanco}

