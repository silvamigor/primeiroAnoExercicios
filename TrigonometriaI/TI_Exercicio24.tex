%%%%%%%%%%%%%%%%%%%%%%%%%%%%%%%%%
%
%       EXERCÍCIO
%
%%%%%%%%%%%%%%%%%%%%%%%%%%%%%%%%%

\ifdefstring{\atividade}{prova}{%
    \ifdefstring{\modo}{objetivo}{%
        \renewcommand{\valorquestao}{\ValorQObj\ ponto}
    }{%
        \renewcommand{\valorquestao}{\ValorQDisc\ pontos}
    }%
}%

\begin{exercicioBanco}[\valorquestao]
Em um cartão quadrado \(ABCD\), de área igual a 256 cm\(^{2}\), destaca-se uma região triangular \(ABP\), conforme mostra a figura.

\medskip

\begin{figure}[H]
    \centering
    \includegraphics[width=0.15\textwidth]{../TI_Exercicio24_Fig.png}
\end{figure}

\smallskip

Determine o perímetro da região delimitada pelo triângulo \(ABP\).

% Define as alternativas
\newcommand{\alternativas}{%
    \begin{center}
        \begin{tabularx}{\textwidth}{XXXXX}
            (a) \(8(2 + \sqrt{2})\). &
            (b) \(6(3 + \sqrt{3})\). &
            (c) \(24\sqrt{3}\). &
            (d) \(8(3 + \sqrt{3})\). &
            (e) \(32\sqrt{3}\).
        \end{tabularx}
    \end{center}
}

% Define a resposta correta
\newcommand{\resposta}{D}

% Lógica condicional para exibição
\ifdefstring{\atividade}{lista}{%
    \alternativas
    \vspace{0.5em}
    
    \noindent\textbf{Resposta:} letra \textbf{\resposta}.
}{%
    \ifdefstring{\modo}{objetivo}{%
        \alternativas
    }{%
        % modo = discursiva → não mostra alternativas
    }
}
\end{exercicioBanco}

