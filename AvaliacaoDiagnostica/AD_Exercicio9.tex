%%%%%%%%%%%%%%%%%%%%%%%%%%%%%%%%%
%
%       EXERCÍCIO
%
%%%%%%%%%%%%%%%%%%%%%%%%%%%%%%%%%

\ifdefstring{\atividade}{prova}{%
    \ifdefstring{\modo}{objetivo}{%
        \renewcommand{\valorquestao}{\ValorQObj\ ponto}
    }{%
        \renewcommand{\valorquestao}{\ValorQDisc\ pontos}
    }%
}%

\begin{exercicioBanco}[\valorquestao]
Na figura abaixo, os segmentos \(\overline{AB}\) e \(\overline{MC}\) são paralelos. Determine a medida de \(MC\) em metros.

\medskip

%
\begin{center}
    \begin{tikzpicture}[scale=0.5]

        % Pontos no solo
        \coordinate (O) at (0,0);
        \coordinate (A) at (5,0);
        \coordinate (M) at (12,0);

        % Pontos superiores
        \coordinate (B) at (5,1.2);
        \coordinate (C) at (12,3);

        % Solo
        \draw[thick] (O) -- (M);

        % Verticais
        \draw (A) -- (B);
        \draw (M) -- (C);

        % Linha de visada
        \draw[dashed] (O) -- (C);

        % Pontos
        \fill (O) circle (0.1);
        \fill (A) circle (0.1);
        \fill (M) circle (0.1);

        % Rótulos dos pontos
        \node[below] at (O) {$O$};
        \node[below] at (A) {$A$};
        \node[below] at (M) {$M$};
        \node[above] at (B) {$B$};
        \node[above] at (C) {$C$};

        % Medidas horizontais
        \node[below] at ($(O)!0.5!(A)$) {$50\,\text{m}$};
        \node[below] at ($(A)!0.5!(M)$) {$70\,\text{m}$};

        % Medida vertical
        \node[right] at ($(A)!0.5!(B)$) {$10\,\text{m}$};

    \end{tikzpicture}
\end{center}
%

\medskip

% Define as alternativas
\newcommand{\alternativas}{%
    \begin{center}
        \begin{tabularx}{\textwidth}{XXXXX}
            (a) \(20\). &
            (b) \(22\). &
            (c) \(24\). &
            (d) \(30\). &
            (e) \(60\).
        \end{tabularx}
    \end{center}
}

% Define a resposta correta
\newcommand{\resposta}{C}

% Lógica condicional para exibição
\ifdefstring{\atividade}{lista}{%
    \alternativas
    \vspace{0.5em}
    
    \noindent\textbf{Resposta:} letra \textbf{\resposta}.
}{%
    \ifdefstring{\modo}{objetivo}{%
        \alternativas
    }{%
        % modo = discursiva → não mostra alternativas
    }
}
\end{exercicioBanco}

