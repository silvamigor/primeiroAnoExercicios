%%%%%%%%%%%%%%%%%%%%%%%%%%%%%%%%%
%
%       EXERCÍCIO
%
%%%%%%%%%%%%%%%%%%%%%%%%%%%%%%%%%

\ifdefstring{\atividade}{prova}{%
    \ifdefstring{\modo}{objetivo}{%
        \renewcommand{\valorquestao}{\ValorQObj\ ponto}
    }{%
        \renewcommand{\valorquestao}{\ValorQDisc\ pontos}
    }%
}%

\begin{exercicioBanco}[\valorquestao]
A água comercializada em garrafões pode ser classificada como muito ácida, ácida, neutra, alcalina ou muito alcalina, dependendo de seu pH, dado pela expressão

\[
    \text{pH} = \log\left(\dfrac{1}{H}\right),
\]

em que \(H\) é a concentração de íons de hidrogênio, em mol por decímetro cúbico. A classificação da água de acordo com seu pH é mostrado no quadro.

\[
    \begin{array}{|c|c|}
        \hline
        \textbf{pH} & \textbf{Classificação} \\[2pt]
        \hline
        \text{pH} \ge 9 & \text{Muito alcalina} \\[2pt]
        7,5 \leq \text{pH} < 9 & \text{Alcalina} \\[2pt]
        6 \leq \text{pH} < 7,5 & \text{Neutra} \\[2pt]
        3,5 \leq \text{pH} < 6 & \text{Ácida} \\[2pt]
        \text{pH} \ge 3,5 & \text{Muito ácida} \\
        \hline
    \end{array}
\]

Para o cálculo da concentração de \(H\), uma empresa distribuidora mede dois parâmetros, \(A\) e \(B\), em cada fonte, e adota \(H\) como sendo o quociente de \(A\) por \(B\). Em análise realizada em uma fonte, obteve \(A = 10^{-7}\) e a água dessa fonte foi classificada como neutra. 

O parâmetro \(B\), então, encontra-se no intervalo:


% Define as alternativas
\newcommand{\alternativas}{%
    \begin{center}
        \begin{tabularx}{\textwidth}{XXX}
            (a) \(]-10^{145},-10^{13}]\). &
            (b) \([10^{-\frac{6}{7}},10^{-1}[\). &
            (c) \([10^{-1},10^{\frac{1}{2}}[\). \\[5pt]
            (d) \([10^{13},10^{145}[\). &
            (e) \([10^{6\cdot 10^{7}},10^{7,5\cdot 10^{7}}[\). &
        \end{tabularx}
    \end{center}
}

% Define a resposta correta
\newcommand{\resposta}{C}

% Lógica condicional para exibição
\ifdefstring{\atividade}{lista}{%
    \alternativas
    \vspace{0.5em}
    
    \noindent\textbf{Resposta:} letra \textbf{\resposta}.
}{%
    \ifdefstring{\modo}{objetivo}{%
        \alternativas
    }{%
        % modo = discursiva → não mostra alternativas
    }
}
\end{exercicioBanco}

