%%%%%%%%%%%%%%%%%%%%%%%%%%%%%%%%%
%
%       EXERCÍCIO
%
%%%%%%%%%%%%%%%%%%%%%%%%%%%%%%%%%

\ifdefstring{\atividade}{prova}{%
    \ifdefstring{\modo}{objetivo}{%
        \renewcommand{\valorquestao}{\ValorQObj\ ponto}
    }{%
        \renewcommand{\valorquestao}{\ValorQDisc\ pontos}
    }%
}%

\begin{exercicioBanco}[\valorquestao]
Se \(\log_{7}(875) = a\), então \(\log_{35}(245)\) é igual a:

% Define as alternativas
\newcommand{\alternativas}{%
    \begin{center}
        \begin{tabularx}{\textwidth}{XXXXX}
            (a) \(\dfrac{a+2}{a+7}\). &
            (b) \(\dfrac{a+2}{a+5}\). &
            (c) \(\dfrac{a+5}{a+2}\). &
            (d) \(\dfrac{a+7}{a+2}\). &
            (e) \(\dfrac{a+5}{a+7}\).
        \end{tabularx}
    \end{center}
}

% Define a resposta correta
\newcommand{\resposta}{C}

% Lógica condicional para exibição
\ifdefstring{\atividade}{lista}{%
    \alternativas
    \vspace{0.5em}
    
    \noindent\textbf{Resposta:} letra \textbf{\resposta}.
}{%
    \ifdefstring{\modo}{objetivo}{%
        \alternativas
    }{%
        % modo = discursiva → não mostra alternativas
    }
}
\end{exercicioBanco}

