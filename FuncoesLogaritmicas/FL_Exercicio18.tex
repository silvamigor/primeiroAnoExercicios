%%%%%%%%%%%%%%%%%%%%%%%%%%%%%%%%%
%
%       EXERCÍCIO
%
%%%%%%%%%%%%%%%%%%%%%%%%%%%%%%%%%

\ifdefstring{\atividade}{prova}{%
    \ifdefstring{\modo}{objetivo}{%
        \renewcommand{\valorquestao}{\ValorQObj\ ponto}
    }{%
        \renewcommand{\valorquestao}{\ValorQDisc\ pontos}
    }%
}%

\begin{exercicioBanco}[\valorquestao]
Um jardineiro cultiva plantas ornamentais e as coloca à venda quando estas atingem \(30\) centímetros de altura. Esse jardineiro estudou o crescimento de suas plantas, em função do tempo, e deduziu uma fórmula que calcula a altura em função do tempo, a partir do momento em que a planta brota do solo até o momento em que ela atinge sua altura máxima de \(40\) centímetros. A fórmula é

\[
h = 5 \cdot \log_{2}(t + 1),
\]

em que \(t\) é o tempo contado em dias e \(h\), a altura da planta em centímetros.

A partir do momento em que uma dessas plantas é colocada à venda, em quanto tempo, em dias, ela alcançará sua altura máxima?


% Define as alternativas
\newcommand{\alternativas}{%
    \begin{center}
        \begin{tabularx}{\textwidth}{XXXXX}
            (a) \(63\). &
            (b) \(96\). &
            (c) \(128\). &
            (d) \(192\). &
            (e) \(255\).
        \end{tabularx}
    \end{center}
}

% Define a resposta correta
\newcommand{\resposta}{D}

% Lógica condicional para exibição
\ifdefstring{\atividade}{lista}{%
    \alternativas
    \vspace{0.5em}
    
    \noindent\textbf{Resposta:} letra \textbf{\resposta}.
}{%
    \ifdefstring{\modo}{objetivo}{%
        \alternativas
    }{%
        % modo = discursiva → não mostra alternativas
    }
}
\end{exercicioBanco}

