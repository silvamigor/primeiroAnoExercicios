%%%%%%%%%%%%%%%%%%%%%%%%%%%%%%%%%
%
%       EXERCÍCIO
%
%%%%%%%%%%%%%%%%%%%%%%%%%%%%%%%%%

\ifdefstring{\atividade}{prova}{%
    \ifdefstring{\modo}{objetivo}{%
        \renewcommand{\valorquestao}{\ValorQObj\ ponto}
    }{%
        \renewcommand{\valorquestao}{\ValorQDisc\ pontos}
    }%
}%

\begin{exercicioBanco}[\valorquestao]
Na figura, temos o gráfico da função \(f(x) = a + \log_{b}(x)\). Determine o valor de \(a + b\).

\begin{center}
    \begin{tikzpicture}[scale=0.7]
        % Eixos
        \draw[->] (-0.5,0) -- (9.5,0) node[right] {$x$};
        \draw[->] (0,-1) -- (0,2) node[above] {$y$};

        % Curva y = log2(x/4)
        \draw[domain=1.4:9, smooth, samples=200, thick, purple] plot (\x, {ln(\x/4)/ln(2)}) node[above right] {\(f(x)\)};

        % Linha tracejada horizontal (y=1)
        \draw[dashed, gray] (0,1) -- (8,1);

        % Linha tracejada vertical (x=8)
        \draw[dashed, gray] (8,0) -- (8,1);

        % Ponto em (8,1)
        \filldraw (8,1) circle (1.5pt);

        % Marcas nos eixos
        \node[below] at (4,0) {4};
        \node[below] at (8,0) {8};
        \node[left] at (0,1) {1};
    \end{tikzpicture}
\end{center}
 
% Define as alternativas
\newcommand{\alternativas}{%
    \begin{center}
        \begin{tabularx}{\textwidth}{XXXXX}
            (a) \(-1\). &
            (b) \(\dfrac{1}{5}\). &
            (c) \(2^{3}\). &
            (d) \(\log_{2}(3)\). &
            (e) \(0\).
        \end{tabularx}
    \end{center}
}

% Define a resposta correta
\newcommand{\resposta}{E}

% Lógica condicional para exibição
\ifdefstring{\atividade}{lista}{%
    \alternativas
    \vspace{0.5em}
    
    \noindent\textbf{Resposta:} letra \textbf{\resposta}.
}{%
    \ifdefstring{\modo}{objetivo}{%
        \alternativas
    }{%
        % modo = discursiva → não mostra alternativas
    }
}
\end{exercicioBanco}

