\documentclass[a4paper,12pt]{article}

% --------------------------------------------------------
% CODIFICAÇÃO E TIPOGRAFIA
% --------------------------------------------------------
\usepackage[utf8]{inputenc}       % Permite acentuação direta no código-fonte
\usepackage[T1]{fontenc}          % Usa codificação T1 (melhor para português)
\usepackage[brazil]{babel}        % Tradução e hifenização para português
\usepackage{microtype}            % Melhora o espaçamento entre letras/palavras
\usepackage{parskip}              % Espaço entre parágrafos em vez de recuo

% --------------------------------------------------------
% PACOTES MATEMÁTICOS
% --------------------------------------------------------
\usepackage{amsmath}              % Ambientes matemáticos como align, equation
%\usepackage{amssymb}              % Símbolos matemáticos adicionais
\usepackage{amsthm}               % Ambientes de teoremas
\usepackage{mathtools}            % Extensões do amsmath (ex: \coloneqq)
\usepackage{mathrsfs}             % Fonte caligráfica com \mathscr
\usepackage{bbm}                  % Indicador \mathbbm{1}
\usepackage{dsfont}               % Alternativa para conjuntos: \mathds
\usepackage{commath}              % Notações como \abs{}, \norm{}, \eval{}
\usepackage{pgfmath}             % Permite realizar cálculos matemáticos (somas, produtos, números aleatórios, etc.)
\usepackage{siunitx}              % sistema de unidades
\sisetup{group-separator = {\,}} % espaço fino

% --------------------------------------------------------
% FONTE MODERNA (TEXTO E MATEMÁTICA)
% --------------------------------------------------------
%\usepackage{newtxtext}            % Fonte do texto (estilo Times)
%\usepackage{newtxmath}            % Fonte matemática compatível com pacotes AMS
\usepackage[bitstream-charter]{mathdesign} % Fonte elegante e matemática harmoniosa

% --------------------------------------------------------
% AMBIENTES TEÓRICOS PERSONALIZADOS
% --------------------------------------------------------
\theoremstyle{definition}
\newtheorem{definicao}{Definição}       % Definição numerada por seção
\newtheorem{exemplo}{Exemplo}           % Exemplo numerado por seção

%\theoremstyle{plain}
\newtheorem{teorema}{Teorema}           % Teorema numerado por seção
\newtheorem{lema}[teorema]{Lema}                 % Lema com numeração conjunta
\newtheorem{proposicao}[teorema]{Proposição}     % Proposição idem
\newtheorem{corolario}[teorema]{Corolário}       % Corolário idem
\newtheorem{exercicio}[teorema]{Exercício}

%\theoremstyle{remark}
\newtheorem{observacao}[teorema]{Observação}     % Observação com mesmo contador

% --------------------------------------------------------
% TABELAS E FIGURAS
% --------------------------------------------------------
\usepackage{graphicx}             % Inclusão de imagens
\usepackage{float}                % Controle de posicionamento (ex: [H])
\usepackage{caption}              % Personalização de legendas
\usepackage{subcaption}          % Subfiguras com \begin{subfigure}
\usepackage{booktabs}            % Tabelas com qualidade profissional
\usepackage{array}               % Mais opções em colunas de tabelas
\usepackage{multirow}            % Células que ocupam várias linhas
\usepackage{multicol}            % Disposição em colunas múltiplas
\usepackage{tabularx}            % Tabelas com colunas de largura ajustável
\usepackage{longtable}           % Tabelas que quebram página
\usepackage{arydshln}            % Linhas tracejadas horizontais e verticais
\usepackage{makecell}            % Células com múltiplas linhas (e quebra de linha)
\usepackage{diagbox}             % Cabeçalho diagonal em tabelas
\usepackage{pdflscape}           % Gira páginas (modo paisagem)
\newcolumntype{L}[1]{>{\raggedright\let\newline\\\arraybackslash\hspace{0pt}}m{#1}}
\newcolumntype{C}[1]{>{\centering\let\newline\\\arraybackslash\hspace{0pt}}m{#1}}
\newcolumntype{R}[1]{>{\raggedleft\let\newline\\\arraybackslash\hspace{0pt}}m{#1}}

% --------------------------------------------------------
% FORMATAÇÃO E ESTRUTURA
% --------------------------------------------------------
\usepackage{geometry}            % Configuração de margens
\geometry{a4paper, left=1cm, right=1cm, bottom=1.5cm, top=1.5cm, headsep=1cm, footskip=1cm}
\usepackage{titlesec}            % Controle sobre títulos de seções
\usepackage{fancyhdr}            % Cabeçalhos e rodapés personalizados
\pagestyle{fancy}
\fancyhf{}                % Limpa cabeçalho e rodapé
\fancyfoot[R]{\thepage}   % Numeração à direita no rodapé
\renewcommand{\headrulewidth}{0pt} % Remove linha no topo
\usepackage{enumitem}            % Listas com controle de espaçamento e símbolo
\usepackage{tikz}                % Desenho de gráficos vetoriais (diagramas, etc.)
\usetikzlibrary{matrix,arrows,patterns,snakes,decorations.pathreplacing,3d,arrows.meta,calc}
\usepackage{etoolbox}             % Permite lógica condicional e manipulação de comandos
\usepackage{background}           % Permite adicionar conteúdo fixo no fundo da página (ex: linhas, marcas d'água)
\SetBgContents{} % Remove conteúdo padrão ("DRAFT")
\definecolor{background-color}{RGB}{233 227 206}

% --------------------------------------------------------
% GRÁFICOS E PLOTAGENS
% --------------------------------------------------------
\usepackage{pgfplots}             % Criação de gráficos vetoriais em 2D e 3D
\pgfplotsset{compat=1.18}         % Compatibilidade com versão atual do pacote
\usepackage{currfile} % Permite saber qual é o caminho completo do arquivo atual,

% --------------------------------------------------------
% LINKS E CORES
% --------------------------------------------------------
\usepackage{xcolor}              % Definição de cores (texto, links, etc.)
\usepackage[hidelinks]{hyperref} % Links sem moldura colorida no PDF
\usepackage{url}                 % Quebra automática de URLs
% Dica: se quiser links coloridos, use:
% \usepackage[colorlinks=true, linkcolor=blue, citecolor=blue, urlcolor=blue]{hyperref}

% --------------------------------------------------------
% CAIXAS E DESTAQUES VISUAIS
% --------------------------------------------------------
\usepackage[most]{tcolorbox}      % Criação de caixas coloridas e personalizadas para destaques, avisos, exemplos, etc.

% --------------------------------------------------------
% CÓDIGOS E ALGORITMOS
% --------------------------------------------------------
\usepackage{listings}            % Inclusão de código-fonte com destaque
\usepackage{algorithm}           % Ambiente de algoritmo
\usepackage[noend]{algpseudocode} % Sintaxe tipo pseudocódigo


% --------------------------------------------------------
% COMANDOS
% --------------------------------------------------------

\newcommand{\bbN}{\mathbb{N}}
\newcommand{\bbZ}{\mathbb{Z}}
\newcommand{\bbQ}{\mathbb{Q}}
\newcommand{\bbR}{\mathbb{R}}

\newcommand{\assunto}{FUNÇÕES LOGARÍTMICAS}
\newcommand{\atividade}{lista} % prova ou lista
\newcommand{\turma}{CONTROLE AMBIENTAL 1.\textordmasculine\ ANO}
\newcommand{\numProva}{A}
\newcommand{\professor}{Igor Martins Silva}
\newcommand{\bimestre}{1}
\newcommand{\ano}{2026}
\newcommand{\mesTexto}{fevereiro}
\newcommand{\mesNum}{02}
\newcommand{\dia}{19}
\newcommand{\dataTexto}{\dia\ de \mesTexto\ de \ano}
\newcommand{\dataNun}{\dia/\mesNum/\ano}

\newcommand{\titulo}{%
  \ifdefstring{\atividade}{prova}{Avaliação de Matemática}{%
  \ifdefstring{\atividade}{lista}{Lista de exercícios de Matemática}{%
  \ifdefstring{\atividade}{as}{Avaliação Somativa}{%
  \ifdefstring{\atividade}{ad}{Avaliação Diagnóstica}{%
  Atividade de Matemática}}}}%
}

\newcommand{\emtipo}{%
  \ifdefstring{\tipo}{individual}{\tipo}{em \tipo}%
}

\newcommand{\subtitulobase}[3]{%
    % #1 = título da 1ª coluna
    % #2 = conteúdo da 1ª coluna
    % #3 = mostra número da prova? (vazio ou algo)
    
    \begin{center}
        CEFET-MG -- Campus Contagem \\[3pt]
        \titulo\ -- \bimestre.\textordmasculine\ Bimestre de \ano \\[3pt]
        \professor
    \end{center}
    
    \begin{center}
        \vspace{15pt}
        \begin{tabular}{|C{210pt}|C{70pt}|C{210pt}|}
            \hline
            \textbf{#1} &
            \textbf{DATA} &
            \textbf{TURMA} \\
            \hline
            #2 & \dataNun & \turma \\
            \hline
        \end{tabular}
    \end{center}
    
    #3
    
    \vspace{15pt}
}

\newcommand{\subtitulolis}{%
  \subtitulobase{ASSUNTO}{\assunto}{}%
}

\newcommand{\subtitulopr}{%
  \subtitulobase{ASSUNTO}{\assunto}{\boxed{\numProva}}%
}

\newcommand{\subtituloas}{%
  \subtitulobase{DISCIPLINA}{MATEMÁTICA}{}%
}

\newcommand{\subtituload}{%
  \subtitulobase{DISCIPLINA}{MATEMÁTICA}{}%
}

\newcommand{\valorquestao}{}

\newenvironment{exercicioBanco}[1][]{%
    \ifdefstring{\atividade}{prova}{%
        \begin{exercicio}[#1]%
    }{%
        \begin{exercicio}%
    }%
}{%
    \end{exercicio}
}

\ifdefstring{\atividade}{lista}{
    \pagecolor{background-color}
}{
    
}



\hypersetup{
    pdfauthor={\professor},
    pdftitle={\titulo -- \assunto},
}

\title{\titulo -- \assunto}
\author{\professor}
\date{\data}

%************* INÍCIO DO DOCUMENTO *************
\begin{document}
    \subtitulo
        
    \phantomsection
    \addcontentsline{toc}{section}{Exercício 1}
    \input{../FL_Exercicio1}
    \noindent\rule{\linewidth}{1pt}  % linha horizontal fina
    
    \phantomsection
    \addcontentsline{toc}{section}{Exercício 2}
    %%%%%%%%%%%%%%%%%%%%%%%%%%%%%%%%%
%
%       EXERCÍCIO
%
%%%%%%%%%%%%%%%%%%%%%%%%%%%%%%%%%

\ifdefstring{\atividade}{prova}{%
    \ifdefstring{\modo}{objetivo}{%
        \renewcommand{\valorquestao}{\ValorQObj\ ponto}
    }{%
        \renewcommand{\valorquestao}{\ValorQDisc\ pontos}
    }%
}%

\begin{exercicioBanco}[\valorquestao]
Se \(\log_{3}(x) = 3,6704\), então:

% Define as alternativas
\newcommand{\alternativas}{%
    \begin{center}
        \begin{tabularx}{\textwidth}{XXXXX}
            (a) \(0 < x < 3\). &
            (b) \(3 < x < 10\). &
            (c) \(10 < x < 27\). &
            (d) \(27 < x < 81\). &
            (e) \(81 < x < 100\).
        \end{tabularx}
    \end{center}
}

% Define a resposta correta
\newcommand{\resposta}{D}

% Lógica condicional para exibição
\ifdefstring{\atividade}{lista}{%
    \alternativas
    \vspace{0.5em}
    
    \noindent\textbf{Resposta:} letra \textbf{\resposta}.
}{%
    \ifdefstring{\modo}{objetivo}{%
        \alternativas
    }{%
        % modo = discursiva → não mostra alternativas
    }
}
\end{exercicioBanco}


    \noindent\rule{\linewidth}{1pt}  % linha horizontal fina
    
    \phantomsection
    \addcontentsline{toc}{section}{Exercício 3}
    %%%%%%%%%%%%%%%%%%%%%%%%%%%%%%%%%
%
%       EXERCÍCIO
%
%%%%%%%%%%%%%%%%%%%%%%%%%%%%%%%%%

\ifdefstring{\atividade}{prova}{%
    \ifdefstring{\modo}{objetivo}{%
        \renewcommand{\valorquestao}{\ValorQObj\ ponto}
    }{%
        \renewcommand{\valorquestao}{\ValorQDisc\ pontos}
    }%
}%

\begin{exercicioBanco}[\valorquestao]
Calcule a expressão \(4^{\log_{2}(7)} + \log_{2}(8^{7})\).

% Define as alternativas
\newcommand{\alternativas}{%
    \begin{center}
        \begin{tabularx}{\textwidth}{XXXXX}
            (a) \(35\). &
            (b) \(56\). &
            (c) \(49\). &
            (d) \(70\). &
            (e) \(81\).
        \end{tabularx}
    \end{center}
}

% Define a resposta correta
\newcommand{\resposta}{D}

% Lógica condicional para exibição
\ifdefstring{\atividade}{lista}{%
    \alternativas
    \vspace{0.5em}
    
    \noindent\textbf{Resposta:} letra \textbf{\resposta}.
}{%
    \ifdefstring{\modo}{objetivo}{%
        \alternativas
    }{%
        % modo = discursiva → não mostra alternativas
    }
}
\end{exercicioBanco}


    \noindent\rule{\linewidth}{1pt}  % linha horizontal fina
    
    \phantomsection
    \addcontentsline{toc}{section}{Exercício 4}
    \input{../FL_Exercicio4}
    \noindent\rule{\linewidth}{1pt}  % linha horizontal fina
    
    \phantomsection
    \addcontentsline{toc}{section}{Exercício 5}
    %%%%%%%%%%%%%%%%%%%%%%%%%%%%%%%%%
%
%       EXERCÍCIO
%
%%%%%%%%%%%%%%%%%%%%%%%%%%%%%%%%%

\ifdefstring{\atividade}{prova}{%
    \ifdefstring{\modo}{objetivo}{%
        \renewcommand{\valorquestao}{\ValorQObj\ ponto}
    }{%
        \renewcommand{\valorquestao}{\ValorQDisc\ pontos}
    }%
}%

\begin{exercicioBanco}[\valorquestao]
Se \(n\) é um número inteiro maior do que \(2\), o valor 
%
\[
    \log_{n}\left(\log_{n}\left(\sqrt[n]{\sqrt[n]{\sqrt[n]{\sqrt[n]{n}}}}\right)\right)
\]
%
é:

% Define as alternativas
\newcommand{\alternativas}{%
    \begin{center}
        \begin{tabularx}{\textwidth}{XXXXX}
            (a) \(3\). &
            (b) \(-3\). &
            (c) \(4\). &
            (d) \(-4\). &
            (e) \(2\).
        \end{tabularx}
    \end{center}
}

% Define a resposta correta
\newcommand{\resposta}{D}

% Lógica condicional para exibição
\ifdefstring{\atividade}{lista}{%
    \alternativas
    \vspace{0.5em}
    
    \noindent\textbf{Resposta:} letra \textbf{\resposta}.
}{%
    \ifdefstring{\modo}{objetivo}{%
        \alternativas
    }{%
        % modo = discursiva → não mostra alternativas
    }
}
\end{exercicioBanco}


    \noindent\rule{\linewidth}{1pt}  % linha horizontal fina
    
    \phantomsection
    \addcontentsline{toc}{section}{Exercício 6}
    %%%%%%%%%%%%%%%%%%%%%%%%%%%%%%%%%
%
%       EXERCÍCIO
%
%%%%%%%%%%%%%%%%%%%%%%%%%%%%%%%%%

\ifdefstring{\atividade}{prova}{%
    \ifdefstring{\modo}{objetivo}{%
        \renewcommand{\valorquestao}{\ValorQObj\ ponto}
    }{%
        \renewcommand{\valorquestao}{\ValorQDisc\ pontos}
    }%
}%

\begin{exercicioBanco}[\valorquestao]
Sejam \(\log(5) = m\), \(\log(2) = p\) e \(N = 125\sqrt[\raisebox{0.7ex}{$\scriptscriptstyle 3$}]{\dfrac{1562,5}{\sqrt[5]{2}}}\). O valor de \(\log_{5}(N)\) em função de \(m\) e \(p\) é:

% Define as alternativas
\newcommand{\alternativas}{%
    \begin{center}
        \begin{tabularx}{\textwidth}{XXXXX}
            (a) \(\dfrac{75m + 6p}{15m}\). &
            (b) \(\dfrac{70m - 6p}{15m}\). &
            (c) \(\dfrac{75m - 6p}{15m}\). &
            (d) \(\dfrac{70m + 6p}{15m}\). &
            (e) \(\dfrac{70m + 6p}{15p}\).
        \end{tabularx}
    \end{center}
}

% Define a resposta correta
\newcommand{\resposta}{B}

% Lógica condicional para exibição
\ifdefstring{\atividade}{lista}{%
    \alternativas
    \vspace{0.5em}
    
    \noindent\textbf{Resposta:} letra \textbf{\resposta}.
}{%
    \ifdefstring{\modo}{objetivo}{%
        \alternativas
    }{%
        % modo = discursiva → não mostra alternativas
    }
}
\end{exercicioBanco}


    \noindent\rule{\linewidth}{1pt}  % linha horizontal fina
    
    \phantomsection
    \addcontentsline{toc}{section}{Exercício 7}
    %%%%%%%%%%%%%%%%%%%%%%%%%%%%%%%%%
%
%       EXERCÍCIO
%
%%%%%%%%%%%%%%%%%%%%%%%%%%%%%%%%%

\ifdefstring{\atividade}{prova}{%
    \ifdefstring{\modo}{objetivo}{%
        \renewcommand{\valorquestao}{\ValorQObj\ ponto}
    }{%
        \renewcommand{\valorquestao}{\ValorQDisc\ pontos}
    }%
}%

\begin{exercicioBanco}[\valorquestao]
Para quais valores reais de \(x\) existe \(\log_{x+1}(x^{2} + 3x - 18)\)?

% Define as alternativas
\newcommand{\alternativas}{%
    \begin{center}
        \begin{tabularx}{\textwidth}{XXXXX}
            (a) \(\bbR\). &
            (b) \(\varnothing\). &
            (c) \([0,\infty[\). &
            (d) \(]-\infty,-2]\). &
            (e) \(]3,\infty[\).
        \end{tabularx}
    \end{center}
}

% Define a resposta correta
\newcommand{\resposta}{E}

% Lógica condicional para exibição
\ifdefstring{\atividade}{lista}{%
    \alternativas
    \vspace{0.5em}
    
    \noindent\textbf{Resposta:} letra \textbf{\resposta}.
}{%
    \ifdefstring{\modo}{objetivo}{%
        \alternativas
    }{%
        % modo = discursiva → não mostra alternativas
    }
}
\end{exercicioBanco}


    \noindent\rule{\linewidth}{1pt}  % linha horizontal fina
    
    \phantomsection
    \addcontentsline{toc}{section}{Exercício 8}
    %%%%%%%%%%%%%%%%%%%%%%%%%%%%%%%%%
%
%       EXERCÍCIO
%
%%%%%%%%%%%%%%%%%%%%%%%%%%%%%%%%%

\ifdefstring{\atividade}{prova}{%
    \ifdefstring{\modo}{objetivo}{%
        \renewcommand{\valorquestao}{\ValorQObj\ ponto}
    }{%
        \renewcommand{\valorquestao}{\ValorQDisc\ pontos}
    }%
}%

\begin{exercicioBanco}[\valorquestao]
Para quais valores reais de \(x\) existe \(\log_{5}(5x-2) + \log_{5}(x-3)\)?

% Define as alternativas
\newcommand{\alternativas}{%
    \begin{center}
        \begin{tabularx}{\textwidth}{XXX}
            (a) \(\{x \in \bbR \mid x > 3\}\). &
            (b) \(\{x \in \bbR \mid 1 < x < 2\}\). &
            (c) \(\{x \in \bbR \mid x \ge 0\}\). \\[5pt]
            (d) \(\{x \in \bbR \mid x \le 0\}\). &
            (e) \(\{x \in \bbR \mid -1 \le x \le 1\}\). &
        \end{tabularx}
    \end{center}
}

% Define a resposta correta
\newcommand{\resposta}{A}

% Lógica condicional para exibição
\ifdefstring{\atividade}{lista}{%
    \alternativas
    \vspace{0.5em}
    
    \noindent\textbf{Resposta:} letra \textbf{\resposta}.
}{%
    \ifdefstring{\modo}{objetivo}{%
        \alternativas
    }{%
        % modo = discursiva → não mostra alternativas
    }
}
\end{exercicioBanco}


    \noindent\rule{\linewidth}{1pt}  % linha horizontal fina
    
    \phantomsection
    \addcontentsline{toc}{section}{Exercício 9}
    %%%%%%%%%%%%%%%%%%%%%%%%%%%%%%%%%
%
%       EXERCÍCIO
%
%%%%%%%%%%%%%%%%%%%%%%%%%%%%%%%%%

\ifdefstring{\atividade}{prova}{%
    \ifdefstring{\modo}{objetivo}{%
        \renewcommand{\valorquestao}{\ValorQObj\ ponto}
    }{%
        \renewcommand{\valorquestao}{\ValorQDisc\ pontos}
    }%
}%

\begin{exercicioBanco}[\valorquestao]
Determine o valor de \(\log_{\sqrt{2}}\big(\log_{3}(2)\cdot \log_{4}(3)\big)\).

% Define as alternativas
\newcommand{\alternativas}{%
    \begin{center}
        \begin{tabularx}{\textwidth}{XXXXX}
            (a) \(-2\). &
            (b) \(-1\). &
            (c) \(0\). &
            (d) \(1\). &
            (e) \(2\).
        \end{tabularx}
    \end{center}
}

% Define a resposta correta
\newcommand{\resposta}{A}

% Lógica condicional para exibição
\ifdefstring{\atividade}{lista}{%
    \alternativas
    \vspace{0.5em}
    
    \noindent\textbf{Resposta:} letra \textbf{\resposta}.
}{%
    \ifdefstring{\modo}{objetivo}{%
        \alternativas
    }{%
        % modo = discursiva → não mostra alternativas
    }
}
\end{exercicioBanco}


    \noindent\rule{\linewidth}{1pt}  % linha horizontal fina
    
\end{document}
%*************** FINAL DO DOCUMENTO ***************
