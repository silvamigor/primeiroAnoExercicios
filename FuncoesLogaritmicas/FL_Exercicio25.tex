%%%%%%%%%%%%%%%%%%%%%%%%%%%%%%%%%
%
%       EXERCÍCIO
%
%%%%%%%%%%%%%%%%%%%%%%%%%%%%%%%%%

\ifdefstring{\atividade}{prova}{%
    \ifdefstring{\modo}{objetivo}{%
        \renewcommand{\valorquestao}{\ValorQObj\ ponto}
    }{%
        \renewcommand{\valorquestao}{\ValorQDisc\ pontos}
    }%
}%

\begin{exercicioBanco}[\valorquestao]
Determine o conjunto solução da equação
%
\[
    \big(\log_{3}(x)\big)^{2} = 2 + \log_{9}(x^{2}).
\]
%

% Define as alternativas
\newcommand{\alternativas}{%
    \begin{center}
        \begin{tabularx}{\textwidth}{XXXXX}
            (a) \(\{\frac{1}{4}, \frac{2}{3}\}\). &
            (b) \(\{\frac{1}{2}, 16\}\). &
            (c) \(\{\frac{1}{3}, 9\}\). &
            (d) \(\{-1,4\}\). &
            (e) \(\{0,1\}\).
        \end{tabularx}
    \end{center}
}

% Define a resposta correta
\newcommand{\resposta}{C}

% Lógica condicional para exibição
\ifdefstring{\atividade}{lista}{%
    \alternativas
    \vspace{0.5em}
    
    \noindent\textbf{Resposta:} letra \textbf{\resposta}.
}{%
    \ifdefstring{\modo}{objetivo}{%
        \alternativas
    }{%
        % modo = discursiva → não mostra alternativas
    }
}
\end{exercicioBanco}

