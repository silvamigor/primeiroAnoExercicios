%%%%%%%%%%%%%%%%%%%%%%%%%%%%%%%%%
%
%       EXERCÍCIO
%
%%%%%%%%%%%%%%%%%%%%%%%%%%%%%%%%%

\ifdefstring{\atividade}{prova}{%
    \ifdefstring{\modo}{objetivo}{%
        \renewcommand{\valorquestao}{\ValorQObj\ ponto}
    }{%
        \renewcommand{\valorquestao}{\ValorQDisc\ pontos}
    }%
}%

\begin{exercicioBanco}[\valorquestao]
Sejam \(\log(5) = m\), \(\log(2) = p\) e \(N = 125\sqrt[\raisebox{0.7ex}{$\scriptscriptstyle 3$}]{\dfrac{1562,5}{\sqrt[5]{2}}}\). O valor de \(\log_{5}(N)\) em função de \(m\) e \(p\) é:

% Define as alternativas
\newcommand{\alternativas}{%
    \begin{center}
        \begin{tabularx}{\textwidth}{XXXXX}
            (a) \(\dfrac{75m + 6p}{15m}\). &
            (b) \(\dfrac{70m - 6p}{15m}\). &
            (c) \(\dfrac{75m - 6p}{15m}\). &
            (d) \(\dfrac{70m + 6p}{15m}\). &
            (e) \(\dfrac{70m + 6p}{15p}\).
        \end{tabularx}
    \end{center}
}

% Define a resposta correta
\newcommand{\resposta}{B}

% Lógica condicional para exibição
\ifdefstring{\atividade}{lista}{%
    \alternativas
    \vspace{0.5em}
    
    \noindent\textbf{Resposta:} letra \textbf{\resposta}.
}{%
    \ifdefstring{\modo}{objetivo}{%
        \alternativas
    }{%
        % modo = discursiva → não mostra alternativas
    }
}
\end{exercicioBanco}

