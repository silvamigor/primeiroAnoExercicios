%%%%%%%%%%%%%%%%%%%%%%%%%%%%%%%%%
%
%       EXERCÍCIO
%
%%%%%%%%%%%%%%%%%%%%%%%%%%%%%%%%%

\ifdefstring{\atividade}{prova}{%
    \ifdefstring{\modo}{objetivo}{%
        \renewcommand{\valorquestao}{\ValorQObj\ ponto}
    }{%
        \renewcommand{\valorquestao}{\ValorQDisc\ pontos}
    }%
}%

\begin{exercicioBanco}[\valorquestao]
Em 2011, um terremoto de magnitude \(9{,}0\) na escala Richter causou um devastador tsunami no Japão, provocando um alerta na usina nuclear de Fukushima. Em 2013, outro terremoto, de magnitude \(7{,}0\) na mesma escala, sacudiu Sichuan (sudoeste da China), deixando centenas de mortos e milhares de feridos. A magnitude de um terremoto na escala Richter pode ser calculada por
\[
M = \frac{2}{3}\,\log\!\left(\frac{E}{E_0}\right),
\]
sendo \(E\) a energia, em kWh, liberada pelo terremoto e \(E_0\) uma constante real positiva. Considere que \(E_1\) e \(E_2\) representam as energias liberadas nos terremotos ocorridos no Japão e na China, respectivamente. 

Qual a relação entre \(E_1\) e \(E_2\)?


% Define as alternativas
\newcommand{\alternativas}{%
    \begin{center}
        \begin{tabularx}{\textwidth}{XXXXX}
            (a) \(E_1 = E_2 + 2\). &
            (b) \(E_1 = 10^{2}\cdot E_2\). &
            (c) \(E_1 = 10^{3}\cdot E_2\). &
            (d) \(E_1 = 10^{\tfrac{9}{7}}\cdot E_2\). &
            (e) \(E_1 = 10^{\tfrac{9}{7}}\cdot E_2\).
        \end{tabularx}
    \end{center}
}

% Define a resposta correta
\newcommand{\resposta}{C}

% Lógica condicional para exibição
\ifdefstring{\atividade}{lista}{%
    \alternativas
    \vspace{0.5em}
    
    \noindent\textbf{Resposta:} letra \textbf{\resposta}.
}{%
    \ifdefstring{\modo}{objetivo}{%
        \alternativas
    }{%
        % modo = discursiva → não mostra alternativas
    }
}
\end{exercicioBanco}

