%%%%%%%%%%%%%%%%%%%%%%%%%%%%%%%%%
%
%       EXERCÍCIO
%
%%%%%%%%%%%%%%%%%%%%%%%%%%%%%%%%%

\ifdefstring{\atividade}{prova}{%
    \ifdefstring{\modo}{objetivo}{%
        \renewcommand{\valorquestao}{\ValorQObj\ ponto}
    }{%
        \renewcommand{\valorquestao}{\ValorQDisc\ pontos}
    }%
}%

\begin{exercicioBanco}[\valorquestao]
Resolva o sistema
%
\[
    \left\{
    \begin{array}{rcl}
        a + b &=& 20, \\[3pt]
        \log(a) + \log(b) &=& 2.
    \end{array}
    \right.
\]
%

% Define as alternativas
\newcommand{\alternativas}{%
    \begin{center}
        \begin{tabularx}{\textwidth}{XXX}
            (a) \(a = 0\) e \(b = -10\). &
            (b) \(a = -10\) e \(b = -10\). &
            (c) \(a = 10\) e \(b = 10\). \\[5pt]
            (d) \(a = -10\) e \(b = 0\). &
            (e) \(a = 0\) e \(b = 10\). &
        \end{tabularx}
    \end{center}
}

% Define a resposta correta
\newcommand{\resposta}{C}

% Lógica condicional para exibição
\ifdefstring{\atividade}{lista}{%
    \alternativas
    \vspace{0.5em}
    
    \noindent\textbf{Resposta:} letra \textbf{\resposta}.
}{%
    \ifdefstring{\modo}{objetivo}{%
        \alternativas
    }{%
        % modo = discursiva → não mostra alternativas
    }
}
\end{exercicioBanco}

